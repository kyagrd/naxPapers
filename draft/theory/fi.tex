\section{\Fi}
\Fi\ is an extension of \Fw\ with indexed types, but less powerful than
the Calculus of Constructions (CC). That is, all typeable terms in \Fi\ are
typeable terms in \Fi\ and all typable terms in \Fi\ are typable terms in CC.
Since CC is normalizing, normalization of \Fi\ is automatic.
We describe \Fw\ and CC as an instances of the Pure Type System (PTS),
and then describe \Fi\ as an extension of \Fw\ and a restriction of CC.

PTS typing rules
\[
 \inference[Ax]{|-\Gamma & (s_1,s_2)\in\calA}{\Gamma |- s_1 : s_2}
 ~~~~
 \inference[Var]{|-\Gamma & (x:A)\in\Gamma}{\Gamma |- x:A} 
 ~~~~
 \inference[Conv]{\Gamma |- a:A & \Gamma |- B:s & A=B}{\Gamma |- a:B}
\]
\[
 \inference[Pi]{\Gamma |- A:s_1 & \Gamma,x:A |- B:s_2 & (s_1,s_2,s_3)\in\calR}
               {\Gamma |- (x:A)\to B : s_3}
\]
\[
 \inference[Lam]{\Gamma |- (x:A)\to B : s & \Gamma,x:A |- b:B}
                {\Gamma |- \lambda x:A.b : (x:A)\to B}
 ~~~~
 \inference[App]{\Gamma |- b:(x:A)\to B & \Gamma |- a:A}{\Gamma |- b\;a:[a/x]B}
\]


\[
 \inference{}{|-\cdot}
 ~~~~
 \inference{x\notin\dom(\Gamma) & |-\Gamma}{|-\Gamma,x:A}
\]

PTS is instantiated by the triple $(\calS,\calA,\calR)$ where
$\calS$ is the set of sorts ($s,s_1,s_2,s_3\in\calS$ in the PTS typing rules),
$\calA\subset\calS\times\calS$ is the set of axioms (used in Ax rule), and
$\calR\subset\calS\times\calS\times\calS$ is the set of rules (used in Pi rule).

\Fw\ is an instance of PTS where
\[\calS=\{*,\square\}\]
\[\calA=\{(*,\square)\}\]
\[\calR=\{(*,*,*),(\square,*,*),(\square,\square,\square)\}\]
Each element in $\calR$ stands for certain ability of dependency:
$(*,*,*)$ enables functions at the value level
(values depending on values)
$(\square,*,*)$ enables parametric polymorphism
(values depending on types)
$(\square,\square,\square)$ enables functions at the type level
(types depending on types).

CC extends \Fw\ with one additional rule
$(*,\square,\square)$, which enalbes types depending on values
(sometiles called \emph{value dependency}).
That is, CC is an instance of PTS where
\[\calS=\{*,\square\}\]
\[\calA=\{(*,\square)\}\]
\[\calR=\{(*,*,*),(\square,*,*),(\square,\square,\square)
         ,(*,\square,\square)\}\]

\Fi\ is also an extension of \Fw\, but a restriction of CC.  Instead of
allowing full value dependency (given by $(*,\square,\square)$ in CC),
we only allow value dependency partially via special syntax as follows:
\[
 \inference[Let]{\Gamma |- A:s & \Gamma, x:A |- b:B}
                {\Gamma |- \mathsf{let}~x:A=a~\mathsf{in}~b : [a/x]B}
 ~~~~
 \inference[Pi$_{(*,\square,\square)}$]
     {\Gamma |- A:* & \Gamma |- B:\square}
     {\Gamma |- (x:A)\to B : \square}
\]
The Let rule can be derived in CC by combining Pi rule and App rule
instantiated by $(*,\square,\square)$. However, we only allow
value dependency $(*,\square,\square)$ in this limited form, but
disallow arbitrary labmbda abstractions to form value dependent functions
by limiting the Pi rule.

Normalization of \Fi\ is automatic since \Fi\ is a strict subset of CC,
which is known to be strongly normalizing.

