\section{Introduction}

We have designed and implemented a prototype of Nax, which is
a strongly normalizing functional language supporting following features
(to be elaborated in \S\ref{sec:bg}):
\begin{description}
\item[Indexed types with static term indices]
Both types and terms can appear as indices in Nax.
We focus on term indices since type indices are rather mild extension to
the type system compared to term indices.
Indices are static in Nax. That is, indices are only used for type checking
but computationally irrelevant. For instance, using length indexed lists
should be no less efficient than using ordinary lists without indices in Nax.

\item[Recursive types of unrestricted polarity but restricted elimination]
It is well known that unrestricted recursive types enable diverging computation
even without any recursion at term level. To design a normalizing language
that support recursive types, we should make a design choice that limits
the ability of recursive types in one way or another. There are two possible
design choices. We may restrict either the formation of recursive types
(\ie, type definition) or the elimination of recursive types
(\ie, pattern matching). We choose the latter for Nax.

\item[Mendler style iteration and recursion combinators]
Any practically useful normalizing language needs principled recursion
combinators, which allow limited forms of recursion guaranteed to normalize.
Since Nax allows indexed types, we need recursion combinators that accommodate
to arbitrary indices. Since Nax allows arbitrary recursive type formation,
we need recursion combinators that guarantee normalization for arbitrary
recursive types. Mendler style combinators meet thees needs.

\item[Type inference (or, reconstruction) from minimal annotation]
When we extend the Hindley-Milner type system with indexed types, we no longer
have type inference from completely unannotated terms. Although complete
type inference is not possible, partial type inference (or, reconstruction) is
still possible when sufficient amount of type annotations is provided.
The Nax syntax systematically requires type annotations on necessary places
that are sufficient to infer types only from those annotations.
\end{description}

\paragraph{Write about why Nax matters}
make out some interesting story out of above:
each of the features have been studied in one way or another,
but what are we doing more or better or fun?

