%-----------------------------------------------------------------------------
%
%               Template for sigplanconf LaTeX Class
%
% Name:         sigplanconf-template.tex
%
% Purpose:      A template for sigplanconf.cls, which is a LaTeX 2e class
%               file for SIGPLAN conference proceedings.
%
% Guide:        Refer to "Author's Guide to the ACM SIGPLAN Class,"
%               sigplanconf-guide.pdf
%
% Author:       Paul C. Anagnostopoulos
%               Windfall Software
%               978 371-2316
%               paul@windfall.com
%
% Created:      15 February 2005
%
%-----------------------------------------------------------------------------


\documentclass[preprint]{sigplanconf}

% The following \documentclass options may be useful:
%
% 10pt          To set in 10-point type instead of 9-point.
% 11pt          To set in 11-point type instead of 9-point.
% authoryear    To obtain author/year citation style instead of numeric.

\usepackage[fleqn]{amsmath}
\usepackage{amssymb}
\usepackage{amsthm}
\usepackage{semantic}
\usepackage{color}
\usepackage{fancybox}
\usepackage{framed}
\usepackage{listings}
\usepackage{todonotes}
%% \usepackage{comment}

\newcommand{\KYA}[1]{\todo[inline,author=KiYung,size=\small,color=pink]{#1}}

\definecolor{grey}{rgb}{0.8,0.8,0.8}

\newcommand{\Fig}[1]{Figure\,\ref{fig:#1}}
\newcommand{\newFi}[1]{\colorbox{grey}{\ensuremath{#1}}}

\newcommand{\eg}{{e.g.}}
\newcommand{\ie}{{i.e.}}

\newcommand{\Fi}{\ensuremath{\mathsf{F}_i}}
\newcommand{\Fw}{\ensuremath{\mathsf{F}_\omega}}
\newcommand{\fix}{\mathsf{fix}}
\newcommand{\Fix}{\mathsf{Fix}}
\newcommand{\Fixw}{\ensuremath{\Fix_{\omega}}}
\newcommand{\Fixi}{\ensuremath{\Fix_{i}}}

\renewcommand{\l}{\lambda}

\newcommand{\dom}{\mathop{\mathsf{dom}}}
\newcommand{\FV}{\mathop{\mathrm{FV}}}

\theoremstyle{plain}
\newtheorem{proposition}{Proposition}
\newtheorem*{proposition*}{Proposition}
\newtheorem{theorem}{Theorem}
\newtheorem{lemma}{Lemma}
\newtheorem{corollary}{Corollary}

\theoremstyle{remark}
\newtheorem{remark}{Remark}

\theoremstyle{definition}
\newtheorem{definition}{Definition}



\begin{document}

\conferenceinfo{WXYZ '05}{date, City.} 
\copyrightyear{2005} 
\copyrightdata{[to be supplied]} 

\titlebanner{banner above paper title}        % These are ignored unless
\preprintfooter{short description of paper}   % 'preprint' option specified.

\title{System \Fi}
\subtitle{a higher-order polymorphic $\lambda$-calculus with erasable term indices}

\authorinfo{Ki Yung Ahn\and Tim Sheard}
           {Portland State University}
	   {kya@cs.pdx.edu\and sheard@cs.pdx.edu}

\authorinfo{Marcelo Fiore\and Andrew M. Pitts}
           {University of Cambridge}
	   {\{Marcelo.Fiore,Andrew.Pitts\}@cl.cam.ac.uk}

\maketitle

\begin{abstract}
We describe System \Fi, a polymorphic lambda calculus, which supports
term-indexed datatypes, such as length-indexed lists. System \Fi\ is
the smallest extension to System \Fw\ we could devise that can
express term indices. Term-indexed datatypes are expressible
in richer systems, such as the Implicit Calculus Constructions (ICC),
but such systems come coupled with orthogonal features
such as large eliminations and dependent types.
We argue there are important pedagogical benefits of isolating the minimal
features to support term-indexing. We relate System \Fi to other
known systems. We prove erasure properties of \Fi\ which captures the idea
that term indices are erasable, since they are irrelevant for reduction.
Erased \Fi\ terms are \Fw\ terms, so \Fi inherits
the strong normalization property from System \Fw.
We also show that System \Fi\ is logically consistent
by observing that System \Fi\ is embeddable into a subset of ICC,
which is known to be logically consistent.
We argue that System \Fi\ provides a theory for analyzing term-indexed
Generalized Algebraic Data Structures and could provide the basis
for the design of a light-weight logically-sound system, providing some of
the benefits of logical systems based upon dependent calculi,
but with much less complexity.
\end{abstract}

\category{D.3.3}{Programming Languages}{Language Constructs and Features}[data types and structures]
\category{F.4.1}{Mathematical Logic and Formal Languages}{Mathematical Logic}[lambda calculus and related systems]

\terms
Languages, Theory

\keywords
term-indexed data types, generalized algebraic data types,
higher-order polymorphism, type constructor polymorphism, higher-kinded types,
impredicative encoding

\chapter{Introduction}\label{ch:intro}

In this dissertation, we contribute to answering the question:
``how does one build a seamless system where programmers can both
write (functional) programs and formally reason about those programs''.
We set the scope of our research by clarifying what we mean by
\emph{programming} and \emph{formal reasoning}, and how they are related
(\S\ref{sec:intro:scope}). We motivate our research by describing the gap
between the designs of functional programming languages and
formal reasoning systems (\S\ref{sec:intro:motiv}), and assert
our thesis (\S\ref{sec:intro:motiv}).

TODO

preliminary concepts (\S\ref{sec:intro:concepts})

contributions (\S\ref{sec:intro:contrib})

methodology and overview (\S\ref{sec:intro:overview})

TODO

\section{Programming and Formal Reasoning}\label{sec:intro:scope}

\subsection{Scope of our research}
In our research, \emph{programming} refers to writing programs in
modern functional programming languages (\eg, Haskell, ML) that support
(recursive) datatypes, higher-order functions, (parametric) polymorphism,
and type inference. When we say \emph{functional programming languages},
\emph{functional languages}, or \emph{programming languages},
we mean such modern functional programming languages. We do not consider,
so called, dynamically typed functional languages (\eg, Lisp, Erlang)
in this dissertation.

\emph{Formal reasoning}, or \emph{logical reasoning},
refers to constructing proofs with dependently-typed formal reasoning systems
(\eg, Coq, Agda) that support (inductive) datatypes, higher-order functions,
polymorphism (\ie, type arguments to type constructors), and, of course,
dependent types (\ie, term arguments to type constructors). When we say
\emph{formal reasoning systems}, \emph{logical reasoning systems},
or, simply \emph{reasoning systems}, we mean such dependently-typed
formal reasoning systems.

Functional languages and reasoning systems are closely related.
Proofs in reasoning systems are similar in structure to programs
in functional languages. For example, the Haskell program \textit{id},
which computes a value of type $A$ when given a value of type $A$,
and, the Agda proof \textit{id}, which proves that the proposition $A$
is true when given a proof of $A$, are very similar:
\begin{center}
\begin{singlespace}
\begin{minipage}{.4\linewidth}
	\textsc{Haskell} \vspace*{.5em} \\
\textit{id} $::$ $A -> A$ \\
\textit{id} = $\lambda x -> x$
\end{minipage}
\begin{minipage}{.4\linewidth}
	\textsc{Agda}  \vspace*{.5em} \\
\textit{id} : $A -> A$ \\
\textit{id} = $\lambda x -> x$
\end{minipage}
\end{singlespace}
\end{center}\vspace*{.5em}
Such similarities are not accidental but intended by the design of
reasoning systems, based on the observation that proofs corresponds
to programs and propositions corresponds to types. This observation
is known as the Curry--Howard correspondence \cite{Howard69}.
The \emph{reasoning systems} we consider in this dissertation
are the ones that are based on the Curry--Howard correspondence.
We will explain the basic concepts to understand this correspondence
in the following subsection.

We assume that readers are familiar with \emph{programming}
in \emph{functional languages} and some basic concepts of lambda calculus
(\eg, $\beta$-reduction, normal forms). We assume that the readers have
basic knowledge on logic (\eg, concepts of axioms, inference, and proofs)
but not necessary familiar with \emph{formal reasoning}
in \emph{reasoning systems}.

\subsection{The Curry--Howard correspondence}
In late 1960s, \citet{Howard69} observed that intuitionistic natural deduction,
which is a proof system of a formal logic, and a typed lambda calculus,
which is a model of computation, are directly related. This relation,
known as the Curry--Howard correspondence, is established as follows:
\begin{itemize}
\item A proposition in the natural deduction corresponds to
	a type in the lambda calculus.
\item A proof for the proposition corresponds to
	a term of the type, to which that proposition corresponds.
\item Simplification of proofs correspond to computation,
	that is, simplification of terms.
\end{itemize}
We explain this correspondence with a very simple
version of intuitionistic natural deduction and
a simply-typed lambda calculus (Figure\;\ref{fig:nd}).\footnote{
	The readers who are familiar to the literature on natural deduction
	would notice that the left column in Figure\;\ref{fig:nd} is not
	the same style as the natural deduction formalized by
	\citet{Gentzen35,Gentzen69}. It has a has the flavours of
	both natural deduction and the sequent calculus. The context $\Gamma$
	are part of the syntax of judgments as in sequent calculus,
	but rather than the very syntactic structural rules (weakening,
	contraction, permutation) of sequent calculus, we simply rely on
	$A^x \in \Gamma$ to use the hypothetical propositions from $\Gamma$.
	We choose to formalize natural deduction this way to emphasize
	the structural similarities to the typical formalization
	for typing rules of lambda calculi.}

Traditionally, in Hilbert-style deduction, a formal logic is formalized
by a minimal set of inference rules (\eg, modus ponens
$\inference{\Gamma |- A -> B & \Gamma |- A}{\Gamma |- B}$)
and a set of axiom schemes (\eg, $A -> A$ and $A -> (B -> A)$ where
the meta-variables $A$ and $B$ can be instantiated to arbitrary propositions).
\emph{Natural deduction}, on the other hand, is another style of
formalizing logic mostly by a set of inference rules and a minimal
(often empty) set of axiom schemes.

There are many variations of formal logic based on natural deduction
(and also its corresponding typed lambda calculus) depending on
the set of logical connectives, constants, and quantifiers they support.
We illustrate a intuitionistic natural deduction with implication ($->$) and
falsity ($\bot$), and, its corresponding simply-typed lambda calculus (STLC)
in Figure\;\ref{fig:nd}. The implication and the falsity in natural deduction
corresponds to the function type ($->$) and the void type ($\bot$) in the STLC.
In the inference rules (\rulename{Ax}) and (\rulename{$->_\textsc{I}$}),
$x$ on $A^x$ is a meta-tag to distinguish between possibly multiple
occurrences of $A$ in $\Gamma$ (\eg, $\Gamma = A^{x_1},B^{x_2},A^{x_3}$).

\begin{figure}
\[
\begin{array}{ccc}
	\text{Intuitionistic natural deduction} & &
	\text{Typing rules of STLC} \\ \\
\inference{A^x \in \Gamma}{\Gamma |- A} & (\textsc{Ax}) &
\inference{x:A \in \Gamma}{\Gamma |- A} \\ \\
\inference{\Gamma,A^x |- B}{\Gamma |- A -> B} & (->_\textsc{I}) &
\inference{\Gamma,x:A |- B}{\Gamma |- \l x.t : A -> B} \\ \\
\inference{\Gamma |- A -> B & \Gamma |- A}{\Gamma |- B} & (->_\textsc{E}) &
\inference{\Gamma |- t : A -> B & \Gamma |- s : A}{\Gamma |- t\;s : B} \\ \\
\inference{\Gamma |- \bot}{\Gamma |- A} & (\bot_\textsc{E})  &
\inference{\Gamma |- t:\bot}{\Gamma |- \texttt{elim}_\bot\,t : A}
\end{array}
\]
\caption{Intuitionistic natural deduction with implication and falsity,
	and its corresponding simply-typed lambda calculus.}
\label{fig:nd}
\end{figure}

Note that each typing rule (in the right column) has exactly the same structure
as its corresponding inference rule (in the left column), except for
the variables and terms appearing on the right-hand side of the colon
(\eg, $x$ in $x:A$ and $t$ in $t:A -> B$). So, a type correct term,
which is justified by the derivation following the typing rules of
the lambda calculus, captures the structure of its corresponding proof
by natural deduction. For instance, $(\l x. x)$ is a term that captures
the structure of a proof for $A -> A$.
For this reason, such type correct terms in reasoning systems
are called proof terms, proof objects, or simply proofs.

We explained that proofs and propositions in the natural deduction
corresponds to terms and types in the lambda calculus. Lastly,
we need to show the correspondence between simplification of proofs
and simplification of terms (\ie, computation) in order to establish
the Curry--Howard correspondence between the natural deduction
and the lambda calculus.

There can be multiple proofs for the same proposition, and,
correspondingly, there can be more than one term of the same type.
Some of these terms are closely related while others are rather independent.
For example, $\l x_1.\l x_2.x_1$ and $\l x_1.\l x_2.x_2$ are rather 
independent terms that inhabit the same type $A -> A -> A$.
The typing derivations for these terms (right column) and
its corresponding proofs (left column) are illustrated in
Figure\;\ref{fig:proofAAA}.
\begin{figure}
\[\setpremisesend{.1em}
\inference[$(->_\textsc{I})^{x_1}$]
	{\hspace*{-3em}
	 \inference[$(->_\textsc{I})^{x_2}$]
	 	{\hspace*{-3em}
		 \inference[$(\textsc{Ax})^{x_1}$]
			{A^{x_1}\in A^{x_1},A^{x_2}}
			{A^{x_1},A^{x_2} |- A} }
		{A^{x_1} |- A -> A} }
	{|- A -> A -> A }
\qquad
\inference[($->_\textsc{I}$)]
	{\hspace*{-1.3em}
	 \inference[($->_\textsc{I}$)]
		{\hspace*{-1.5em}
		 \inference[(\textsc{Ax})]
			{x_1 : A \in x_1 : A, x_2 : A}
			{x_1 : A, x_2 : A |- x_1 : A} }
		{x_1 : A |- \l x_2.x_1 : A -> A} }
	{|- \l x_1.\l x_2.x_1 : A -> A -> A }
\]
\[\setpremisesend{.1em}
\inference[$(->_\textsc{I})^{x_1}$]
	{\hspace*{-3em}
	 \inference[$(->_\textsc{I})^{x_2}$]
		{\hspace*{-3em}
		 \inference[$(\textsc{Ax})^{x_2}$]
			{A^{x_2}\in A^{x_1},A^{x_2}}
			{A^{x_1},A^{x_2} |- A} }
		{A^{x_1} |- A -> A} }
	{|- A -> A -> A }
\qquad
\inference[($->_\textsc{I}$)]
	{\hspace*{-1.3em}
	 \inference[($->_\textsc{I}$)]
		{\hspace*{-1.5em}
		 \inference[(\textsc{Ax})]
			{x_2 : A \in x_1 : A, x_2 : A}
			{x_1 : A, x_2 : A |- x_2 : A} }
		{x_1 : A |- \l x_2.x_2 : A -> A} }
	{|- \l x_1.\l x_2.x_2 : A -> A -> A }
\]
\caption{Typing derivations (right) for terms of type $A -> A -> A$
	and their corresponding proofs (left).}
\label{fig:proofAAA}
\end{figure}

On the other hand, there are closely related terms of the same type.
For example, $(\l x_1.x_2) x_3$ reduces to $x_2$ by a $\beta$-reduction step
(\ie, $(\l x_1.x_2) x_3 -->_\beta x_2$). A term like $x_2$, which does not
($\beta$-)reduce any further, is called a ($\beta$-)\emph{normal form} or
a ($\beta$-)\emph{normal term}. \citet{Pra65} established a notion of
reduction and normalization for natural deduction. One can reduce a proof
when there is a consecutive use of introduction and elimination rules.
An introduction rule introduces a certain form of propositions
in the conclusion (below the horizontal bar), and, an elimination rule
uses propositions that form in the premises (above the horizontal bar).
The rules (\rulename{$->_\textsc{I}$}) and (\rulename{$->_\textsc{E}$})
are introduction and elimination rules for implication ($->$).
So, we need to show the correspondence between reduction of proofs
over implication in the natural deduction and ($\beta$-reduction) over
type correct terms in the lambda calculus,
in order to establish the Curry--Howard correspondence.

Figure\;\ref{fig:proofreduce} illustrates that
the reduction of a proof involving consecutive uses of
(\rulename{$->_\textsc{I}$}) and (\rulename{$->_\textsc{E}$})
corresponds to the $\beta$-reduction of well-typed terms.\footnote{
	Introduction and elimination rules for ($->$) consecutively used
	in the other order corresponds to $\eta$-reduction
	(\ie, $(\l x. t\;x) -->_\eta t$ where $x$ does not appear free in $t$).
	In this dissertation, we only consider $\beta$-reduction.}
To redcuce the proof, we replace the uses of $A^x$ in the premise of
$(->_\textsc{I})^x$ with the derivations $D'$, which deduces $A$ from
the original context $\Gamma$, so that we can remove $A^x$ from
the left-hand sides of the turstile ($|-$) throughout the proof.
The change of subscripts from $\mathcal{D}_{[\Gamma,A^x]}$ to
$\mathcal{D}_{[\Gamma]}$, before and after the reduction,
denotes that we consistently removed $A^x$ from the context
(\ie, left-hand sides of $|-$). We leave it as an exercise for
the readers to construct a corresponding proof for the reduction
$(\l x_1.x_2) x_3 -->_\beta x_2$ from the previous paragraph
(\textit{hints}: Start with $\Gamma = x_2:A,x_3:B$).

\begin{figure}
Reducing a proof involving consecutive uses of
(\rulename{$->_\textsc{I}$}) and (\rulename{$->_\textsc{E}$}):
\[
\inference[($->_\textsc{E}$)]
	{\hspace*{-3em}
	 \inference[$(->_\textsc{I})^{x}$]
		{\hspace*{-1em}
		 \inference*
			{\hspace*{-4em}
			 \inference[$(\textsc{Ax})^{x}$]
				{A^x\in\Gamma,A^x,\Gamma'}
				{\Gamma,A^x,\Gamma' |- A} \\
			 \vdots \\
			 \mathcal{D}_{[\Gamma,A^x]} }
			{\Gamma,A^x |- B} }
		{\Gamma |- A -> B}
	&\inference*{\mathcal{D}'_{[\Gamma]}}{\Gamma |- A} }
	{\Gamma |- B}
\quad
\longrightarrow
~
\inference*
	{\inference*
		{\mathcal{D}'_{[\Gamma,\Gamma']}}
		{\Gamma,\Gamma' |- A} \\
	 \vdots_{\phantom{G}} \\
	 \mathcal{D}_{[\Gamma]} }
	{\Gamma |- B}
\]
Reduction of a well-typed term along with its typing derivation:
\[
\inference[($->_\textsc{E}$)]
	{\hspace*{-3em}
	 \inference[$(->_\textsc{I})$]
		{\hspace*{-1em}
		 \inference*
			{\hspace*{-4em}
			 \inference[$(\textsc{Ax})$]
				{x:A\in\Gamma,x:A,\Gamma'}
				{\Gamma,x:A,\Gamma' |- x:A} \\
			 \vdots \\
			 \mathcal{D}_{[\Gamma,x:A]} }
			{\Gamma,x:A |- t:B} }
		{\Gamma |- \l x.t : A -> B}
	&\inference*{\mathcal{D}'_{[\Gamma]}}{\Gamma |- s:A} }
	{\Gamma |- (\l x.t)\,s:B}
\quad
\longrightarrow_\beta
~
\inference*
	{\inference*
		{\mathcal{D}'_{[\Gamma,\Gamma']}}
		{\Gamma,\Gamma' |- s:A} \\
	 \vdots_{\phantom{G}} \\
	 \mathcal{D}_{[\Gamma]} }
	{\Gamma |- t[s/x]:B}
\]
\caption{Reduction of a proof involving introduction and elimination rules
	for implication, and its corresponding $\beta$-reduction 
	of a well-typed term.}
\label{fig:proofreduce}
\end{figure}

We illustrated the Curry--Howard correspondence between
a minimal intuitionistic logic and its corresponding lambda calculus
In Figure\;\ref{fig:nd}. That is, a proposition, its proof, and
simplification of proofs (or, proof reduction) corresponds to
a term, its type, and simplification of terms (or, computation).
The first and second pieces of the corresponds, a proposition and
its proof corresponding to a term and its type, is quite obvious from
the structural similarity between the inference rules and the typing rules.
The last piece, correspondence between proof reduction and computation,
needs further analysis on the implication and elimination rules.
We illustrated the correspondence between proof reduction over implication
and $\beta$-reduction, which describes the computation over functions,
in Figure \ref{fig:proofreduce}. The correspondence between
the proof reduction over falsity and the computation over the void type
hold vacuously. There is no proof reduction and computation over $\bot$
since it lacks the introduction rule -- it only has the elimination rule
(\rulename{$\bot_\textsc{E}$}).

\section{Datatypes and recursion}
Programming languages support language constructs other than functions,
such as datatypes and polymorphism.

sums, products, aaaa

functional programming languages,
such as ML and Haskell,

are
logical connectives and 


TODO
$\forall$
$\exists$

What datatypes corresponds to
datatypes
In Figure\;\ref{fig:nd} has falsity ($\bot$), which corresponds to
a degenerative datatype (the void type), but




\section{Logical consistency and normalization}



if we know that every term is normalizing


datatypes and recursion



\section{Motivation}\label{sec:intro:motiv}
Since the observation of the Curry--Howard correspondence, logicians and
programming language researchers have dreamed of building a system in which
one can both write programs (\ie, model computation) and formally reason about
(\ie, construct proofs of) the properties (\ie, types) of those programs.

However, building a practical system that unifies programming and
formal reasoning, based on the Curry--Howard correspondence, is still
an open research problem. The gap between the conflicting
design goals of typed functional programming languages, such as ML and Haskell,
and formal reasoning systems, such as Coq and Agda, is still wide.

\begin{itemize}

\item
Programming languages are typically designed to achieve
computational expressiveness. They often sacrifice logical consistency
to achieve this goal. Programmers should be able to
conveniently express all possible computations, regardless of whether those
computations have a logical interpretation (by the Curry--Howard correspondence)
or not.

\item
Formal reasoning systems are typically designed to achieve logical consistency.
They often sacrifice computational expressiveness to achieve that goal.
Users expect that it is only possible to prove true propositions,
and it is impossible to prove falsity. They are willing to live with
the difficultly (or even inability) to express certain computations
within the reasoning system in order to achieve logical consistency.

\end{itemize}

As a result, the recursion schemes of programming languages and
formal reasoning systems differ considerably.
Programming languages provide unrestricted general recursion
to conveniently express computations that may or may not terminate.
Formal reasoning systems provide induction principles for sound reasoning,
or, in the computational view, principled recursion schemes
that can only express terminating computation.

The two different design goals also lead to significant differences
in their type system design as well.
Programming languages are based on \emph{recursive types},
which place only syntactic restrictions on the definition of new datatypes.
Programmers can express computations over a wide variety of datatypes.
In addition, most (statically typed) functional programing languages have
clear distinction between terms and types (\ie, terms do not appear in types).
Reasoning systems are usually based on \emph{inductive types},
which place semantic restrictions, only accepting datatype definitions
that support conventional\footnote{We will introduce
	what ``conventional'' style is in \S\ref{sec:intro:concpets:recursive}
	as opposed to the Mendler style in \S\ref{sec:intro:concepts:mendler}.
	} induction principles.
In addition, most reasoning systems, based on the Curry--Howard correspondence,
allow types to depend on terms (\ie, terms can appear in types) to specify
fine grained properties.

This dissertation explores a sweet spot where one can benefit from
the advantages of both programming languages and formal reasoning systems.
That is, we design a unified language system, called Nax, which is
logically consistent while being able to conveniently express
many useful computations. We do this by placing few restriction on type definitions,
as is done in programming languages, but also provide a rich set of
non-conventional recursion schemes (or, induction principles) that
always terminate. These non-conventional recursion schemes are known as
\emph{the Mendler style}. Another major design choice in Nax is
supporting \emph{term indices} in types, a middle ground, which sits between
polymorphic types and dependent types.

In the following section, we explain what we mean by the sweet spot between
programming languages and reasoning systems. Our thesis is that
the design choices we explain below are reasonable for achieving
the goal of combining programming languages and reasoning systems.

\section{Thesis}\label{sec:intro:thesis}
Whatever design choices we make, the sweet spot should have the following features.

\begin{enumerate}[(1)]
 \item \textbf{A convenient programming} style
         supported by the major constructs of
         modern functional programming languages: 
         parametric polymorphism, recursive datatypes,
         recursive functions, and type inference,
 \item \textbf{An expressive logic}
         that can specify fine-grained program properties using types, and terms that
         witness proofs of these properties 
         (the Curry--Howard correspondence),
 \item \textbf{A small theory} based upon a minimal foundational calculus that is
         expressive enough to support the programming features, expressive
         enough to embed propositions and proofs about
         programs, and logically consistent
         to avoid paradoxical proofs in the logic, and
 \item \textbf{A simple implementation} that keeps the trusted base small.
\end{enumerate}
We claim that a language design based on \emph{Mendler-style recursion schemes}
and \emph{term-indexed types} can lead to a system that supports these four
features.

\paragraph{}
From a bird's-eye view, the following chapters back up our claim as follows:
Mendler-style recursion schemes support (1) because they are based on
parametric polymorphism and well-defined over a wide range of datatypes.
Term-indexed types support (2), because they can statically track program
properties. For instance the size of data structures can be tracked by using
a natural number term in their types.
To support (3), we design several foundational calculi, each which extends
a well known polymorphic lambda calculus with term-indexed types.
Mendler-style recursion schemes also also support (4) because their
termination is type-based -- no need for an auxiliary termination checker.

In next section, we summarize important ideas mentioned in our thesis above.

\section{Preliminary concepts}\label{sec:intro:concepts}
We give summaries of the following preliminary concepts:
Curry--Howard correspondence (\S\ref{sec:intro:concepts:CH}),
Mendler-style recursion schemes
(\S\ref{sec:intro:concepts:CH}, \S\ref{sec:intro:concepts:mendler}),
and term-indexed types (\S\ref{sec:intro:concepts:indexed}).
Further details and historical backgrounds on each of these concepts
will appear in the following chapters (see \S\ref{sec:intro:overview}
for the overview of chapter organization).

\subsection{The Curry--Howard correspondence and Normalization}
\label{sec:intro:concepts:CH}
One promising approach to designing a system that unifies
logical reasoning and programming is \emph{the Curry--Howard correspondence}.
Howard \cite{Howard69} observed that a typed model of computation
(\ie, a typed lambda calculus) gives an interpretation to a (natural deduction)
proof system (for an intuitionistic logic). More specifically, one can interpret
a type (in the lambda calculus) as a formula (in the logic) and
a term of that type, as a proof for that formula. For instance,
the typing rule for function applications (APP) in a typed lambda calculus
corresponds to Modus Ponens (MP) in a logic:
\[ \inference[(APP)]{\Gamma |- t_1 : A -> B & \Gamma |- t_2 : A}{
        \Gamma |- t_1~t_2 : B}
 ~~~~~~~~
   \inference[(MP)]{A -> B & A}{B}
\]
As you can see above, combining terms ($t_1$ and $t_2$) to build a new term
($t_1~t_2$) can be interpreted as combining proofs for formulae
($A -> B$ and $A$), to construct a proof for a new formula ($B$).
More generally, we may expect that programming (\ie, building larger terms)
corresponds to constructing larger proofs, but only when the typed lambda calculi
meets certain standards -- \emph{type soundness} and \emph{normalization}.

The Curry--Howard correspondence is a promising approach to designing a
unified system for both logical reasoning and programming. Only one language
system is needed for both the logic and the programming language. An
alternate approach is to use an external logical language to talk about
programs as the objects that the logic reasons about. In this approach, one
has the obligation to argue that the soundness of the logic, with respect to
the programming language semantics, holds.

Under the Curry--Howard correspondence, the logic is internally related to the
semantics of program -- there is no need to argue for the soundness of the
logic,  externally outside of the programming language system. The soundness
of the logic follows directly from the type soundness of the language under
the Curry--Howard correspondence.

Let us consider a proposition to be true
(or, valid) when it has a canonical (\ie, cut-free) proof.
That is, there is a program, whose type is the proposition under
consideration, and that program has a normal form. 
By type soundness, any term,
of that type, will preserve its type during the reduction steps. Thus
reduction preserves truthfulness. If we assume
that the language is normalizing (\ie, every well-typed term reduces to
a normal form), then any term of that type which is a non-canonical proof,
implies the existence of a canonical proof, which in turn implies that
the proposition specified by the type is indeed true. That is, all provable
propositions are valid (\ie, the logic is sound) when the language is
\emph{type sound} and \emph{normalizing}.

\emph{Normalization} is also essential for the consistency of the logic.
For the lambda calculus to be interpreted as a \emph{consistent} logic,
there must be no diverging terms. A diverging term (\ie, a term that does
not have a normal form) may inhabit any arbitrary type. Thus, a diverging term
can be a proof for any proposition under the Curry--Howard correspondence.
General purpose functional programming languages (\eg, Haskell and ML), that
support unrestricted general recursion, cannot be interpreted as a consistent
logic, since they allow diverging terms (\ie, non-terminating programs).
For example, a diverging Haskell definition $\textit{loop} = \textit{loop}$
can be given an arbitrary type such as
$\textit{loop}\mathrel{::}\textit{Bool}$,
$\textit{loop}\mathrel{::}\textit{Int} -> \textit{Bool}$,
and even $\textit{loop}\mathrel{::}\forall a. a$, which is a proof of false.


Therefore, useful logical reasoning systems based on the Curry--Howard
correspondence (\eg, Coq, Agda) never support language features that can
lead to diverging terms. For example, in both Coq and Agda,
unrestricted general recursion (at term level) is not supported. 
Instead, these logical reasoning systems
often provide principled recursion schemes over recursive types that are
guaranteed to normalize. 

Recursive types (\ie, recursion at type level)
can also lead to diverging terms when they are not restricted carefully.
Many of the conventional logical reasoning systems, based on
Curry--Howard correspondence, restrict recursive types in a way,
which is not an ideal design choice, if one's goal is a unified system for
logic and programming. Our approach explores another design space not yet
completely explored. We introduce both approaches to restricting recursive
types to ensure normalization in the following two subsections.


\subsection{Restriction on recursive types for normalization}
\label{sec:intro:concpets:recursive}
We have argued that normalization is essential for logical reasoning systems
based on the Curry--Howard correspondence. One challenge to the successful
design of reasoning systems is how to restrict recursion at the type level
so that all well-typed terms have normal forms. There are two different
design choices illustrated in Figure~\ref{fig:approaches}. 
The conventional approach restricts the formation of recursive types
(\ie, the restriction is in datatype definition), and
the Mendler-style approach restricts the elimination of the values of
recursive types (\ie, the restriction is in pattern matching).

\begin{figure}
{\centering
\begin{tabular}{p{3cm}|p{12.5cm}}
\parbox{3cm}{~~Functional\\programming\\$~~~~$language} &
\parbox{12.5cm}{
 kinding:~
  \inference[($\mu$-form)]{\Gamma |- F : * -> *}{\Gamma |- \mu F : *} \\
 \\
 typing:\quad
  \inference[($\mu$-intro)]{\Gamma |- t : F (\mu F)}{\Gamma |- \In~t:\mu F} ~~~~
  \inference[($\mu$-elim)]{\Gamma |- t : \mu F}{\Gamma |- \unIn~t : F (\mu F)}\\
 \\
 reduction:
  \inference[(\unIn-\In)]{}{\unIn~(\In~t) \rightsquigarrow t}
} \\
\\ \hline\hline
\parbox{3cm}{$~$Conventional\\$~~~$approach for\\consistent logic} &
\parbox{12.5cm}{$\phantom{a}$\\
 kinding:~
  \inference[($\mu$-form$^{+}$)]{ \Gamma |- F : * -> * 
                           & \mathop{\mathsf{positive}}(F)}
                           {\Gamma |- \mu F : *} \\
 \\
 typing:~
  \text{{\small($\mu$-intro)} and {\small($\mu$-elim)}
                same as functional language} \\
  \[\inference[(\It)]{\Gamma |- t : \mu F & \Gamma |- \varphi : F A -> A}
                     {\Gamma |- \It~\varphi~t : A}\]
 reduction:~ \text{{\small(\unIn-\In)} same as functional language}
  \[\inference[(\It-\In)]{}{\It~\varphi~(\In~t) \rightsquigarrow
                            \varphi~(\textsf{map}_F~(\It~\varphi)~t)}\]
}
\\ \hline
\parbox{3cm}{Mendler-style\\$~~$approach for\\consistent logic} &
\parbox{12.5cm}{$\phantom{a}$\\
 kinding:~ \text{{\small($\mu$-form)} same as functional language} \\
 \\
 typing:~
  \text{{\small($\mu$-intro)} same as functional language}
  \[\inference[(\MIt)]
     { \Gamma |- t : \mu F &
       \Gamma |- \varphi : \forall X . (X -> A) -> F X -> A}
     {\Gamma |- \MIt~\varphi~t : A} \]
 reduction:~
  \inference[(\MIt-\In)]
     {}
     {\MIt~\varphi~(\In~t) \rightsquigarrow \varphi~(\MIt~\varphi)~t}
}
\end{tabular} }
\caption{Two different approaches to designing logical reasoning systems
         (in contrast to functional languages)}
\label{fig:approaches}
\end{figure}

\paragraph{Recursive types in functional programming languages.}
Let us start with a review of the theory of recursive types used
in functional programming languages. Here, the term
language is not expected to be normalizing, so restrictions are few.

Just as we can capture the essence of unrestricted general recursion at term
level, by a fixpoint operator (usually denoted by \textsf{Y} or \textbf{fix}),
we can capture the essence of recursive types by the
use of a recursive type operator, $\mu$, at type level. 
The rules for the formation {\small($\mu$-form)},
introduction {\small($\mu$-intro)}, and elimination {\small($\mu$-elim)} of
the recursive type operator $\mu$ are described in Figure \ref{fig:approaches}.
We also need a reduction rule {\small(\unIn-\In)}, which relates \In,
the data constructor for recursive types, and \unIn, the destructor for
recursive types, at the term level.

Surprisingly (if you hadn't known), the recursive type operator, $\mu$,
as described in Figure \ref{fig:approaches}, is already powerful enough to
express non-terminating programs, even without introducing the general recursive
{\em term} operator, \textbf{fix}, to the language. We illustrate this below.
First, here is a short reminder of how a fixpoint at the term level operates.
The typing rule and the reduction rule for \textbf{fix} can be given as follows:
\[ \text{typing:}~ \inference{\Gamma |- f : A -> A}{\textbf{fix}\,f : A}
 \qquad\qquad
   \text{reduction}:~ \textbf{fix}\,f \rightsquigarrow f(\textbf{fix}\,f)
\]
We can actually implement \textbf{fix}, using $\mu$, as follows
(using some Haskell-like syntax):
\begin{align*}
& \textbf{data}~T\;a\;r = C\;(r -> a) \quad
          \texttt{-}\texttt{-}~\text{\small a non-recursive datatype} \\
& w \,:\, \mu(T\;a) -> a ~~ \quad
          \texttt{-}\texttt{-}~\text{\small an encoding of the untyped
                                     $(\lambda x.x\;x)$
                                     in a typed language}~ \\
& w = \lambda x . \,\textbf{case}~\unIn~x~\textbf{of}~C\;f -> f\;x \\
& \textbf{fix} \,:\, (a -> a) -> a \quad
          \texttt{-}\texttt{-}~\text{\small an encoding of 
                                     $(\lambda f.(\lambda x.f(x\;x))\,
                                                 (\lambda x.f(x\;x)))$} \\
& \textbf{fix} = \lambda f. (\lambda x. f (w\;x))\,(\In(C(\lambda x. f (w\;x))))
\end{align*}

Thus, to avoid the loss of termination guarantees, we need to alter the rules
for $\mu$, in someways, to ensure a consistent logic. One way, is to restrict
the rule {\small $\mu$-form}; the other way, is to restrict the rule
{\small $\mu$-elim}. Once we decide which of these two alterations of the
rules we will use, the design of principled recursion combinators (\eg, \It\
for the former and \MIt\ for the latter) follows from that choice.

\paragraph{Positive (recursive) datatypes and negative (recursive) datatypes.}
Positive datatypes have recursion only in covariant positions.
For example, $\mu\,T_2$, where $\textbf{data}\;T_2 = C_2\,(Bool -> r)$,
is a positive datatype since the recursive argument $r$ in
the base structure $T_2$ only appears in the covariant position.
Recursive datatypes that have no function arguments are by default
positive datatypes. For instance, the natural number datatype $\mu\,N$,
where $\textbf{data}\,N\,r=S\,r \mid Z$, is a positive datatype.


Negative datatypes have recursion in contravariant positions.
Note that $\mu(T\;a)$ in the example above is a negative datatype
since the recursive argument $r$ in the base structure $T$ appears
in the contravariant position. Another example of a negative datatype is
$\mu\,T'$, where $\textbf{data}\;T'\;r = C'\,(r -> r)$, since $r$ in $T'$
appears in both contravariant and covariant positions.

\paragraph{Recursive types in the conventional approach to consistent logic.}
In the conventional approach, the formation (\ie, datatype definition) of
recursive types is restricted, but arbitrary elimination (\ie, pattern matching)
over the values of recursive types is allowed. In particular, the formation of
negative recursive types is restricted. Only positive recursive types are
supported. Thus, in Figure \ref{fig:approaches}, we have a restricted version of
the formation rule {\small($\mu$-form$^{+}$)} has an additional condition that
require $F$ to be positive. The other rules {\small($\mu$-intro)},
{\small($\mu$-elim)}, and {\small(\unIn-\In)} remain the same as for
functional languages. Since we have restricted the recursive types
at the type level and we do not have general recursion at the term level,
the language is indeed normalizing. However, we can neither write
interesting (\ie, recursive) programs that involves recursive types nor
inductively reason about those programs, unless we have principled recursion
schemes that are guaranteed to normalize. One such recursion scheme is called
iteration (\aka\ catamorphism). The typing rules for the conventional iteration
\It\ are illustrated in Figure \ref{fig:approaches}. Note, we have the typing
rule {\small(\It)} and the reduction rule {\small(\It-\In)} for \It\,
in addition to the rules for the recursive type operator $\mu$.

\paragraph{Recursive types in the Mendler-style approach to consistent logic.}
In the Mendler-style approach, we allow arbitrary formation
(\ie, datatype definition) of recursive types, but we restrict
the elimination (\ie, pattern matching) over the values of recursive types. 
The formation rule {\small($\mu$-form)} remains the same as
for functional languages. That is, we can define arbitrary recursive types,
both positive and negative. However, we no longer have the elimination
rule {\small($\mu$-elim)}. That is, we are not allowed to pattern match over
the values of recursive types in the normal fashion. We can only pattern match
over the values of recursive types through the Mendler-style recursion
combinators. The rules for the Mendler-style iteration combinator \MIt\
are illustrated in Figure \ref{fig:approaches}.
Note, there are no rules for \unIn\ in the Mendler-style approach.
The typing rule {\small($\mu$-elim)} is replaced by {\small(\MIt)} and
the reduction rule {\small(\unIn-\In)} is replaced by {\small(\MIt-\In)}.
More precisely, the typing rule {\small \MIt} is both an elimination rule
for recursive types and a typing rule for the Mendler-style iterator.
You can think of the rule {\small(\MIt)} as replacing both the elimination rule
{\small($\mu$-elim)} and the typing rule for conventional iteration
{\small(\It)}, but in a safe way that guarantees normalization.

\subsection{Justification of the Mendler-style as a design choice.}
\label{sec:intro:concepts:mendler}
We choose to base our approach to the design of a seamless synthesis of both
logic and programming on the Mendler-style. It restricts the elimination (\ie,
pattern matching) over values of recursive types, rather restricting the
formation (\ie, datatype definition) of recursive types (a more conventional
approach). The impact of this design choice is that it enables the logic to
include all datatype definitions that are used in functional programming
languages.

Functional programming promotes ``functions as first class values''.
It is natural to pass functions as arguments and embed functions into
(recursive) datatypes. If embedding functions in datatypes is allowed,
we can embed a function whose domain is the very type we are defining.
For example, the recursive datatype definition
$\mathbf{data}~T = C\;(T -> \textit{A})$ in Haskell is such a recursive
datatype definition. Such datatypes are called negative recursive datatypes
since the recursive occurrence $T$ appears in a negative position.
We say that $T$ is in a negative position, since $(T -> A)$ is analogous to
$(\neg T \land A)$ when we think of $->$ as a logical implication. There exist
both interesting and useful examples in functional programming involving
negative datatypes. In \S\ref{sec:msf}, we illustrate that
the Mendler-style recursion scheme we discovered can express
interesting examples involving negative datatypes.

Recall that the motivation of this dissertation research
(quoting again from \S\ref{sec:intro:motiv})
is to contribute to answering the question of ``how does one build a
seamless system where programmers can both write (functional) programs and
formally reason about those programs''. Under the Curry--Howard correspondence,
to formally reason about a program, the logic needs to refer to the type of
the program, since the type, interpreted as a proposition, describes
the property of the program. Since the Mendler-style approach does not
restrict recursive datatype definitions, we can directly refer to the types
of programs that use negative recursive datatypes.

The Mendler style is a promising approach to building a unified system because
all the recursive types (both positive and negative) are definable and
the recursion schemes over those types are normalizing.
Although the conventional approach is widely followed
in the design of formal reasoning systems (\eg, Coq, Agda), it cannot directly
refer to programs that use non-positive recursive types.One may object that
it is possible to indirectly model negative recursive types
in the conventional style, via alternative equivalent encodings
which map negative recursive types into positive ones. But, such
encodings do not align with our motivation towards a seamless unified
system for both programming and reasoning. It is undesirable to require
programmers to significantly change their programs just to reason about them.
If the change is unavoidable, it should be kept small. That is,
the changed program should syntactically resemble the original program,
which the programmer would usually write in a functional programming language.
In Chapter 3, we show a number of examples of programs written in
the Mendler style that look more close to the programs written using
general recursion than the programs written in the conventional style.

\subsection{Term-indexed types, type inference, and datatypes}
\label{sec:intro:concepts:indexed}
One of the most frequently asked questions about our design choices for Nax,
regarding term-indexed types, is ``why not dependent types?". Our answer
is that a moderate extension to the polymorphic calculus is a better candidate
than a dependently typed calculus as the basis for a practical programming
system. Recall, that we hope to design a unified system for programming
as well as reasoning. Language designs based on indexed types can
benefit from existing compiler technology and type inference algorithms
for functional programming languages. In addition, theories for
term-indexed datatypes are simpler than theories for full-fledged
dependent datatypes, because term-indexed datatypes can be encoded as
functions (using Church-like encodings).

The implementation technology for functional programming languages based on
polymorphic calculi is quite mature. There exist industrial
strength implementations, such as the Glasgow Haskell Compiler (GHC),
whose intermediate core language is an extension of \Fw.
Our term-indexed calculi described in Part \ref{part:Calculi} are closely
related to \Fw\ by an index-erasure property. The hope is that
our implementation can benefit from these technologies.

Type inference algorithms for functional programming languages are often
based on certain restrictions of the Curry-style polymorphic lambda calculi.
These restrictions are designed to avoid higher-order unification during
type inference.
We develop a conservative extension of Hindley--Milner type inference for
Nax (Chapter \ref{ch:naxTyInfer}). This is possible because Nax is based on our
term-indexed calculi (Part \ref{part:Calculi}). Dependently typed languages,
on the other hand, are often based on bidirectional type checking, which
requires annotations on top level definitions, rather than
Hindley--Milner-style type inference.

In dependent type theories, datatypes are usually supported as primitive
constructs with axioms, rather than as functional encodings
(\eg, Church encodings). One can give functional encodings for datatypes
in a dependent type theory, but one soon realizes that the induction principles
(or, dependent eliminators) for those datatypes cannot be derived within
the pure dependent calculi \cite{Geuvers01}.
So, dependently typed reasoning systems support datatypes as primitives.
For instance, Coq is based on Calculus of Inductive Constructions, which
extends Calculus of Constructions \cite{CoqHue86} with dependent datatypes
and their induction principles.

In contrast, in polymorphic type theories, all imaginable datatypes
within the calculi have functional encodings (\eg, Church encodings).
For instance, \Fw\ need not introduce datatypes as primitive constructs,
since \Fw\ can embed all imaginable datatypes, including non-regular
recursive datatypes with type indices. 

Another reason to use term-indexed calculi, rather than dependent type theories,
is to extend the application of Mendler-style recursion schemes,
which are well-understood in the context of \Fw.
Researchers have thought about (though not published)\footnote{
     Tarmo Uustalu described this on a whiteboard
     when we met with him at the University of Cambridge in 2011.
     We discuss this in the related work chapter (\S\ref{sec:relwork:dep}).}
Mendler-style primitive recursion over dependently-typed functions
over positive datatypes (\ie, datatypes that have a map), but not for
negative (or, mixed-variant) datatypes. In our term-indexed calculi,
we can embed Mendler-style recursion schemes (just as we embedded them in \Fw)
that are also well-defined for negative datatypes.

\section{Contributions}\label{sec:intro:contrib}
This dissertation makes contributions in several areas.
\begin{itemize}
\item[1.]
    It organizes and expands the realm of \emph{Mendler-style recursion schemes}
    (Part~\ref{part:Mendler})

\item[2.] It establishes a meta-theories for \emph{term-indexed types}
        (Part~\ref{part:Calculi}),

\item[3.] It designs a practical language (with an implementation)
        \emph{in the sweet spot} between programming and logical reasoning
        (Part~\ref{part:Nax}), and

\item[4.] It identifies several interesting open problems related to above.
\end{itemize}

\subsection{Contributions related to the Mendler style}
We organize a hierarchy of Mendler-style recursion schemes in two dimensions.
The first dimension is the abstract operations they support. For instance,
the Mendler-style iteration (\MIt) supports a single abstract operation
the recursive call. All the other Mendler-style recursion schemes
support the recursive call and an additional set of abstract operations. 
The second dimension is over the kind of the datatypes they operate over.
For example, \texttt{Nat} has kind $*$, while \texttt{Vec}
has kind $* -> \mathtt{Nat} -> *$. Each recursion scheme is actually a
family of recursion combinators sharing the same term definition
(\ie, uniformly defined) but with different type signatures at each kind.

We expand the realm of Mendler-style recursion schemes in several ways.
First, we report on a new recursion scheme $\MsfIt$, which is useful
for negative datatypes.  Second, we study the termination behaviors
of Mendler-style recursion schemes. Some recursion schemes (\eg, \MIt, \MsfIt)
always terminate for any recursive type, while others (\eg, \McvPr) only
terminate for certain classes of recursive types. Third, we extend
all Mendler-style recursion schemes to be expressive over term-indexed types.
The Mendler style has been studied in the context of \Fw\ (and several
extensions) which can express \emph{type}-indexed types. To extend Mendler-style
recursion schemes to be expressive over \emph{term}-indexed types, we report on
several theories for calculi (\Fi\ and \Fixi) that support term indices.
This is another important area of our contribution.

We provide examples that illustrate when each recursion scheme is useful
in Chapter \ref{ch:mendler}. The most interesting example among them is
the type-preserving evaluator for a simply-typed Higher-Order Abstract Syntax
(HOAS) (\S\ref{sec:evalHOAS}), which involves negative datatypes with indices.
This example is our novel discovery, which reports that
a type-preserving evaluator for a simply-typed HOAS can be expressed within \Fw.

In addition, we develop a better understanding of some existing
Mendler-style recursion schemes. For instance, the existence of
Mendler-style course-of-values recursion (\McvPr) is reported
in the literature, but the calculus that can embed \McvPr\ was unknown.
We embed Mendler-style course-of-values recursion into \Fixi\ 
(or into \Fixw\ \cite{AbeMat04}, when we do not consider term-indices).

\subsection{Contributions to the theory of Term-Indexed Types}
Mendler-style recursion schemes have been studies in the context of
polymorphic lambda calculi. For instance, \citet{AbeMatUus03} embedded 
Mendler-style iteration (\MIt) into \Fw\ and \citet{AbeMat04} embedded
Mendler-style primitive recursion (\MPr) into \Fixw. These calculi
support type-indexed types.

To extend the realm of Mendler-style recursion schemes to include
term-indexed types, we extended \Fw\ and \Fixw\ to support term indices.
In Part~\ref{part:Calculi}, we present our new calculi
\Fi\ (Chapter~\ref{ch:fi}), which extends \Fw\ with term indices, and
\Fixi\ (Chapter~\ref{ch:fixi}), which extends \Fixw\ with term indices.
These calculi have an erasure property that states that well-typed terms
in each calculus are also well typed terms (when erased) in the 
underlying calculus. For instance, any well typed term in \Fi\ is also
a well-typed term in \Fw, and there are no additional well-typed terms
in \Fi\ that are not well-typed in \Fw.

Our new calculi, \Fi\ and \Fixi, are strongly normalizing and
logically consistent. We show strong normalization and logical consistency
using the erasure properties. That is, strong normalization and
logical consistency of \Fi\ and \Fixi\ are inherited from \Fw\ and \Fixw.
Since \Fi\ and \Fixi\ are strong normalizing and logically  consistent,
the Mendler-style recursion schemes that can be embedded into these calculi
are adequate for logical reasoning as well as programming.

\subsection{Contributions in the design of the Nax language}
We design and implement a prototypical language Nax that explores
the sweet spot between programming oriented systems and logic oriented systems.
The language features supported by Nax provide the advantages
of both programming oriented systems and logic oriented systems.
Nax supports both term- and type-indexed datatypes,
rich families of Mendler-style recursion combinators,
and a conservative extension of Hindley--Milner type inference.
We designed Nax so that its foundational theory and
implementation framework could be kept simple.

Term- and type-indexed datatypes can express fine grained program properties
via the Curry--Howard correspondence, as in logic oriented systems. Although
not as flexible as full-fledged dependent types, indexed datatypes can
still express program invariants, such as stack-safe compilation
(\S\ref{sec:example}), and size invariants on data structures.
Index types can simulate much of what
dependent types can do using singleton types. Since Nax has only erasable
indices, the foundational theory can be kept simple, and it supports
features that have the advantages of programming oriented systems 
(\eg, type inference, arbitrary recursive datatypes).

Adopting Mendler style provides merits of both programming oriented systems
and logic oriented systems. Since Mendler style is elimination based, one can
define all recursive datatypes usually supported in functional programming
languages. In addition, the programs written using Mendler-style recursion
combinators look more similar to the programs written using general recursion
than programs written in Squiggol style.
Since Nax supports only the well-behaved (\ie, strongly normalizing)
Mendler-style recursion combinators, it is safe to construct proofs using them.
In addition, Mendler-style recursion combinators are naturally well-defined
over indexed datatypes, which are essential to express fine-grained program
properties. Mendler style provides type based termination, that is, termination
is a by-product of type checking. Thus, it makes the implementation framework
simple since we do not need extra termination checking theories or algorithm.

Hindley--Milner-style type inference is familiar 
to functional programmers.
Nax can infer types for all programs that involve only regular datatypes,
which are already inferable in Hindley--Milner, without any type annotation.
Nax requires programs involving indexed datatypes to annotate their eliminators
by index transformers, which specify the relation between the input type index
and the result type. Eliminators of non-recursive datatypes are case expressions
and eliminators of recursive datatypes are Mendler-style recursion combinators.

\subsection{Contributions identifying open problems}
We identify several open problems alongside the contributions mentioned
in previews subsections. We will discuss the details of these open problems
in the future work chapter (Chapter \ref{ch:futwork}).
Here, we briefly introduce two of them.

\paragraph{Handling different interpretations of $\mu$ in one language system:}
Nax provides multiple recursion schemes (or, induction principles) used
to describe different kinds of recursive computations over recursive datatypes.
These recursion schemes are all motivated by concrete examples, which explains
the need for multiple schemes. It is more convenient to express various kinds of
recursive computations in Nax, by choosing a recursion scheme that fits
the structure of the computation, than in those systems that provide
only one induction scheme. However, there is theoretical difficulty
handling multiple interpretations of the recursive type operator $\mu$
in one language system.

Recall that we can embed datatypes as functional encodings in
our indexed type theory. Recursive datatypes and their recursion schemes in Nax
are embedded using Mendler-style encodings.
In Mendler style, one encodes the recursive type operator $\mu$
and its eliminator (the recursion scheme) as a pair.
So, there are several different encodings of $\mu$,
one for each recursion scheme. Some recursion schemes subsume others
(\ie, the more expressive one can simulate the other).

It would have been easy to describe the theory for Nax if we had
one most powerful recursion scheme that subsumes all the others,
which leads to a single interpretation of $\mu$. Unfortunately, we know of
no Mendler-style recursion scheme that subsumes all the other recursion schemes
in Nax. For instance, iteration (\MIt) can be subsumed by either 
iteration with a syntactic inverse (\MsfIt) or primitive recursion (\MPr).
But, there is no known recursion scheme that can subsume both \MsfIt\ and \MPr.

However, we strongly believe that it is okay to apply \MsfIt\ to
the result of \MPr\ (when \MPr\ produces a recursive value) and vice versa.
Intuitively, the different interpretations of $\mu$ only matter during
the internal computation of the recursion scheme. That is, one may consider
that (recursive) values resulting from different recursion schemes
share a common abstract representation of $\mu$.
The theoretical justification for this is still ongoing work.

\paragraph{Deriving positivity (or monotonicity) from polarized kinds:}
One can extend the kind syntax of arrow kinds in \Fw\ with polarities
($p\kappa_1 -> \kappa_2$ where the polarity $p$ is either $+$, $-$, or $0$)
to track whether a type constructor argument is used in
covariant (positive), contravariant (negative), or
mixed-variant (both positive and negative) positions.
It is still an open problem whether it is possible to derive monotonicity
(\ie, the  existence of a map) for a type constructor from its polarized kind,
without examining the type constructor definition.

We identified a useful application for a solution to this open problem.
We discovered an embedding of Mendler-style course-of-values recursion in
a polarized system for positive (or monotone) type constructors.
That is, once you can show the existence of a map for a datatype,
course-of-values recursion always terminates.
However, in a practical language system, it is not desirable to burden users
with the manual derivation for every datatype on which they might want to
perform course-of-values recursion. If the type system can automatically
categorize datatypes that have maps from their polarized kinds,
this burden can be alleviated.

\newpage{}
\section{Methodology and Overview}\label{sec:intro:overview}
\begin{figure}
\centering
$\xymatrix{
 \textrm{STLC}
     \ar@{-->}[dd]_{\begin{smallmatrix}
                      \textrm{type}\\
                      \textrm{polymorphism}
                    \end{smallmatrix}} \\
      &&& \textrm{Hindley--Milner} \ar[dd] & \\
 \F  \ar[dd]_{\begin{smallmatrix}
                \textrm{type constructor} \\
                \textrm{polymorphism}
              \end{smallmatrix}}
     \ar@{.>}[rrru]|{\textrm{rank-1 restriction}}
                &&&& \\
                 &&& \textrm{Nax's type inference\hspace{-3em}} & \\
 \Fw   \ar@/^/@{=>}[rr]
       \ar@{~>}[dd]_{\begin{smallmatrix}
                       \textrm{polaraized kinds,}\\
                       \textrm{equi-recursive } \mu
                     \end{smallmatrix}}
                && \Fi \ar@{~>}[dd] \ar@/^/@{:>}[ll] \ar@{.>}[ru] &&
             \\ &&&& ~ \MIt \;~\phantom{\subset}~~~\; \MsfIt \hspace{-4em}
                            \ar@/^1pc/@{{ }{>}{>}}[llu]
                          & \ar@<-.5ex>@/_1pc/@{{ }{>}{>}}[lllu] \\
 \Fixw \ar@/^/@{=>}[rr]^{\textrm{term index}}
                && \Fixi \ar@/^/@{:>}[ll]^{\textrm{index erasure}}
                  && ~ \MPr \quad
                             \ar@{}[u]^{\bigcap}
			     \ar@/_/@{{ }{>}{>}}[ll]_{\textrm{embeded into}} \\
                  &&&& \McvPr
                             \ar@{}[u]^{\bigcap}
                      \ar@<.5em>@{}[u]_{\begin{smallmatrix}
                                         _\textrm{(termination behavior~}\\
                                         ^\textrm{~is not subsumed here)}
                                        \end{smallmatrix}}
			     \ar@/^/@{{ }{>}{>}}[llu] \\
}$

\caption{Summary on how key concepts are related}
\label{fig:overview}
\end{figure}

This dissertation consist of five parts:
\begin{itemize}
\item Part \ref{part:Prelude} (Prelude),
\item Part \ref{part:Mendler} (The Mendler style),
\item Part \ref{part:Calculi} (Term-indexed lambda calculi),
\item Part \ref{part:Nax} (The Nax language), and
\item Part \ref{part:Postlude} (Postlude).
\end{itemize}
The three parts in the middle describe the three steps of our approach. 
First, we explore new ideas about
Mendler-style recursion schemes driven from concrete examples
using Haskell (with some GHC extensions). Second, we develop
theories (\ie, lambda calculi) for term-indexed datatypes to prove that
the Mendler-style recursion schemes are well-defined over indexed datatypes
and have the expected termination behavior. Third, we design a language system
with practical features that implements our ideas and is based on the
theory we have developed. Figure~\ref{fig:overview} summarizes the organization of
key concepts throughout the dissertation.

\paragraph{Part \ref{part:Prelude} (Prelude)}\hspace{-1em} 
comprises Chapter \ref{ch:intro} (which you are currently reading)
and Chapter \ref{ch:poly} which 
reviews the theory of several well-known typed lambda calculi:
the simply-typed lambda calculus (STLC) (\S\ref{sec:stlc}),
System \F\ (\S\ref{sec:f}),
System \Fw\ (\S\ref{sec:fw}), and
the Hindley--Milner type system (\S\ref{sec:hm}).

In Sections \ref{sec:stlc}-\ref{sec:fw}, we review strong normalization proofs
(using saturated sets) for each of the three calculi:
STLC (no polymorphism), System \F\ (polymorphism over types), and
System \Fw\ (polymorphism over type constructors).

The later proofs each extend the normalization proof of the previous calculus.
We will use the strong normalization of System \Fw\ to show that
our term-indexed lambda calculi in Part \ref{part:Calculi} are
strongly normalizing. Readers familiar with strong normalization proofs
of these calculi may skip or quickly skim over these sections.
It is worth noticing two stylistic choices in our formalization of
System \F\ and \Fw: (1) terms are in Curry style and
(2) typing contexts are divided into two parts
    (one for type variables and the other for term variables).
This choice prepares readers for our formalization of the term-indexed calculi
in Part \ref{part:Calculi}, which have Curry-style terms and
typing contexts divided into two parts.

In \S\ref{sec:hm}, we review the type inference algorithm for
the Hindley--Milner type system (\S\ref{sec:hm}).
The Hindley--Milner type system (HM) is a restriction of System~\F,
which makes it possible to infer types without any type annotation on terms.
Later in Part~\ref{part:Nax} Chapter \ref{ch:naxTyInfer},
we formulate a conservative extension of HM, which restricts
the term-indexed calculus System \Fi\ (Chapter \ref{ch:fi}) in a similar manner.

\paragraph{Part \ref{part:Mendler} (the Mendler style)}\hspace{-1em} introduces
the concept of Mendler-style recursion schemes (Chapter \ref{ch:mendler})
using examples written in Haskell (with some GHC extensions). The readers
of Chapter \ref{ch:mendler} need no background knowledge on typed lambda calculi
but only some familiarity with functional programming. We explain the concepts of
a number of Mendler-style recursion schemes, their termination properties, and
how one recursion scheme is related to another. 
We also provide semi-formal proofs of termination
for some of the recursion schemes (\MIt\ and \MsfIt) by embedding them
into the \Fw\ fragment of Haskell. More formal and general
proofs, by embedding the schemes into our term-indexed lambda calculi, come later in
Part~\ref{part:Calculi}.

The Mendler-style recursion schemes discussed in Chapter \ref{ch:mendler}
include iteration (\MIt), iteration with syntactic inverse (\MsfIt),
primitive recursion (\MPr), course-of-values iteration (\McvIt),
and course-of-values recursion (\McvPr). Of these, \MsfIt\ was discovered
while writing this dissertation.
There are even more Mendler-style recursion schemes, which are not
discussed in Chapter \ref{ch:mendler} -- we give pointers to them in our
related work chapter (Chapter \ref{ch:relwork} of Part \ref{part:Postlude}).

\paragraph{Part \ref{part:Calculi} (term-indexed lambda calculi)}\hspace{-1em}
develops theories for term-indexed types.
We formalize two term-indexed lambda calculi,
System \Fi\ (Chapter \ref{ch:fi}) and System \Fixi\ (Chapter \ref{ch:fixi}),
which are extensions of polymorphic calculi.
System \Fi\ extends System \Fw\ with term indices and
System \Fixi\ extends System \Fixw\ \cite{AbeMat04} with term indices.

We prove both strong normalization and logical consistency of
these term-indexed calculi using an index erasure property.
The index erasure property of a term-indexed calculus
projects a typing in the term-index calculi into
the polymorphic calculus that the term-indexed calculus extends.
That is, all well-typed terms in \Fi\ and \Fixi\ are
also well-typed typed terms in \Fw\ and \Fixw.


By embedding Mendler-style  recursion schemes into our term-indexed lambda calculi,
we prove that those schemes are well-defined and
terminate over term-indexed datatypes.  For instance,
\MIt\ and \MsfIt\ can be embedded into System \Fi,
and, \MPr\ and \McvPr\ can be embedded into System \Fixi.

\paragraph{Part \ref{part:Nax} (the Nax language)}\hspace{-1em} consists of
three chapters.
First, we introduce the features of Nax (Chapter \ref{ch:naxFeatures})
in a tutorial format using small Nax code snippets as examples.
Next, we discuss the design principles of the type system (Chapter \ref{ch:nax})
by comparing it to two other systems: Haskell's datatype promotion and Agda.
In Chapter \ref{ch:nax} we develop
larger and more practical examples,
a type-preserving interpreter and a stack safe compiler.
Lastly, we discuss type inference in Nax (Chapter \ref{ch:naxTyInfer}),
which is a conservative extension of the Hindley--Milner type system (HM).
That is, any program whose type is inferable in HM, can also have its type
inferred in Nax without any annotation. Programs involving
term- or type-indexed datatypes, which are not supported in HM, need
some annotation to infer their types in Nax. These annotations are only
required on three syntactic entities (datatype declarations, case expressions,
and Mendler-style recursion combinators).

\paragraph{Part \ref{part:Postlude} (Postlude)}\hspace{-1em} closes
the dissertation by summarizing
  related work (Chapter~\ref{ch:relwork}),
  future work (Chapter~\ref{ch:futwork}), and
  conclusions (Chapter~\ref{ch:concl}).


\section{\Fi}
\Fi\ is an extension of \Fw\ with indexed types, but less powerful than
the Calculus of Constructions (CC). That is, all typeable terms in \Fi\ are
typeable terms in \Fi\ and all typable terms in \Fi\ are typable terms in CC.
Since CC is normalizing, normalization of \Fi\ is automatic.

\subsection{\Fi\ in Curry style}

\begin{align*}
&\text{Kinds}
 	& \kappa		&~ ::= ~ *
				\mid \ka -> \kappa
\\
&\text{Kind Arguments}
 	& \ka		&~ ::= ~ \kappa \mid A
\\
%% &\text{Type Schemes}
%% 	& \sigma	&~ ::= ~ A
%% 			\mid (x:A) -> \sigma
%% \\
%% &\text{Kind Schemes}
%% 	& \delta	&~ ::= ~ \kappa
%% 			\mid (x:A) -> \delta
%%			\mid (X:\kappa) -> \delta
%% \\
&\text{Type Constructors}
	& A,B,F,G		&~ ::= ~ X
				\mid \lambda X^\ka.F
				\mid F\,I
				\mid A-> B
				\mid \forall X^\ka . B
\\
&\text{Type Arguments}
	& I			&~ ::= ~ G \mid t
\\
&\text{Terms}
	& r,s,t			&~ ::= ~ x \mid \lambda x.t \mid r\;s
\\
&\text{Contexts}
	& \Delta		&~ ::= ~ \cdot
				\mid \Delta, x : A
				\mid \Delta, X^\ka
\end{align*}
%% \caption{Syntax of \Fi}
%% \label{tbl:syntaxFi}

\paragraph{Wellformed typing context}
\[ \inference{}{|- \cdot}
 ~~~~
   \inference{|- \Delta & \Delta |- \kappa:\square}{|- \Delta,X^\kappa}
 ~~~~
   \inference{|- \Delta & \Delta |- A:*}{|- \Delta,X^A}
 ~~~~
   \inference{|- \Delta & \Delta |- A:*}{|- \Delta,x:A}
\]

\paragraph{Sorting.}
\[ \inference{}{\Delta |- *:\square}
 ~~~~
   \inference{\Delta |- \kappa:\square & \Delta |- \kappa':\square}
             {\Delta |- \kappa -> \kappa' : \square}
 ~~~~
   \inference{\Delta |- A:* & \Delta |- \kappa:\square}
             {\Delta |- A -> \kappa : \square}
\]
\paragraph{Kinding.}
\[ \inference{X^\kappa\in\Delta & |- \Delta}
             {\Delta |- X : \kappa}
 ~~~~
   \inference{\Delta,X^\ka |- F : \kappa}
             {\Delta |- \lambda X^\ka.F : \ka->\kappa}
 ~~~~
   \inference{\Delta |- F : \ka->\kappa & \Delta |- I : \ka}
             {\Delta |- F\,I : \kappa}
\]
\[ \inference{\Delta |- A : * & \Delta |- B : *}{\Delta |- A -> B : * }
 ~~~~
   \inference{\Delta, X^\ka |- B : *}{\Delta |- \forall X^\ka . B : *}
 ~~~~
   \inference{\Delta |- A : \kappa & \Delta |- \kappa = \kappa' : \square}
             {\Delta |- A : \kappa'}
\]

\paragraph{Typing.}
\[ \inference{x:A \in \Delta & |- \Delta}{\Delta |- x:A}
 ~~~~
   \inference{\Delta, x:A |- t : B}{\Delta |- \lambda x.t : A-> B}
 ~~~~
   \inference{\Delta |- r : A-> B & |- \Delta |- s : A}{\Delta |- r\;s : B}
\]
\[ \inference{\Delta, X^\ka |- t : B}{\Delta |- t : \forall X^\ka.B}
 ~~~~
   \inference{\Delta |- t : \forall X^\ka.B & \Delta |- I:\ka}
             {\Delta |- t : [I/X]B}
 ~~~~
   \inference{\Delta |- t :A & \Delta |- A = B : *}{\Delta |- t : B}
\]

\paragraph{Kind equality}
\[ \inference{}{\Delta |- * = *:\square}
\]
\[
   \inference{ \Delta |- \kappa_1 = \kappa_1':\square
             & \Delta |- \kappa_2 = \kappa_2':\square}
             {\Delta |- \kappa_1-> \kappa_2 = \kappa_1'-> \kappa_2':\square}
 ~~~~
   \inference{\Delta |- A=A':* & \Delta |- \kappa=\kappa':\square}
             {\Delta |- A -> \kappa = A' -> \kappa' : \square}
\]
\[ \inference{\Delta |- \kappa=\kappa':\square}
             {\Delta |- \kappa'=\kappa:\square}
 ~~~~
   \inference{ \Delta |- \kappa =\kappa' :\square
             & \Delta |- \kappa'=\kappa'':\square}
             {\Delta |- \kappa=\kappa'':\square}
\]

\paragraph{Type constructor equality.}
%% Computation Axioms
\[ \inference{\Delta,X^\ka |- F:\kappa & \Delta |- I:\ka}
             {\Delta |- (\lambda X^\ka.F)\,I = [I/X]F:\kappa}
\]
%% Congruences
\[ \inference{\Delta |- X:\ka }{\Delta |- X=X:\ka}
 ~~~~
   \inference{\Delta |- F=F':\ka->\kappa & \Delta |- I=I':\ka}
             {\Delta |- F\,I = F'\,I' : \kappa}
\]
 ~~~~
\[ \inference{\Delta,X^\ka |- F=F' : \kappa}
             {\Delta |- \lambda X^\ka.F=\lambda X^\ka.F' : \ka->\kappa}
 ~~~~
   \inference{\Delta,X^\ka |- A=A':*}
             {\Delta |- \forall X^\ka.A=\forall X^\ka.A':*}
\]
\[ \inference{\Delta |- A=A':* & \Delta |- B=B':*}{\Delta |- A-> B=A'-> B':*}
\]
%% Symmetry and Transitivity
\[ \inference{\Delta |- F = F' : \kappa}{\Delta |- F' = F : \kappa}
 ~~~~
   \inference{\Delta |- F = F' : \kappa & \Delta |- F' = F'' : \kappa}
             {\Delta |- F = F'' : \kappa}
\]

\paragraph{Term equality}
\[ \inference{\Delta |- x:A}{\Delta |- x=x:A}
 ~~~~
   \inference{\Delta,x:A |- t:B & \Delta |- s:A}
             {\Delta |- (\lambda x.t)s=[s/x]t : B}
 ~~~~
   \inference{\Delta |- r=r':A-> B & \Delta |- s=s':A}{\Delta |- r\;s=r'\;s':B}
\]
\[ \inference{\Delta |- t=t':A}{\Delta |- t'=t:A}
 ~~~~
   \inference{\Delta |- t=t':A & \Delta |- t'=t'':A}{\Delta |- t=t'':A}
\]








\subsection{\Fi\ in regards to PTS}
We describe \Fw\ and CC as an instances of the Pure Type System (PTS),
and then describe \Fi\ as an extension of \Fw\ and a restriction of CC.

%% \begin{figure}
\[\text{Syntax}~~~~~ a,b,A,B ::= s \mid x \mid b~a \mid \lambda x:A.B
                                   \mid (x:A) -> B \]
\[\text{Typing}~~~~~\]
\[
 \inference[Ax]{|-\Gamma & (s_1,s_2)\in\calA}{\Gamma |- s_1 : s_2}
 ~~~~
 \inference[Var]{|-\Gamma & (x:A)\in\Gamma}{\Gamma |- x:A} 
 ~~~~
 \inference[Conv]{\Gamma |- a:A & \Gamma |- B:s & A=B}{\Gamma |- a:B}
\]
\[
 \inference[Pi]{\Gamma |- A:s_1 & \Gamma,x:A |- B:s_2 & (s_1,s_2,s_3)\in\calR}
               {\Gamma |- (x:A)-> B : s_3}
\]
\[
 \inference[Lam]{\Gamma |- (x:A)-> B : s & \Gamma,x:A |- b:B}
                {\Gamma |- \lambda x:A.b : (x:A)-> B}
 ~~~~
 \inference[App]{\Gamma |- b:(x:A)-> B & \Gamma |- a:A}{\Gamma |- b\;a:[a/x]B}
\]

\[\text{Context}~~~~~
 \inference{}{|-\cdot}
 ~~~~
 \inference{x\notin\dom(\Gamma) & |-\Gamma}{|-\Gamma,x:A}
\]
%% \caption{Pure Type System}
%% \label{fig:PTSrule}
%% \end{figure}

PTS is instantiated by the triple $(\calS,\calA,\calR)$ where
$\calS$ is the set of sorts ($s,s_1,s_2,s_3\in\calS$ in the PTS typing rules),
$\calA\subset\calS\times\calS$ is the set of axioms (used in Ax rule), and
$\calR\subset\calS\times\calS\times\calS$ is the set of rules (used in Pi rule).

\Fw\ is an instance of PTS where
\[\calS=\{*,\square\}
  ~~~~~
  \calA=\{(*,\square)\}
  ~~~~~
  \calR=\{(*,*,*),(\square,*,*),(\square,\square,\square)\}
\]
Each element in $\calR$ stands for certain ability of dependency:
$(*,*,*)$ enables functions at the value level
(values depending on values)
$(\square,*,*)$ enables parametric polymorphism
(values depending on types)
$(\square,\square,\square)$ enables functions at the type level
(types depending on types).

CC extends \Fw\ with one additional rule
$(*,\square,\square)$, which enables types depending on values
(\emph{dependent types}, or, types with \emph{value dependency}).
That is, CC is an instance of PTS where
\[\calS=\{*,\square\}
  ~~~~~
  \calA=\{(*,\square)\}
  ~~~~~
  \calR=\{(*,*,*),(\square,*,*),(\square,\square,\square)
         ,(*,\square,\square)\}\]

\Fi\ is also an extension of \Fw\, but a restriction of CC.  Instead of
allowing full value dependency (given by $(*,\square,\square)$ in CC),
we only allow value dependency partially via special syntax as follows:
\[
 \inference[Let]{\Gamma |- a:A & \Gamma, x:A |- b:B}
                {\Gamma |- \mathsf{let}~x:A=a~\mathsf{in}~b : [a/x]B}
\]
\[
 \inference[Pi$_{(*,\square,\square)}$]
     {\Gamma |- A:* & \Gamma |- B:\square}
     {\Gamma |- (x:A)-> B : \square}
\]
The Let rule can be derived in CC by combining Pi rule and App rule.  However,
we only allow value dependency $(*,\square,\square)$ in this limited form,
but disallow arbitrary lambda abstractions to form value dependent functions
by limiting the Pi rule on $(*,\square,\square)$.

Normalization of \Fi\ is automatic since \Fi\ is a strict subset of CC,
which is known to be strongly normalizing.

\subsection{examples}
For better intuition, let us see some examples what we can express and
what we cannot in \Fi.
\newcommand{\List}{\mathsf{List}}
\newcommand{\Vect}{\mathsf{Vec}}

\paragraph{Ordinary (parametric) polymorphism} is obviously possible,
since \Fi\ is an extension of \Fw. Note, $\Gamma |- (x:*) -> B : *$
where $\Gamma,x:* |- B:*$, since $(\square,\star,\star) \in\calR$.
If you'd prefer \Fw\-like syntax we could use alternative notation
$\forall(x:*).B$ for $(x:*) -> B$. With the support of ordinary polymorphism,
we can write functions over polymorphic lists as follows
where $\List : * -> *$ (details of the function body is omitted):
\[ (\Lambda a:*.\lambda x:\List~a. \cdots) : \forall(a:*).\List~a -> \List~a \]
The uppercase lambda ($\Lambda$) is an alternative notation in flavour of \Fw
for the ordinary lowercase lambda ($\lambda$)
More generally, we can be polymorphic over any \Fw\ kind such as $* -> *$.
For instance, $\forall(x:* -> *).B$.

\paragraph{Polymorphism over indexed values} is also possible.
Note, $\Gamma |- (x:A) -> B : *$ where $\Gamma |- A:*$ and $\Gamma,x:A |- B:*$.
With the support of polymorphism over indexed values, we can write functions
over type indexed lists as follows (details of the function body is omitted)
where $\Vect : \Nat -> *$:
\[
(\Lambda n:\Nat.\lambda x:\Vect~n. \cdots) : \forall(n:\Nat).\Vect~n -> \Vect~n 
\]
More generally, $\Gamma |- (x:\ka) -> B : *$ where $\Gamma, x:\ka |- B:*$
can be written as $\forall(x:\ka).B$ in flavour of \Fw.

We call the (parametric) polymorphism of \Fi, which includes
both the ordinary polymorphism of \Fw\ and the polymorphism involving indices,
\emph{generalized polymorphism}.

To be more precise, this is not exactly the polymorphism we would expect
in functional languages with indexed types (\eg, GADT extension in Haskell),
since we usually expect index variables like $n$ bound by $\Lambda$ to be
erasable (or, computationally irrelevant) in contrast to the variables bound
by ordinary $\lambda$. Here in PTS, which is in Church style, $n$ can also
appear as computationally relevant in the function body. Unlike ordinary
polymorphism where the polymorphic variable is a type variable that cannot
live in value world, nothing prevents the index variables like $n$ to be used
in the value level. This is what Abel and Matthes in CSL 04 paper meant that
the Church style becomes richer than the Curry style as we move from \Fw\ to
a dependently typed setting such as CC. I think it is better to formalize
in Curry style if we want to capture the erasure property of polymorphism
correctly in \Fi.

\paragraph{Large eliminations resulting in constant kinds} are possible.
Large eliminations are type computations depending on values.  For example,
computing an $n$-ary boolean function type given a natural number index $n$
can be defined as follows (details of the function body is omitted):
\[
(\lambda n:\Nat. \stackrel{n}{\overbrace{\Bool-> \cdots-> \Bool}}\;\, -> \Bool)
  : (n:\Nat) -> *
\]
Note, the resulting kind ($*$) is constant, being independent of $n$.
So, the example above is a large elimination resulting in a constant kind,
which does not involve any kind computation.

However, large eliminations resulting in varying kinds that depend on
input index values are not possible, unless you use the let syntax to make
the input index values available (then, the input indices do not vary
at runtime after all). For example, large elimination computing an $n$-ary
type constructor given a natural number index $n$ line the following
is \emph{not} possible to define in \Fi:
\[
(\lambda n:\Nat. \cdots \cdots)
  : (n:Nat) -> \,\;\stackrel{n}{\overbrace{* -> \cdots -> *}}\;\, -> *
\]
This restriction is to make the kind inference doable.


\section{Embedding datatypes and their eliminators} \label{sec:data}
We demonstrate some examples of embedding datatypes into System \Fi.
%% TODO cite some paper that does this with System Fw or System F
We first illustrate embeddings for both non-recursive datatypes and
recursive datatypes, where we use Church encodings \cite{Church33}
for data constructors (\S\ref{ssec:embedChurch}). Then, we illustrate
a more involved embedding for the recursive datatypes based on two-level types
(\S\ref{ssec:embedTwoLevel}).

\subsection{Embedding datatypes using Church-encoded terms}
\label{ssec:embedChurch}
\begin{figure}
\begin{align*}
&\!\!\!\!\!\!\mathtt{Bool} &=~& \forall X.X -> X -> X \\
&\!\!\!\!\!\!\mathtt{true}  &\!\!\!:~~& \texttt{Bool} ~~=~ \l x_1.\l x_2. x_1 \\
&\!\!\!\!\!\!\mathtt{false} &\!\!\!:~~& \texttt{Bool} ~~=~ \l x_1.\l x_2. x_2 \\
&\!\!\!\!\!\!\mathtt{elim_{Bool}} &\!\!\!:~~& \texttt{Bool} -> \forall X.X -> X -> X \\
&	&=~& \l x.\l x_1. \l x_2. x\;x_1\,x_2 \qquad
(\textbf{if}~x~\textbf{then}~x_1~\textbf{else}~x_2)
\end{align*}\vspace*{-19pt} \\ \vspace*{-4pt}
\rule{\linewidth}{.4pt}
\begin{align*}
&\!\!\!\!\!\!A_1\times A_2 &=~& \forall X. (A_1 -> A_2 -> X) -> X \\
&\!\!\!\!\!\!\mathtt{pair} &\!\!\!:~~&
	\forall A_1^{*}.\forall A_2^{*}.A_1\times A_2
	~~=~ \l x_1.\l x_2.\l x'.x'\,x_1\,x_2 \\
&\!\!\!\!\!\!\mathtt{elim_{(\times)}} &\!\!\!:~~&
	\forall A_1^{*}.\forall A_2^{*}.A_1\times A_2 ->
	\forall X. (A_1 -> A_2 -> X) -> X \\
	& &=~& \l x.\l x'.x\;x' \\
 &&&\!\!\!\!\!\!\!\!\text{(by passing appropriate values to $x'$, we get}\\
 &&&\!\!\!\!\textit{fst} = \l x.x(\l x_1.\l x_2.x_1),~
            \textit{snd} = \l x.x(\l x_1.\l x_2.x_2) ~)
\end{align*} \vspace*{-19pt} \\ \vspace*{-4pt}
\rule{\linewidth}{.4pt}
\begin{align*}
&\!\!\!\!\!\!A_1+A_2 &=~&\forall X^{*}. (A_1 -> X) -> (A_2 -> X) -> X \\
&\!\!\!\!\!\!\mathtt{inl} &\!\!\!:~~& \forall A_1^{*}.\forall A_2^{*}.A_1-> A_1+A_2
	~~=~ \l x. \l x_1. \l x_2 . x_1\,x \\
&\!\!\!\!\!\!\mathtt{inr} &\!\!\!:~~& \forall A_1^{*}.\forall A_2^{*}.A_2-> A_1+A_2
	~~=~ \l x. \l x_1. \l x_2 . x_2\,x \\
&\!\!\!\!\!\!\mathtt{elim_{(+)}} &\!\!\!:~~&
	\forall A_1^{*}.\forall A_2^{*}.(A_1+A_2) -> \\
	&&& \forall X^{*}. (A_1 -> X) -> (A_2 -> X) -> X \\
	& &=~& \l x.\l x_1. \l x_2. x\;x_1\,x_2 \\
	&&&			(\textbf{case}~x~\textbf{of}~
				\{\mathtt{inl}~x' -> x_1\;x';
				  \mathtt{inr}~x' -> x_2\;x'\})
\end{align*}~\vspace*{-10pt}
\caption{Embedding non-recursive datatypes}
\label{fig:churchnonrec}
\end{figure}
\begin{figure}
\begin{align*}
&\!\!\!\!\!\!\mathtt{List} &\!\!\!\!\!=~& \l A^{*}.\forall X^{*}.(A-> X-> X)-> X-> X
	\\
&\!\!\!\!\!\!\texttt{cons} &\!\!\!\!\!:~~& \forall A^{*}.A-> \mathtt{List}\,A-> \mathtt{List}\,A \\
& & & \qquad~\qquad~\quad\, =~\l x_a.\l x.\l x_c.\l x_n. x_c\,x_a\,(x\;x_c\,x_n) \\
&\!\!\!\!\!\!\mathtt{nil} &\!\!\!\!\!:~~& \forall A^{*}.\texttt{List}\,A
~~=~ \l x_c.\l x_n.\l x_n \\
&\!\!\!\!\!\!\mathtt{elim_{List}} &\!\!\!\!:~~& \forall A^{*}.\texttt{List}\,A ->
	\forall X^{*}.(A -> X -> X) -> X -> X \\
& &\!\!\!\!\!=~& \l x.\l x_c. \l x_n.x\;x_c\,x_n\qquad
	\text{(\textit{foldr} $x_z$ $x_c$ $x~$ in Haskell)}
\end{align*}\vspace*{-19pt} \\ \vspace*{-4pt}
\rule{\linewidth}{.4pt}
\begin{align*}
&\!\!\!\!\!\!\mathtt{Powl} &\!\!\!\!\!=~& \l A^{*}.\\
&&&\forall X^{*-> *}.(A-> X(A\times A)-> XA)-> XA -> XA \\
&\!\!\!\!\!\!\texttt{pcons} &\!\!\!\!\!:~~& \forall A^{*}.A-> \mathtt{Powl}(A\times A)-> \mathtt{Powl}\,A \\
&&& \qquad~\qquad~\quad\, ~=~ \l x_a.\l x.\l x_c.\l x_n. x_c\,x_a\,(x\;x_c\,x_n) \\
&\!\!\!\!\!\!\mathtt{pnil} &\!\!\!\!\!:~~& \forall A^{*}.\texttt{Powl}\,A
~~~=~ \l x_c.\l x_n.\l x_n \\
&\!\!\!\!\!\!\mathtt{elim_{Powl}} &\!\!\!\!:~~& \forall A^{*}.\texttt{Powl}\,A -> \\
&&& \forall X^{*-> *}.(A -> X(A\times A) -> XA) -> XA -> XA \\
& &\!\!\!\!\!=~& \l x.\l x_c. \l x_n.x\;x_c\,x_n
\end{align*}\vspace*{-19pt} \\ \vspace*{-4pt}
\rule{\linewidth}{.4pt}
\begin{align*}
&\!\!\!\!\!\!\mathtt{Vec} &\!\!\!\!\!\!\!\!=~& \l A^{*}.\l i^{\mathtt{Nat}}.\\
&&&	\forall X^{\mathtt{Nat}-> *}.
	(\forall i^\mathtt{Nat}.A-> X\{i\}-> X\{\mathtt{succ}\,i\}) ->  \\
&&& \qquad~\qquad X\{\texttt{zero}\} -> X\{i\} \\
 &\!\!\!\!\!\!\texttt{vcons} &\!\!\!\!\!\!\!\!:~& \forall A^{*}.\forall i^\mathtt{Nat}.A-> \mathtt{Vec}\,A\,\{i\}-> \mathtt{Vec}\,A\,\{\mathtt{succ}\,i\} \\
&&&\;\qquad\qquad\quad =~ \l x_a.\l x.\l x_c.\l x_n. x_c\,x_a\,(x\;x_c\,x_n) \\
&\!\!\!\!\!\!\mathtt{vnil} &\!\!\!\!\!\!\!\!:~& \forall A^{*}.\texttt{Vec}\,A\,\{\mathtt{zero}\} 
~~~=~ \l x_c.\l x_n.\l x_n \\
&\!\!\!\!\!\!\mathtt{elim_{Vec}} &\!\!\!\!\!\!\!\!:~& \forall A^{*}.\forall i^\mathtt{Nat}.\texttt{Vec}\,A\,\{i\} -> \\
&&& \forall X^{\mathtt{Nat}-> *}.(\forall i^\mathtt{Nat}.A -> X\{i\} -> X\{\mathtt{succ}\,i\}) -> \\
&&& \qquad~\qquad X\{\mathtt{zero}\} -> X\{i\} \\
& &\!\!\!\!\!=~& \l x.\l x_c. \l x_n.x\;x_c\,x_n
\end{align*} ~\vspace*{-14pt}
\caption{Embedding recursive datatypes}
\label{fig:churchrec}
\end{figure}
\citet{Church33} demonstrated an embedding of natural numbers into
the untyped $\lambda$-calculus, which he invented, in order to argue
that the $\lambda$-calculus  expressive enough for the foundation of
logic and arithmetic. Church encoded the data constructors of natural numbers,
successor and zero, as higher-order functions,
$\mathtt{succ}=\l x.\l x_s.\l x_z.x_s(x\,x_s x_z)$ and
$\mathtt{zero}=\l x_s.\l x_z.x_z$.
The heart of the Church encoding is that a value is encoded as its elimination.
The bound variables $x_s$ and $x_z$ stands for the operations needed for
eliminating the successor case and the zero case. The Church encodings of
successor and zero states that: to eliminate $\mathtt{succ}\,x$, apply $x_s$
to the elimination of the predecessor $(x\,x_s x_z)$; and,
to eliminate $\mathtt{zero}$, just return $x_z$.
Since values themselves are eliminators,
eliminator can be defined as applying the value itself to the needed operations
for each data constructor case. For instance, we can define an eliminator
for natural numbers as $\mathtt{elim_{Nat}}=\l x.\l x_s.\l x_z.x\,x_s x_z$,
which is just an $\eta$-expansion of the identity function $\l x.x$.
Church encoded natural numbers are typable in polymorphic $\lambda$-calculi,
such as System \Fw, as follows:\vspace*{-2pt}
\begin{align*}
&\texttt{Nat} &=~& \forall X^{*}.(X -> X) -> X -> X \qquad\qquad\qquad\\
&\texttt{succ} &\!\!\!:~~& \texttt{Nat} -> \texttt{Nat}
	~~ =~ \l x.\l x_s.\l x_z.x_s(x\,x_s x_z) \\
&\texttt{zero} &\!\!\!:~~& \texttt{Nat} \qquad\quad\,
	~~ =~ \l x_s.\l x_z.x_z \\
&\mathtt{elim_{Nat}} &\!\!\!:~~& \texttt{Nat} -> \forall X^{*}.(X -> X)-> X-> X \\
& &=~& \l x.\l x_s.\l x_z.x\,x_s x_z
\end{align*}~\vspace*{-13pt}

Similarly, other datatypes are also embeddable into
polymorphic $\lambda$-calculi in this fashion.
Embeddings of some well-known non-recursive datatypes are illustrated
in Figure \ref{fig:churchnonrec}, and embeddings of the list-like
recursive datatypes, which we discussed earlier as motivating examples
(\S\ref{sec:motiv}), are illustrated in Figure \ref{fig:churchrec}.
Note that the term encodings for the constructors and eliminators of
the list-like datatypes in Figure \ref{fig:churchrec} are exactly the same.
For instance, the term encodings for \texttt{nil}, \texttt{pnil}, and
\texttt{vnil} coincide as $\l x_s.\l x_z.x_z$.

\subsection{
   Embedding the recursive type operators and the Mendler-style iterators}
\label{ssec:embedTwoLevel}
We can divide a recursive datatype definition into two levels,
by factoring out the recursive type operator, which weaves in the recursion
to the datatype definition, and a non-recursive base structure, which describes
the shape (\ie, number of data constructors and their types) of the datatype.
We can program with two-level types in functional languages that support
higher-order polymorphism\footnote{a.k.a. higher-kinded polymorphism
	or type constructor polymorphism}, such as Haskell, as illustrated
in Figure \ref{fig:twoleveltypes}. The function $\mathtt{mit}_\kappa$ descirbes
the Mendler-style iteration\footnote{An iteration is a principled recursion
	scheme guaranteed to terminate for any well-founded input.
	Also known as fold or catamorphism} for the recursive types
defined by $\mu_\kappa$. Although it is possible to wirte programs using
two level datatypes, we would not have gurantee of logical consistency
in such general purpose functional lanugages.

\begin{figure}
\begin{lstlisting}[basicstyle={\ttfamily\small},language=Haskell,mathescape]
newtype $\mu_{*}$ (f :: * -> *)
  = In$_{*}$ (f ($\mu_{*}$ f))

data ListF (a::*) (r::*)
  = Cons a r       | Nil

type List a = $\mu_{*}$ (ListF a)
cons x xs = In$_{*}$ (Cons x xs)
nil       = In$_{*}$ Nil

mit$_{*}$ :: ($\forall$ r.(r->x) -> f r -> x) -> Mu0 f -> x
mit$_{*}$ phi (In$_{*}$ z) = phi (mit$_{*}$ phi) z

newtype $\mu_{(*-> *)}$ (f :: (*->*) -> (*->*)) (a::*)
  = In$_{(*-> *)}$ (f (Mu$_{(*-> *)}$ f)) a

data PowlF (r::*->*) (a::*)
  = PCons a (r(a,a)) | PNil

type Powl a = $\mu_{(*-> *)}$ PowlF a
pcons x xs = In$_{(*-> *)}$ (PCons x xs)
pnil       = In$_{(*-> *)}$ PNil

mit$_{(*-> *)}$ :: ($\forall$ r a.($\forall$a.r a->x a) -> f r a -> x a)
        -> $\mu_{(*-> *)}$ f a -> x a
mit$_{(*-> *)}$ phi (In$_{(*-> *)}$ z) = phi (mit$_{(*-> *)}$ phi) z

-- above is Haskell (with some GHC extensions)
-- below is Haskell-ish psudocode

newtype $\mu_{(\mathtt{Nat}-> *)}$ (f::(Nat->*)->(Nat->*)) {n::Nat}
  = In$_{(\mathtt{Nat}-> *)}$ (f ($\mu_{(\mathtt{Nat}-> *)}$ f)) {n}

data VecF (a::*) (r::Nat->*) {n::Nat} where
  VCons :: a -> r n -> VecF a r {S n}
  VNil  :: VecF a r {Z}

type Vec a {n::Nat} = $\mu_{(\mathtt{Nat}-> *)}$ (VecF a) {n}
vcons x xs = In$_{(\mathtt{Nat}-> *)}$ (VCons x xs)
vnil       = In$_{(\mathtt{Nat}-> *)}$ VNil

mit$_{(\mathtt{Nat}-> *)}$::($\forall$ r n.($\forall$n.r{n}->x{n})->f r {n}->x{n})
        -> $\mu_{(\mathtt{Nat}-> *)}$ f {n} -> x{n}
mit$_{(\mathtt{Nat}-> *)}$ phi (In$_{(\mathtt{Nat}-> *)}$ z) = phi (mit$_{(\mathtt{Nat}-> *)}$ phi) z
\end{lstlisting}
\caption{2-level types and their Mendler-style iterators in Haskell}
\label{fig:twoleveltypes}
\end{figure}

Interestingly, there exists an embedding of the recursive type operator
$\mu_\kappa$, its data constructor $\mathtt{In}_\kappa$,
and the Mendler-style iterator $\mathtt{mit}\kappa$ at each kind $\kappa$
into higher-order polymorhpic lambda calculi, as in Figure \ref{fig:mu}.

However, there does not exist an embedding of destructing
(or, pattern matching away) the $\mathtt{In}_\kappa$ constructors.
It is a known fact that having arbitrary recursive datatypes
and the ability arbitrarily destruct (or, unroll) the values of
recursive types is powerful enough to define non-termiating computation
in a type safe way.

TODO For the non-recursive base structures, or non-recursive datatypes,
we can use the same impredicative encodings in the previous subsection.
That is, booleans, sums, and products are encoded as in Figure TODO.

TODO Encoding of two-level types are more involved than the encoding, but
iterator definitions become more general and uniform.

\begin{figure*}
\begin{align*}
&\mu_\kappa &\!\!\!\!\!~:~~& (\kappa -> \kappa) -> \kappa
  \qquad\qquad\qquad\qquad\quad
  = \l F^{\kappa -> \kappa}.\boldsymbol{\l}\mathbb{I}^\kappa.
  \forall X^\kappa.
  (\forall {X_r}^{\!\!\kappa}.
  	(X_r \stackrel{\kappa\;}{\pmb{\pmb{->}}} X) ->
	(F X_r \stackrel{\kappa\;}{\pmb{\pmb{->}}} X)) -> X\mathbb{I} \\
&\mu_{*} &\!\!\!\!\!~:~~& (* -> *) -> * 
 \qquad\qquad\qquad\qquad\quad~
 = \l F^{* -> *}.\phantom{\boldsymbol{\l}\mathbb{I}^\kappa.}
 \forall X^{*}.(\forall {X_r}^{\!\!*}.(X_r -> X) -> (F\,X_r -> X)) -> X \\
&\mu_{*-> *} &\!\!\!\!\!~:~~& ((* -> *) -> (* -> *)) -> (* -> *) \\
&            &\!\!\!\!\!=~&
  \l F^{(*-> *) -> (*-> *)}.\l X_1^{*}.
   \forall X^{* -> *}.(\forall {X_r}^{\!\!* -> *}.
   (\forall X_1^{*}.X_r X_1 -> X X_1) -> (\forall X_1^{*}.F\,X_r X_1 -> X X_1)) -> X X_1 \\
  &\mu_{\mathtt{Nat}-> *} &\!\!\!\!\!~:~~& ((\mathtt{Nat} -> *) -> (\mathtt{Nat} -> *)) -> (\mathtt{Nat} -> *) \\
&            &\!\!\!\!\!=~&
  \l F^{(\mathtt{Nat}-> *) -> (\mathtt{Nat}-> *)}.\l i_1^\mathtt{Nat}.
  \forall X^{\mathtt{Nat} -> *}.(\forall {X_r}^{\!\!\mathtt{Nat} -> *}.
  (\forall i_1^\mathtt{Nat}.X_r i_1 -> X i_1) -> (\forall i_1^\mathtt{Nat}.F\,X_r i_1 -> X i_1)) -> X i_1 \qquad\qquad
\end{align*}
\begin{align*}
\mathtt{In}_{\kappa} \,~\,&~~:~ \forall F^{\kappa-> \kappa}.
	F(\mu_\kappa F) \stackrel{\kappa\;}{\pmb{\pmb{->}}} \mu_\kappa F
&&=~ \l x_r. \l x_\varphi.x_\varphi\,(\mathtt{mit}_\kappa~x_\varphi)\,x_r
	\qquad~\qquad~\qquad~\qquad~\quad \\
\mathtt{mit}_\kappa &~~:~ \forall F^{\kappa-> \kappa}.\forall X^\kappa.
	(\forall {X_r}^{\!\!\kappa}.
	 (X_r \stackrel{\kappa\;}{\pmb{\pmb{->}}} X) ->
	 (F X_r \stackrel{\kappa\;}{\pmb{\pmb{->}}} X) ) ->
	(\mu_\kappa F \stackrel{\kappa\;}{\pmb{\pmb{->}}} X)
&&=~ \l x_\varphi.\l x_r.x_r\,x_\varphi
\end{align*}
\caption{Embedding of the recursive operators ($\mu_\kappa$),
	their data constructors ($\mathtt{In}_\kappa$),
	and the Mendler-style iterators ($\mathtt{mit}_\kappa$).}
\label{fig:mu}
\end{figure*}


\section{Metatheory} \label{sec:theory}

The expectation is that System \Fi\ has all the nice properties of System \Fw,
yet is more expressive (\ie, can state finer grained program properties)
because of the addition of term-indexed types.

We show some basic well-formedness properties for
the judgments of \Fi\ in \S\ref{ssec:wf}.
We prove erasure properties of \Fi, which captures the idea that indices are
erasable since they are irrelevant for reduction in \S\ref{ssec:erasure}.
We show strong normalization, logical consistence, and subject reduction for
\Fi\ by reasoning about well-known calculi related to \Fi\ in \S\ref{ssec:sn}.

\subsection{Well-formedness and Substitution Lemmas} \label{ssec:wf}

We want to show that kinding and typing derivations give
well-formed results under well-formed contexts. That is,
kinding derivations ($\Delta |- F : \kappa$) result in well-sorted kinds
($|- \kappa$) under well-formed type-level contexts ($|- \Delta$)
(Proposition \ref{prop:wfkind}), and
typing derivations ($\Delta;\Gamma |- t : A$) result in well-kinded types
($\Delta;\Gamma |- A:*$) under well-formed type and term-level contexts
(Proposition \ref{prop:wftype}).\\
\begin{minipage}{.5\linewidth} \vskip1.1ex
\begin{proposition}\hspace*{-2.5ex}
\label{prop:wfkind}
$ \inference*{ |- \Delta & \Delta |- F : \kappa}{
        \qquad |- \kappa:\square \quad} \!$\\
\end{proposition}
\end{minipage}
\begin{minipage}{.5\linewidth} \vskip1.1ex
\begin{proposition} \label{prop:wftype} \hspace{-2.5ex}
$ \inference*{ \Delta |- \Gamma & \Delta;\Gamma |- t : A}{
        \qquad \Delta |- A : * \qquad} \!\!\!$\\
\end{proposition}
\end{minipage}
We can prove these well-formedness properties
by induction over the judgment\footnote{The proof for
        Propositions \ref{prop:wfkind} and \ref{prop:wftype} are
        mutually recursive.
        So, we prove these two propositions at the same time, using a combined judgment $J$,
        which is either a kinding judgment or a typing judgment
        (\ie, $J ::= \Delta |- F : \kappa \mid \Delta;\Gamma |- t : A$).}
and using the substitution lemma below.

\begin{lemma}[substitution]\mbox{}\\[-3mm]
\label{lem:subst}
\begin{enumerate}
\item
%\begin{lemma}[type substitution]
\label{lem:tysubst}
\mbox{\rm (type substitution)}
$\inference{\Delta,X^\kappa |- F:\kappa' & \Delta |- G:\kappa}
        {\Delta |- F[G/X]:\kappa'} $
%\end{lemma}
\medskip

\item
%\begin{lemma}[index substitution]
\label{lem:ixsubst}
\mbox{\rm (index substitution)}
$ \inference{\Delta,i^A |- F:\kappa & \Delta;\cdot |- s:A}
        {\Delta |- F[s/i]:\kappa} $
%\end{lemma}
\medskip

\item
%\begin{lemma}[term substitution]
\label{lem:tmsubst}
\mbox{\rm (term substitution)}
$ \inference{\Delta;\Gamma,x:A |- t:B & \Delta;\Gamma |- s:A}
        {\Delta;\Gamma |- t[s/x]:B} $
%\end{lemma}
\end{enumerate}
\end{lemma}
These substitution lemmas are fairly standard,
comparable to substitution lemmas
in other well-known systems such as \Fw\ or ICC.


\subsection{Erasure Properties} \label{ssec:erasure}

We define a meta-operation of index erasure that projects $\Fi$-types
to $\Fw$-types.
\begin{definition}[index erasure]\label{def:ierase}
\[ \fbox{$\kappa^\circ$}
 ~~~~ ~~
 *^\circ =
 *
 ~~~~ ~~
 (\kappa_1 -> \kappa_2)^\circ =
 {\kappa_1}^\circ -> {\kappa_2}^\circ
 ~~~~ ~~
 (A -> \kappa)^\circ =
 \kappa^\circ
\]
\[ \fbox{$F^\circ$}
 ~~~~ ~~
 X^\circ =
 X
 ~~~~ ~~~~ ~~~~ ~~~~ ~~~~ ~~~~
 (A -> B)^\circ =
 A^\circ -> B^\circ
\]
\[ \qquad \qquad
 (\lambda X^\kappa.F)^\circ =
 \lambda X^{\kappa^\circ}.F^\circ
 ~~~~ ~~~
 (\lambda i^A.F)^\circ =
 F^\circ
\]
\[ \qquad \qquad
 (F\;G)^\circ =
 F^\circ\;G^\circ
 ~~~~ ~~~~ ~~~~ ~~
 (F\,\{s\})^\circ =
 F^\circ
\]
\[ \qquad \qquad
 (\forall X^\kappa . B)^\circ =
 \forall X^{\kappa^\circ} . B^\circ
 ~~~~ ~~~
 (\forall i^A . B)^\circ =
 B^\circ
\]
\[ \fbox{$\Delta^\circ$}
 ~~~~
 \cdot^\circ = \cdot
 ~~~~ ~~
 (\Delta,X^\kappa)^\circ = \Delta^\circ,X^{\kappa^\circ}
 ~~~~ ~~
 (\Delta,i^A)^\circ = \Delta^\circ
\]
\[ \fbox{$\Gamma^\circ$}
 ~~~~
 \cdot^\circ = \cdot
 ~~~~ ~~
 (\Gamma,x:A)^\circ = \Gamma^\circ,x:A^\circ
\]
\end{definition}~\\
In addition, we define another meta-operation, which selects out
all the index variable bindings from the type-level context.
We use this in Theorem \ref{thm:ierasetmctxivs}.
\begin{definition}[index variable selection]
        \[ \fbox{$\Delta^\bullet$} ~~~~ \cdot^\bullet = \cdot \qquad
        (\Delta,X^\kappa)^\bullet = \Delta^\bullet \qquad
        (\Delta,i^A)^\bullet = \Delta^\bullet,i:A
\]
\end{definition}



\begin{theorem}[index erasure on well-sorted kinds]
\label{thm:ierasesorting}
        $\inference{|- \kappa : \square}{|- \kappa^\circ : \square}$
\end{theorem}
\begin{proof}
        By induction on the sort ($\kappa$).
        \qed
\end{proof}
\begin{remark}
For any well-sorted kind $\kappa$ in \Fi,
$\kappa^\circ$ is a well-sorted kind in \Fw.
\end{remark}

\begin{theorem}[index erasure on well-formed type-level contexts]
\label{thm:ierasetyctx}
$ \inference{|- \Delta}{|- \Delta^\circ} $
\end{theorem}
\begin{proof}
        By induction on the type-level context ($\Delta$)
        and using Theorem \ref{thm:ierasesorting}.
        \qed
\end{proof}
\begin{remark}
For any well-formed type-level context $\Delta$ in \Fi,
$\Delta^\circ$ is a well-formed type-level context in \Fw.
\end{remark}

\begin{theorem}[index erasure on kind equality]\label{thm:ierasekindeq}
$ \inference{|- \kappa=\kappa':\square}
        {|- \kappa^\circ=\kappa'^\circ:\square}
$
\end{theorem}
\begin{proof}
        By induction on the kind equality derivation
        ($|- \kappa=\kappa':\square$).
        \qed
\end{proof}
\begin{remark}
For any well-sorted kind equality $|- \kappa=\kappa':\square$ in \Fi,
$\kappa^\circ$ and $\kappa'^\circ$ are the syntactically same \Fw\ kinds.
Note that no variables can appear in the erased kinds by definition of
the erasure operation on kinds.
\end{remark}

\begin{theorem}[index erasure on well-kinded type constructors]
\label{thm:ierasekinding}
\[ \inference{|- \Delta & \Delta |- F : \kappa}
                {\Delta^\circ |- F^\circ : \kappa^\circ}
\]
\end{theorem}
\begin{proof}
        By induction on the kinding derivation ($\Delta |- F : \kappa$).
        We use Theorem \ref{thm:ierasetyctx} in the ($Var$) case,
        Theorem \ref{thm:ierasekindeq} in the ($Conv$) case, and
        Theorem \ref{thm:ierasesorting} in the ($\lambda$) and ($\forall$) cases.
        \qed
\end{proof}
\begin{remark} In the theorem above, $F^\circ$ is a well-kinded type constructor in \Fw.
\end{remark}

\noindent
\begin{minipage}{.55\linewidth}
\begin{lemma}\label{lem:ieraseSubstTyVar}
$(F[G/X])^\circ = F^\circ[G^\circ/X]$
\end{lemma}
\end{minipage}
\begin{minipage}{.4\linewidth}
\begin{lemma}\label{lem:ieraseSubstIndex}
$(F[s/i])^\circ = F^\circ$
\end{lemma}
\end{minipage}
\begin{theorem}[index erasure on type constructor equality]
\label{thm:ierasetyconeq}
\[ \inference{\Delta |- F=F':\kappa}
                {\Delta^\circ |- F^\circ=F'^\circ:\kappa^\circ}
\]
\end{theorem}
\begin{proof}
        By induction on the derivation of the type constructor equality judgment
        ($\Delta |- F=F':\kappa$). We also use Proposition \ref{prop:wfkind}
        and Lemmas \ref{lem:ieraseSubstTyVar} and \ref{lem:ieraseSubstIndex}.
        \qed
\end{proof}
\begin{remark}
When $\Delta |- F=F':\kappa$ is a valid type constructor equality in \Fi,
$\Delta^\circ |- F^\circ=F'^\circ:\kappa^\circ$
is a valid type constructor equlity in \Fw.
\end{remark}

\begin{theorem}[index erasure on well-formed term-level contexts
                prepended by index variable selection]
\label{thm:ierasetmctxivs}
\[ \inference{\Delta |- \Gamma}{\Delta^\circ |- (\Delta^\bullet,\Gamma)^\circ}
\]
\end{theorem}
\begin{proof}
        By induction on the term-level context ($\Gamma$)
        and using Theorem \ref{thm:ierasekinding}.
        \qed
\end{proof}
\begin{remark}
We can also show that
$\inference{\Delta |- \Gamma}{\Delta^\circ |- \Gamma^\circ}$
and prove Corollary \ref{cor:ierasetypingifree} directly.
\end{remark}

\begin{theorem}[index erasure on well-typed terms]
\label{thm:ierasetypingall}
$ \inference{\Delta |- \Gamma & \Delta;\Gamma |- t : A}
                {\Delta^\circ;(\Delta^\bullet,\Gamma)^\circ |- t : A^\circ}
$
\end{theorem}
\begin{proof}
        By induction on the typing derivation ($\Delta;\Gamma |- t : A$).
        We also make use of Theorems
        \ref{thm:ierasesorting},
        \ref{thm:ierasekinding},
        \ref{thm:ierasetyconeq}, and
        \ref{thm:ierasetmctxivs}.
        \qed
\end{proof}
\begin{remark}
        In the theorem above, $t$ is a well typed term
        in \Fw\ as well as in \Fi.
        % TODO also say why this is useful (Nax predefeind term)
\end{remark}

\begin{corollary}[index erasure on index-free well-typed terms]
\label{cor:ierasetypingifree}
\[ \inference{ \Delta |- \Gamma & \Delta;\Gamma |- t : A}
                {\Delta^\circ;\Gamma^\circ |- t : A^\circ}
                {\enspace(\dom(\Delta)\cap\FV(t) = \emptyset)}
\]
\end{corollary}

\subsection{Strong Normalization and Logical Consistency} \label{ssec:sn}
Strong normalization is a corollary of the erasure property since we know that
System~\Fw\ is strongly normalizing. Logical consistency is immediate since
System~\Fi\ is a restriction of \emph{restricted implicit calculus}
\cite{Miquel00} or $\text{ICC}^{-}$ \cite{BarrasB08}, which are
restrictions of ICC~\cite{Miquel01} known to be logically consistent.
Subject reduction is also immediate for the same reason.


\chapter{Related work}\label{ch:relwork}
In this chapter, we discuss additional related work that was not
discussed in previous chapters (\S\ref{sec:intro} and \S\ref{sec:related}).
We discuss five categories of related work:
Mendler-style co-recursion schemes over co-data (\S\ref{sec:relwork:co}),
Mendler-style recursion schemes over multiple values (\S\ref{sec:relwork:mult}),
dependently-typed Mendler-style induction (\S\ref{sec:relwork:dep}), 
the use of sized-types to explain the termination of Mendler-style
recursion schemes (\S\ref{sec:relwork:sized}), and the comparison of
our Mendler-style approach to logical frameworks (\S\ref{sec:relwork:LF}).
\index{Mendler-style!co-recursion scheme}
\index{Mendler-style!multiple values}
\index{co-data}


\input{relwork_mcoit} %% sec:relwork:co

\input{relwork_mmult} %% sec:relwork:mult

\section{Mendler-style induction}
\label{sec:relwork:dep}
\index{Mendler-style!induction}
\index{regular datatype}
\index{datatype!regular}
The dependently-typed version of primitive recursion is called induction.
We formulate Mendler-style induction over regular datatypes as follows.
\vspace*{-2em}
\begin{singlespace}
\[\begin{array}{ll}
\textbf{mind}_{*} \, :
& \!\!\forall (F:* -> *) (A: \mu_{*}F -> *). \\
& ~~ \big(\forall(r:*).\;(cast : r -> \mu_{*}F) \\
& ~\qquad\quad -> (call : (x:r) -> A\,(cast~x)) \\
& ~\qquad\quad -> (y: F\;r) -> A\,(\In_{*}(\textit{fmap}_{F}\;cast\;y)) \big) 
-> (z:\mu_{*}\,f) -> A\;z
\end{array}
\]
\[
\textbf{mind}_{*} ~ \varphi ~ (\In_{*}\;x)
  ~=~ \varphi~~\textit{id}~~(\textbf{mind}_{*}~\varphi)~~x \]
\end{singlespace}\noindent
The definition of Mendler-style induction $\textbf{mind}$\footnote{
	The idea behind $\textbf{mind}_{*}$ comes from discussion with
	Tarmo Uustalu. He described this on a whiteboard when I met with him
	at the University of Cambridge in Fall 2011.}
shows that induction is essentially the same as
the Mendler-style primitive recursion $\MPr$, except
that the type signature involves dependent types.
\index{Mendler-style!primitive recursion}
\index{dependent type}
\index{type!dependent}
Note, the final answer type $(A\;z)$ is dependent on
the recursive argument $z:\mu_{*}F$.
Since $A: \mu_{*}F -> *$ expects a concrete recursive value,
we use $cast$ in the type signature of the $\varphi$ function
to cast $(x:r)$ and $(y:F\,r)$ into $\mu_{*}F$ values, so that
they can be passed to $A$.
In the type signature of $\textbf{mind}$, $cast$ comes before $call$
because the type signature of $call$ depends on $cast$.
When defining $\MPr$, $cast$ and $call$ can come in any order
since there is no dependency in the type signature of $\MPr$.

One important aspect of $\textbf{mind}_{*}$ is that
it is well-defined only over positive $F$, because we relied on
the existence of $\textit{fmap}_F$ to write its type signature.
It is an open question whether one can formulate a Mendler-style induction
that works for negative datatypes.

In the future work section (\S\ref{sec:futwork:mprsi}), we introduce 
another Mendler-style recursion scheme that is useful for
mixed-variant datatypes. The work of a Mendler stylist is never done!



\section[Type-based termination and sized types]{
	 Type-based termination and sized types\footnotemark{} }
\footnotetext{We plan to submit a modified version of this section
		as a part of the TYPES post-proceedings draft.}
\label{sec:relwork:sized}
\index{type-based termination}
\index{termination!type-based}
\index{sized type}
\index{type!sized}
\emph{Type-based termination} (coined by \citet{BartheFGPU04}) stands for
approaches that integrate termination into type checking, as opposed to
syntactic approaches that reason about termination over untyped term structures.
The Mendler-style approach is, of course, type-based.  In fact, the idea of
type-based termination was inspired by \citet{Mendler87,Mendler91}.
In Mendler style, we know that well-typed functions defined using
Mendler-style recursion schemes always terminate.  This guarantee follows
from the design of the recursion scheme, where the use of higher-rank 
polymorphic types in the abstract operations enforce the invariants
necessary for termination.

\index{type-based termination!advantage}
\citet{abel06phd,Abel12talkFICS} summarizes the advantages of
type-based termination as follows:
\emph{communication} (programmers think using types),
\emph{certification} (types are machine-checkable certificates),
\emph{a simple theoretical justification}
        (no additional complication for termination other than type checking),
\emph{orthogonality} (only small parts of the language are affected,
        \eg, principled recursion schemes instead of general recursion),
\emph{robustness} (type system extensions are less likely to
                        disrupt termination checking),
\emph{compositionality}\footnote{This is not listed in Abel's thesis,
                                but comes from his invited talk in FICS 2012.}
        (one needs only types, not the code, for checking the termination), and
\emph{higher-order functions and higher-kinded datatypes}
        (works well even for higher-order functions and non-regular datatypes,
        as a consequence of compositionality).
In his dissertation \cite{abel06phd} (Section 4.4) on sized types,
Abel views the Mendler-style approach as enforcing size restrictions
using higher-rank polymorphism as follows:
\begin{itemize}
\item The abstract recursive type $r$ in Mendler style corresponds to
        $\mu^\alpha F$ in his sized-type system (System \Fwhat),
        where the sized type
        for the value being passed in corresponds to $\mu^{\alpha+1} F$.
\item The concrete recursive type $\mu F$ in Mendler style corresponds to
        $\mu^\infty F$ since there is no size restriction.
\item \index{subtyping}
	By subtyping, a type with a smaller size-index can be cast to
        the same type with a larger size-index.
\end{itemize}
This view is based on the same intuition we discussed in
Chapter \ref{ch:mendler}. Mendler-style recursion schemes terminate --- for
positive datatypes --- because $r$-values are direct subcomponents
of the value being eliminated. They are always smaller
than the value being passed in. Types enforce that recursive calls
are only well-typed, when applied to smaller subcomponents.

\index{primitive recursion}
\index{implicit conversion}
Abel's System \Fwhat\ can express primitive recursion quite naturally
using subtyping. The casting operation $(r -> \mu F)$ in Mendler-style
primitive recursion corresponds to an implicit conversion by subtyping
from $\mu^\alpha F$ to $\mu^\infty F$ because $\alpha \leq \infty$.

System \Fwhat\ \cite{abel06phd} is closely related to
System \Fixw\ \cite{AbeMat04}. Both of these systems are base on
equi-recursive fixpoint types over positive base structures.
Both of these systems are able to embed (or simulate) Mendler-style
primitive recursion (which is based on iso-recursive types) via
the encoding \cite{Geu92} of arbitrary base structures into
positive base structures. In \S\ref{sec:fixi:data}, we rely on
the same encoding, denoted by $\Phi$, when embedding \MPr\ into System \Fixi.

Abel's sized-type approach evidences good intuition concerning the reasons
that certain recursion schemes terminate over positive datatypes.
But, it is not a useful intuition of whether or not those recursion schemes
would terminate for negative datatypes, unless there is an encoding that can
translate negative datatypes into positive datatypes. For primitive recursion,
this is possible (as we mentioned above). However, for our recursion scheme
\MsfIt, which is especially useful over negative datatypes, we do not know of
an encoding that can map the inverse augmented fixpoints into
positive fixpoints. So, it is not clear whether Abel's the sized-type approach
based on positive equi-recursive fixpoint types can provide a good explanation
for the termination of \MsfIt. In \S\ref{sec:futwork:mprsi}, we will discuss
another Mendler-style recursion scheme (\mprsi), which is also useful over
negative datatypes and has a termination property (not yet proved) based on 
the size of the index in the datatype.

\section{Logical Frameworks based on the $\lambda\Pi$-calculus}
\label{sec:relwork:LF}
\index{logical framework}
A ``logical framework'', in a broad sense, refers to
any system that serves as ``a meta-language for
the formalization of deductive systems'' \cite{Pfe02LFintro}.
In a more narrow sense, logical frameworks are systems closely related to
to the Edinburgh Logical Framework (LF or ELF) \cite{Harper87}, which uses
the $\lambda\Pi$-calculus as its specification language. In this section,
we discuss \emph{logical frameworks} in this more narrow sense.

The $\lambda\Pi$-calculus (\aka\ $\lambda\mathbf{P}$) is one of the corners
in Barendregt's $\lambda$-cube \cite{Barendregt91} that is adjacent to
the simply-typed lambda calculus (STLC, or, $\lambda\!\!\rightarrow$).
The $\lambda\Pi$-calculus extends the STLC with dependent types,
but without polymorphism or functions from types to types (type operators).
The syntax of $\lambda\Pi$, extended with constants ($c$), is describe below:
\[
\begin{array}{lll}
\text{Kinds}         & K   & ::= \texttt{type} \mid \Pi x:A.K
        \\
\text{Type Families} & A,B & ::= c \mid \Pi x:A.B \mid \lambda x:A.B \mid AM
        \\
\text{Objects}       & M,N & ::= c \mid x \mid \l x:A.M \mid MN
\end{array}
\]
In logical frameworks, one can introduce new constants by naming types and
objects. These are intended to represent datatypes such as natural numbers,
lists, and may even involve higher-order abstract syntax. However,
these constants are merely syntactic descriptions, not necessary tied to
any specific semantics or logical interpretations. That is, introducing
constants does not automatically supply any recursion schemes or
induction principles, as is done in functional languages or proof assistants
that support new datatypes as a feature. Each logical framework
supports its own meta-logic to give meanings to the logic (or, the language)
specified by introducing such constants. The choice of meta-logic can be either
relational (like a logic programming language),
functional (like a functional programming language), or something else.

Logical frameworks are very flexible for describing many
different logical systems (i.e., formalizing a language) by using
a two-layered approach of a minimal specification language ($\lambda\Pi$)
and a meta-logic. However, this two-layered approach is not ideal
as a programming system. One can model arbitrary
programming languages, giving them semantics
in the logical framework. But, the programming capability of the 
specification language and the meta-logic is limited.
In the remainder of this section, we discuss Twelf, whose meta-logic
is relational, and, Beluga and Delphin, whose meta-logics are functional.

\paragraph{}
Twelf\footnote{\url{http://twelf.org/}} is the most widely used
logical framework. In Twelf, you can define abstract syntax for datatypes
by introducing constants for types and objects involving those types.
For example, you can define natural numbers as follows.\footnote{
        Twelf examples are adopted from Boyland's Twelf Library on Github.\\
        $~~~~~~~$
        \url{https://github.com/boyland/twelf-library}}\vspace*{-2em}
\begin{singlespace}
\begin{verbatim}
  nat : type.       %%% define a type constant
  z : nat.          %%% define a constant for zero
  s : nat -> nat.   %%% define a constant for sucessor
\end{verbatim}
\end{singlespace}\noindent
At this point, the constants \texttt{z} and \texttt{s} are just typed syntax.
Introduction of constants is not associated with any semantics for
the constants, unlike the natively supported inductive datatypes
in Coq or Agda. So, there are no restrictions on how these constants
may be used, such as the positivity constraint on inductive datatypes
in Coq or Agda. One can give meanings to the natural number constants
by defining inductive relations over them. For example, we can define
addition as a ternary relation over natural numbers, as follows:\vspace*{-2em}
\begin{singlespace}
\begin{verbatim}
  plus : nat -> nat -> nat -> type.
  plus/z : plus z Y Y.
  plus/s : plus (s X) Y (s Z)
        <- plus X Y Z.
\end{verbatim}
\end{singlespace}\noindent
The right-hand sides (after the colon) of \verb|plus/z| and \verb|plus/s|
look like a Prolog program defining addition. Twelf's meta-logic is, in fact,
typed first-order relational logic. At the type-level, Twelf predicates are
like pure Prolog programs with type-checking. All computational issues,
such as termination checking, are present at the level of these
relational definitions (as opposed to the introduction of new constants).
Twelf has a termination checker for inductive relations (external to
the type checker) based on lexicographic subterm ordering over untyped terms.
In addition to the type signatures of the relations, one can optionally specify
input/output modes for each of their arguments, if necessary, in order to guide
the termination checker to consider only the input arguments
for termination.\footnote{There are various directives to guide
        checking input/output modes, coverage, and termination in Twelf.
        For further information, see the  documentations from its homepage.}

One cannot write higher-order relations natively in Twelf
because Twelf's meta-logic is first-order, not higher-order.
To write a program using higher-order functions in Twelf, one has to model
one's own object language that is able to support higher-order functions,
then program within that object language rather than programming in
Twelf's meta-logic. We summarize the steps necessary to program
using higher-order functions in Twelf:
\begin{itemize}
\item[(1)] Define an object language syntax
        (as the syntax \texttt{z} and \texttt{s} for natural numbers)
        with bindings (this is done by HOAS), applications, and whatever
        needed to express higher-order functions.
\item[(2)] Define the evaluation semantics of the object language using
        inductive relations (\ie, write an evaluation relation for
        the object language in a Prolog-like way).
\item[(3)] Write programs in the object language by putting
        together pieces of the syntax you defined in (1).
\item[(4)] Finally, evaluate the program by reasoning based on
        the evaluation relation defined in (2).
\end{itemize}
This process is clearly not ideal if the desire is simply to ``program''
with higher-order functions in a type-safe way, possibly with some
termination guarantees. One does not always want to reason about
the meta-theory of the object language in general.

\paragraph{}
Beluga \cite{Pie10} is similar to Twelf, but it is closer to
a functional language since the inductive definitions are functional,
rather than relational. Beluga supports higher-order functions, unlike Twelf.
One can write a natural number addition function in Beluga 
as follows:\footnote{
        Adopted from the Beluga tutorial.
        \url{http://complogic.cs.mcgill.ca/beluga/} }\vspace*{-2em}
\begin{singlespace}
\begin{verbatim}
  rec add : [. nat ] -> [. nat ] -> [. nat ]
    = fn x => fn y =>
      case x of
      | [. z ]   => y
      | [. s N ] => let [. R ] = add [. N ] y in [. s R ]
      ;
\end{verbatim}
\end{singlespace}\noindent
Types like \verb|nat| and \verb|nat -> nat| are called
representation-level types. So, objects like \verb|z| and \verb|s|
are called representation-level objects. Types like \verb|[. nat ]|
and \verb|[. nat ] -> [. nat ]| are called computation-level types.
This \verb|add| function definition is almost identical to typical
recursive function definition of natural number addition
in functional languages, except for the new representation-level
variable binding \verb|R| in the second case branch. In Beluga,
one cannot write \verb|[. s (add [. N ] y) ]| because \verb|s| expects
a representation-level object as its argument.
In Twelf-style logical frameworks, representation-level types are inhabited
only by $\eta$-long $\beta$-normal representation-level objects, which do not
include application forms of computational-level objects.

More generally, computation-level types can have the form \verb|[g . t]|
where \verb|g| is a context object and \verb|t| is a representation-level type.
One of the Beluga's unique features is supporting pattern matching over
computational objects with contexts, and also coverage checking of those
patterns. Computational types with the empty context, of the form
\verb|[. t]|, are inhabited by closed values, which do not involve
any free (representation-level) variables.
Since Beluga has explicit access to context objects, we believe that
it can express what \MsfIt\ can express, and in addition, it can
also express what \textbf{openit} (\S\ref{sec:openit}) can express.

One can also write higher-order functions (\eg, map)\footnote{
        A map function over natural number lists is
        given in the Beluga tutorial. }
in Beluga almost the same way one does in functional programming languages,
except, perhaps, for the tedious representation-level bindings 
(\eg, \verb|R| in the \verb|add| function above).
In regards to higher-order functions, Beluga is in a much better position
than Twelf. Recall that, in Twelf, one needs to model a whole new
functional language by describing its semantics with inductive relations
in order to express higher-order functions.

Termination is not type based in Beluga either. Like Twelf, it needs
an external termination (or totality) checker, but its prototype
implementation currently lacks such a checker. We suspect one of the reasons
why the Beluga implementation does not yet contain a termination checker is
due to the difficulty of checking termination of higher-order functions.
The syntactic approaches to termination, used by logical frameworks
based on first-order meta-logic, may fail to check termination for
many higher-order functions.

\paragraph{}
Delphin \cite{pos08phd} has goals similar to Beluga,
supporting functional programming rather than relational reasoning.
For example, the addition function over natural numbers can be defined
in Delphin as follows.\vspace{-2em}
\begin{singlespace}
\begin{verbatim}
fun plus : <nat> -> <nat> -> <nat> 
  = fn <z>   <M> => <M>
    |  <s N> <M> => let val <x> = plus <N> <M> in <s x> end
    ;
\end{verbatim}
\end{singlespace}\noindent
Although both Beluga and Delphin support similar features with similar syntax,
their theoretical foundations differ \cite{Pie10} on how they treat contexts.
Delphin cannot distinguish open values from closed values as is done in Beluga,
since Delpin does not explicitly manage contexts.
\citet{Pie10} also points out that Delphin tries to reuse Twelf's infrastructure
as much as possible. For instance, the termination checker of Delphin is
based on lexicographic subterm ordering, which is also the case in Twelf.

Although Delphin and Beluga do support higher-order functions,
they do not support polymorphism, but only dependent types by term indexing.
That is, one can only write monomorphic functions. Recall that,
in $\lambda\Pi$, one can only index type families by terms, not types.
Indexing by types would support polymorphism.
This is inconvenient for programming higher-order functions,
because many higher-order functions are polymorphic in nature; users
need to duplicate their definitions for each different type needed.
%% One may work around this limitation by using type-representation objects
%% -- define constant objects that represent types and pass them around instead
%% of passing types.


\section{Summary and Ongoing Work} \label{sec:concl}
System~\Fi\ is a strongly-normalizing, logically-consistent, higher-order
polymorphic lambda calculus that was designed to support the
definition of datatypes indexed by both terms and types.
In terms of expressivity, System~\Fi\ sits between System~\Fw\ and ICC.
We designed System~\Fi\ as a tool to reason about programming
languages with term-indexed datatypes. System \Fi\ can express
a large class of term-indexed datatypes, including datatypes with nested term
indices.

One limitation of System \Fi\ is that it cannot express type level
data structures such as lists that contain type elements.
We hope to overcome this limitation by extending System \Fi\ 
with first-class type representations \cite{DagMcb12}, which reflect
types as term level data (a sort of fully reflective version of
{\small\tt TypeRep} from Section \ref{intro}).

Our goal is to to build a unified programming and reasoning system,
which supports
(1) an expressive class of datatypes including nested term-indexed datatypes
and negative datatypes,
(2) logically consistent reasoning about program properties, and
(3) Hindley--Milner-style type inference.
Towards this goal, we are developing the programming language Nax \cite{AhnSheFioPit12},
which is based on System \Fi. Nax is given semantics in terms of System~\Fi.
That is, all the primitive language constructs of Nax that are not present
in \Fi\ have translations into System \Fi. Such constructs include
Mendler-style recursion combinators, recursive type operator, and
pattern matching.

Some language features we want to include in Nax go beyond \Fi.
There are recursion schemes that guarantee normalization due to paradigmatic use
of indices in datatypes. For instance, some recursive computations always
reduce a natural number term-index in every recursive call.
Although we know that such computations obviously terminate, we cannot express
them in System \Fi, since term-indices in \Fi\ are erasable -- \Fi\ only
accepts terms that are already type-correct in \Fw.
We plan to explore extensions to \Fi\ that enable such computations
while maintailing logical consistency.






\appendix
\section{Appendix Title}

This is the text of the appendix, if you need one.

\acks

Acknowledgments, if needed.

% We recommend abbrvnat bibliography style.

\bibliographystyle{abbrvnat}

% The bibliography should be embedded for final submission.

\bibliography{main}

\end{document}
