\section{Related work}
\label{sec:relwork}
Among theoretical calculi, \Fi\ is most closely related to
Curry-style System \Fw \cite{AbeMatUus03,AbeMatUus05,GHR93}
and Implicit Calculus of Constructions (ICC) \cite{Miquel01}.
All terms typable in Curry-style System \Fw\ are typable in System \Fi\ 
with the same type, and all terms typable in \Fi\ are typable in ICC
with the same type.\footnote{The $*$ kind in \Fw\ and \Fi\ corresponds
	to \textsf{Set} in ICC.}
We have discussed that we can derive strong normalization, logical consistency,
and subject reduction of \Fi, from \Fw\ and a subset of ICC.
In fact, ICC is more than just an extension of \Fi, as described in our work,
with dependent types and stratified universes. ICC includes $\eta$-reduction
and the extensionality typing rule. We do not foresee any problem of adding
$\eta$-reduction and the extensionality typing rule to \Fi. Although
System \Fi\ accepts less terms than ICC, \Fi\ enjoys more simple
erasure properties (Theorem \ref{thm:ierasetypingifree} and
Theorem \ref{thm:ierasetypingall}), which ICC cannot not provide
due to its support for full dependent types. In System \Fi, index terms
appearing in types (\eg, $s$ in $F\{s\}$) are always erasable.
\citet{LingerS08} formalized a more generic framework than ICC, which describes
the erasure on arbitrary Church-style calculi (EPTS) and Curry-style calculi
(IPTS), but only consider $\beta$-equivalence for type conversion.

In \S\ref{ssec:rationale}, we have mentioned that Curry-style calculi enjoys
better reduction properties (\eg,$\beta\eta$-reduction is Church-Rosser)
than Church-style calculi. \citet{Nederpelt73} showed a counterexample to
the Church-Rosser property for $\beta\eta$-reduction of Church-style terms.
\citet{Geuvers92} proved that $\beta\eta$-reduction is Church-Rosser
in functional PTSs, which is a certain class of Church-style calculi.
\citet{Seldin08} discusses the relation between the Church-style typing
and the Curry-style typing.

In a more practical setting for language implementation,

\citet{YorgeyWCJVM12}, inspired by \cite{SHE}, desinged a language extension
to Haskell, promoting datatypes to be used as kinds. For instnace, \texttt{Bool}
is promoted to a kind (\ie, $\texttt{Bool}:\square$) and its data constructors
\texttt{Ture} and \texttt{False} are promoted to type level. To support this
in GHC, they extended System $F_{\!C}$ (GHC's intermediate language, or,
GHC Core), naming the extended GHC Core as System $F_{\!C}^\uparrow$.
The key differnce between $F_{\!C}^\uparrow$ and \Fi\ is in the extension
to the kind syntax, as illustrated below: \vspace*{-2pt}
\[\qquad\quad
\begin{array}{ll}
F_{\!C}^\uparrow\,\textbf{kinds}: &
\kappa ::= * \mid \kappa -> \kappa \mid F \vec{\kappa} \mid \mathcal{X} \mid \forall \mathcal{X}.\kappa \mid \cdots \\
\,\Fi\,\,\textbf{kinds}: &
\kappa ::= * \mid \kappa -> \kappa \mid A -> \kappa \phantom{A^{A^A}}
\end{array}  
\] ~\vspace*{-6pt}\\
In $F_{\!C}^\uparrow$, any type constructor (\texttt{F}) is promoted to 
kind level and becomes a kind when fully appied to other kinds
($\texttt{F}\vec\kappa$). In \Fi, on the other hand, a type can only appear
at the domain of an index arrow kind ($A-> \kappa$). This seemingly small
difference of the kind syntax makes $F_{\!C}^\uparrow$ be a drastically more
expressive language. The promotion of type constructors such as
$\texttt{List}:* -> *$ to a kind cosntructor $\texttt{List}:\square -> \square$
enables data structuers at type level (\eg, $\mathtt{[Nat,Bool,Nat-> Bool]}$).
Also, the promotion polymorphic types naturally motivates kind polymorphism
($\forall \mathcal{X}.\kappa$), which is known to break strong normalization
and cause logical inconsistency \cite{Girard72}. For the purpose of extending
a functional programming language, inconsistency is not a problem. But for
our interest of studying term-indexed datatypes in a logically consistent
calculi, we need a more conservative approach, as in \Fi, starting from
smallest possible extension that does not break normalization or consistency.


\citet{Swamy11}
value dependent types in F-star  from MSR


what others to discuss?

Translating Generalized Algebraic Data Types to System F
Martin Sulzmann and Meng Wang

Stephanie's Rw and related work.
they prove parametricity in the presense of indices of GADTs
