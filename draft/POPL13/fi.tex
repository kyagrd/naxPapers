\section{System \Fi}
\label{sec:Fi}
System \Fi\ is a higher-order polymorphic lambda calculus with term indices.
System \Fi\ was designed to extend System \Fw\ by the inclusion of term indices.
The complete syntax and rules of \Fi\ are described in \Fig{Fi} and
\Fig{eqFi}. The syntax and rules are highlighted by \newFi{\text{grey boxes}}.
The extensions new to \Fi, which are not originally part of \Fw, appear in the boxes.
If one excludes all the grey boxes from \Fig{Fi}
and \Fig{eqFi}, one obtains a version of \Fw. In particular, it is a version of \Fw\
with Curry-style terms and typing contexts separated into two parts
(type level and term level contexts). We first discuss the rational for
these design choices 
(\S\ref{ssec:rationale}), and then introduce the new constructs of \Fi,
which are not found in \Fw\ (\S\ref{ssec:newFi}).

\subsection{Rationale for the design choices} \label{ssec:rationale}
Terms in \Fi\ are Curry style. That is, term level abstractions are unannotated
($\lambda x.t$), and type generalizations ($\forall I$) and type instantiations
($\forall E$) are implicit at term level. A Curry-style calculus generally has
an advantage over its Church-style counterpart when reasoning about properties of
reduction. For instance, the Church-Rosser property naturally holds for 
$\beta$-, $\eta$-, and $\beta\eta$-reduction in the Curry style, but
may not hold in the Church style. This is due to the presence of annotations in
abstractions \cite{Miquel01}.\footnote{The Church-Rosser property,
in its strictest sense (\ie, $\alpha$-equivalence over terms),
generally does not hold in Church-style calculi , but may hold under
	certain approximations, such as modulo ignoring the annotations
	in abstractions.}

Type constructors, on the other hand, remain Church style in \Fi. That is, type level abstractions are
annotated by kinds ($\lambda X^\kappa.F$). Choosing type constructors
to be Church style makes the kind of
a type constructor visually explicit. The choice of style for type constructors
is not as crucial as the choice of style for terms, since the syntax and
kinding rules at type level are essentially a simply typed lambda calculus.
Annotating the type level abstractions with kinds makes kinds
explicit in the type syntax. Since \Fi\ is essentially an extension of \Fw\
with a new formation rule for kinds, making kinds explicit is a pedagogical
tool to emphasize the consequences of this new formation rule.
As a notational convention, we write
$A$ and $B$, instead of $F$ and $G$, where $A$ and $B$ to are expected
to be types (\ie, nullary type constructors) of kind $*$.

In a language with term indices, terms appear in types (e.g., the length index
$(n+m)$ in the type $\textit{Vec}\;\textit{Nat}\;\{n+m\}$).
Such terms contain variables. The binding sites of these variables matter.
In \Fi, we expect such variables to be statically bound. Dynamically bound
index variables would require a dependently typed calculus, such as
the calculus of constructions. To reflect this design choice,
typing contexts are separated into type level contexts ($\Delta$) and
term level contexts ($\Gamma$). Type level (static) variables ($X$ , $i$) are
bound in $\Delta$ and term (dynamic) variables ($x$) are bound in $\Gamma$.
Type level variables are either type constructor variables ($X$) or
term variables to be used as indices ($i$). As a notational convention,
we write $i$, instead of $x$, when term variables are to be used as indices
(\ie, introduced by either index abstraction or index polymorphism).

In contrast to our design choice, System \Fw\ is most often formalized using
a single context, which binds both type variables~($X$) and term 
variables~($x$). 
In such a formalization, the free type variables in the typing of
the term variable must be bound earlier in the context. For example,
if $X$ and $Y$ appear free in the type of $f$, they must appear earlier
in the single context ($\Gamma$) as below:
\[ \Gamma = \dots,X^{\mathtt*},\dots,Y^{\mathtt*},\dots,
		(f:\forall Z^{\mathtt*}.X -> Y -> Z),\dots \]
In such a formalization, the side condition ($X\notin\Gamma$)
in the $(\forall I)$ rule of Figure \ref{fig:Fi} is not necessary,
since such a condition is already a part of the well-formedness condition
for the context (\ie, $\Gamma,X^\kappa$ is well-formed when
$X\notin\FV(\Gamma)$). Thus, for \Fw, it is only a matter of taste
whether to formalize the system using a single context or two contexts,
since they are equivalent formalizations with comparable complexity.

However, in \Fi, we separate the context into two parts to distinguish
term variables used in types (which we call index variables, or indices,
and are bound as $\Delta,i^A$) from the ordinary use of term variables
(which are bound as $\Gamma,x : A$). The expectation is that indices
should have no effect on reduction at the term level.
Although it is imaginable to formalize \Fi\ with a single typing context
and distinguish index variables from ordinary term variables using
more general concepts (\eg, capability, modality), we think that splitting
the typing context into two parts is the simplest solution.

\begin{figure*}
\paragraph{Syntax:}
\begin{align*}
\!\!\!\!\!\!\!\!&\text{Term Variables}
 	& x,i
\\
\!\!\!\!\!\!\!\!&\text{Type Constructor Variables}
 	& X
\\
\!\!\!\!\!\!\!\!&\text{Kinds}
 	& \kappa		&~ ::= ~ *
				\mid \kappa -> \kappa
				\mid \newFi{A -> \kappa}
\\
\!\!\!\!\!\!\!\!&\text{Type Constructors}
	& A,B,F,G		&~ ::= ~ X
				\mid A -> B
				\mid \lambda X^\kappa.F
				\mid F\,G
				\mid \forall X^\kappa . B
				\mid \newFi{\lambda i^A.F
				\mid F\,\{s\}
				\mid \forall i^A . B}
\\
\!\!\!\!\!\!\!\!&\text{Terms}
	& r,s,t			&~ ::= ~ x \mid \lambda x.t \mid r\;s
\\
\!\!\!\!\!\!\!\!&\text{Typing Contexts}
	& \Delta		&~ ::= ~ \cdot
				\mid \Delta, X^\kappa
				\mid \newFi{\Delta, i^A} \\
&	& \Gamma		&~ ::= ~ \cdot
				\mid \Gamma, x : A
\end{align*}

\paragraph{Well-formed typing contexts:}
\[ \fbox{$|- \Delta$}
 ~~~~ ~~~~
   \inference{}{|- \cdot}
 ~~~~
   \inference{|- \Delta & |- \kappa:\square}
             {|- \Delta,X^\kappa}
      \big( X\notin\dom(\Delta) \big)
 ~~~~ \newFi{
   \inference{|- \Delta & \cdot |- A:*}
             {|- \Delta,i^A}
      \big( i\notin\dom(\Delta) \big) }
\]
\[ \fbox{$\Delta |- \Gamma$}
 ~~~~
   \inference{|- \Delta}{\Delta |- \cdot}
 ~~~~
   \inference{\Delta |- \Gamma & \Delta |- A:*}
             {\Delta |- \Gamma,x:A}
      \big( x\notin\dom(\Gamma) \big)
\]
~\\
\paragraph{Sorting:} \fbox{$|- \kappa : \square$}
$ \qquad
 ~~~~
  \inference[($A$)]{}{|- *:\square}
 ~~~~
   \inference[($R$)]{|- \kappa:\square & |- \kappa':\square}
                    {|- \kappa -> \kappa' : \square}
 ~~~~
   \newFi{
   \inference[($Ri$)]{\cdot |- A:* & |- \kappa:\square}
                     {|- A -> \kappa : \square} }
$
~\\ ~\\
\paragraph{Kinding:} \fbox{$\Delta |- F : \kappa$}
$ \quad
 ~~~~
   \inference[($Var$)]{X^\kappa\in\Delta & |- \Delta}
                       {\Delta |- X : \kappa}
 ~~~~
   \inference[($->$)]{\Delta |- A : * & \Delta |- B : *}
                     {\Delta |- A -> B : * }
$
\[
  \inference[($\lambda$)]{|- \kappa:\square & \Delta,X^\kappa |- F : \kappa'}
                          {\Delta |- \lambda X^\kappa.F : \kappa -> \kappa'}
 ~~~~
   \inference[($@$)]{ \Delta |- F : \kappa -> \kappa'
                    & \Delta |- G : \kappa }
                    {\Delta |- F\,G : \kappa'}
 ~~~~
   \inference[($\forall$)]{|- \kappa:\square & \Delta, X^\kappa |- B : *}
                          {\Delta |- \forall X^\kappa . B : *}
\]
\[ \newFi{
  \inference[($\lambda i$)]{\cdot |- A:* & \Delta,i^A |- F : \kappa}
                            {\Delta |- \lambda i^A.F : A->\kappa}
 ~~~~
   \inference[($@i$)]{ \Delta |- F : A -> \kappa
                     & \Delta;\cdot |- s : A }
                     {\Delta |- F\,\{s\} : \kappa}
 ~~~~
   \inference[($\forall i$)]{\cdot |- A:* & \Delta, i^A |- B : *}
                            {\Delta |- \forall i^A . B : *} }
\]
\[ \newFi{
   \inference[($Conv$)]{ \Delta |- A : \kappa
                       & \Delta |- \kappa = \kappa' : \square }
                       {\Delta |- A : \kappa'} }
\]
~\\
\paragraph{Typing:} \fbox{$\Delta;\Gamma |- t : A$}
$ \qquad
 ~~~~
 \inference[($:$)]{(x:A) \in \Gamma & \Delta |- \Gamma} 
                    {\Delta;\Gamma |- x:A}
 ~~~~ \newFi{
   \inference[($:i$)]{i^A \in \Delta & \Delta |- \Gamma} 
                     {\Delta;\Gamma |- i:A} }
$
\[
   \inference[($->$$I$)]{\Delta |- A:* & \Delta;\Gamma,x:A |- t : B}
                        {\Delta;\Gamma |- \lambda x.t : A -> B}
 ~~~~ ~~~~
   \inference[($->$$E$)]{\Delta;\Gamma |- r : A -> B & \Delta;\Gamma |- s : A}
                        {\Delta;\Gamma |- r\;s : B}
\]
\[ \inference[($\forall I$)]{|- \kappa:\square & \Delta, X^\kappa;\Gamma |- t : B}
                            {\Delta;\Gamma |- t : \forall X^\kappa.B}
			    (X\notin\FV(\Gamma))
 ~~~~ ~~~~
   \inference[($\forall E$)]{ \Delta;\Gamma |- t : \forall X^\kappa.B
                            & \Delta |- G:\kappa }
                            {\Delta;\Gamma |- t : B[G/X]}
\]
\[ \newFi{
   \inference[($\forall I i$)]{\cdot |- A:* & \Delta, i^A;\Gamma |- t : B}
                              {\Delta;\Gamma |- t : \forall i^A.B}
   \left(\begin{matrix}
		i\notin\FV(t),\\
		i\notin\FV(\Gamma)\end{matrix}\right)
 ~~~~
   \inference[($\forall E i$)]{ \Delta;\Gamma |- t : \forall i^A.B
                              & \Delta;\cdot |- s:A}
                              {\Delta;\Gamma |- t : B[s/i]} }
\]
\[ \inference[($=$)]{\Delta;\Gamma |- t : A & \Delta |- A = B : *}
                    {\Delta;\Gamma |- t : B}
\]
~\\
\paragraph{Reduction:} \fbox{$t \rightsquigarrow t'$}
$ 
 ~~~~
   \inference{}{(\lambda x.t)\,s \rightsquigarrow t[s/x]}
 ~~~~
   \inference{t \rightsquigarrow t'}{\lambda x.t \rightsquigarrow \lambda x.t'}
 ~~~~
   \inference{r \rightsquigarrow r'}{r\;s \rightsquigarrow r'\;s}
 ~~~~
   \inference{s \rightsquigarrow s'}{r\;s \rightsquigarrow r\;s'}
$
~\\ ~\\
\caption{Syntax, Typing rules, and Reduction rules of \Fi}
\label{fig:Fi}
\end{figure*}

\begin{figure*}
\paragraph{Kind equality:} \fbox{$|- \kappa=\kappa' : \square$}
$ \quad
 ~~~~
   \inference{}{|- * = *:\square}
 ~~~~
   \inference{ |- \kappa_1 = \kappa_1' : \square
             & |- \kappa_2 = \kappa_2' : \square }
             {|- \kappa_1 -> \kappa_2 = \kappa_1' -> \kappa_2' : \square}
 ~~~~ \newFi{
   \inference{\cdot |- A=A':* & |- \kappa=\kappa':\square}
             {|- A -> \kappa = A' -> \kappa' : \square} }
$
\[ \inference{|- \kappa=\kappa':\square}
             {|- \kappa'=\kappa:\square}
 ~~~~
   \inference{ |- \kappa =\kappa' :\square
             & |- \kappa'=\kappa'':\square}
             {|- \kappa=\kappa'':\square}
\]
~\\
\paragraph{Type constructor equality:} \fbox{$\Delta |- F = F' : \kappa$}
$ \quad
 ~~~~
   \inference{\Delta,X^\kappa |- F:\kappa' & \Delta |- G:\kappa}
             {\Delta |- (\lambda X^\kappa.F)\,G = F[G/X]:\kappa'}
 ~~~~ \newFi{
   \inference{\Delta,i^A |- F:\kappa & \Delta;\cdot |- s:A}
             {\Delta |- (\lambda i^A.F)\,\{s\} = F[s/i]:\kappa} }
$
\[ \inference{\Delta |- X:\kappa }{\Delta |- X=X:\kappa}
 ~~~~
   \inference{\Delta |- A=A':* & \Delta |- B=B':*}{\Delta |- A-> B=A'-> B':*}
\]
\[ \inference{|- \kappa:\square & \Delta,X^\kappa |- F=F' : \kappa'}
             {\Delta |- \lambda X^\kappa.F=\lambda X^\kappa.F':\kappa-> \kappa'}
 ~~~~
   \inference{\Delta |- F=F':\kappa->\kappa' & \Delta |- G=G':\kappa}
             {\Delta |- F\,G = F'\,G' : \kappa'}
 ~~~~
   \inference{|- \kappa:\square & \Delta,X^\kappa |- B=B':*}
             {\Delta |- \forall X^\kappa.B=\forall X^\kappa.B':*}
\]
\[ \newFi{
   \inference{\cdot |- A:* & \Delta,i^A |- F=F' : \kappa}
             {\Delta |- \lambda i^A.F=\lambda i^A.F' : A -> \kappa}
 ~~~~
   \inference{\Delta |- F=F':A->\kappa & \Delta;\cdot |- s=s':A}
             {\Delta |- F\,\{s\} = F'\,\{s'\} : \kappa}
 ~~~~
   \inference{\cdot |- A:* & \Delta,i^A |- B=B':*}
             {\Delta |- \forall i^A.B=\forall i^A.B':*} }
\]
\[ \inference{\Delta |- F = F' : \kappa}{\Delta |- F' = F : \kappa}
 ~~~~
   \inference{\Delta |- F = F' : \kappa & \Delta |- F' = F'' : \kappa}
             {\Delta |- F = F'' : \kappa}
\]
~\\
\paragraph{Term equality:} \fbox{$\Delta;\Gamma |- t = t' : A$}
$ \qquad
 ~~~~
   \inference{\Delta;\Gamma,x:A |- t:B & \Delta;\Gamma |- s:A}
             {\Delta;\Gamma |- (\lambda x.t)\,s=t[s/x] : B}
 ~~~~
   \inference{\Delta;\Gamma |- x:A}{\Delta;\Gamma |- x=x:A}
$
\[ \inference{\Delta |- A:* & \Delta;\Gamma,x:A |- t=t':B}
             {\Delta;\Gamma |- \lambda x.t = \lambda x.t':B}
 ~~~~
   \inference{\Delta;\Gamma |- r=r':A-> B & \Delta;\Gamma |- s=s':A}
             {\Delta;\Gamma |- r\;s=r'\;s':B}
\]
\[ \inference{|- \kappa:\square & \Delta, X^\kappa;\Gamma |- t=t' : B}
             {\Delta;\Gamma |- t=t' : \forall X^\kappa.B}
	     (X\notin\FV(\Gamma))
 ~~~~ ~~~~
   \inference{ \Delta;\Gamma |- t=t' : \forall X^\kappa.B
             & \Delta |- G:\kappa }
             {\Delta;\Gamma |- t=t' : B[G/X]}
\]
\[ \newFi{
   \inference{\cdot |- A:* & \Delta, i^A;\Gamma |- t=t' : B}
             {\Delta;\Gamma |- t=t' : \forall i^A.B}
   \left(\begin{smallmatrix}
		i\notin\FV(t),\\
		i\notin\FV(t'),\\
		i\notin\FV(\Gamma)\end{smallmatrix}\right)
 ~~~~
   \inference{ \Delta;\Gamma |- t=t' : \forall i^A.B
             & \Delta;\cdot |- s:A}
             {\Delta;\Gamma |- t=t' : B[s/x]} }
\]
\[ \inference{\Delta;\Gamma |- t=t':A}{\Delta;\Gamma |- t'=t:A}
 ~~~~
   \inference{\Delta;\Gamma |- t=t':A & \Delta;\Gamma |- t'=t'':A}
             {\Delta;\Gamma |- t=t'':A}
\]
~\\
\caption{Equality rules of \Fi}
\label{fig:eqFi}
\end{figure*}

\subsection{The constructs new to \Fi\ compared to \Fw} \label{ssec:newFi}
We expect readers to be familiar with \Fw\
and focus on describing the new constructs of \Fi, which appear in the grey boxes.


\paragraph{Kinds.}
The key extension to \Fw\ is the addition of term-indexed arrow kinds of
the form \newFi{A -> \kappa}. This allows type constructors to have terms
as indices. The rest of the development of \Fi\ flows naturally from
this single extension.

\paragraph{Sorting.} \label{sorting}
The formation of indexed arrow kinds is
governed by the sorting rule \newFi{(Ri)}. The rule $(Ri)$ specifies that
an indexed arrow kind $A -> \kappa$ is well-sorted when $A$ has kind $*$
under the empty type level context ($\cdot$) and $\kappa$ is well-sorted.

Requiring the use of the empty context avoids dependent kinds (\ie, kinds depending on type level or value level
bindings). The type $A$ appearing in
the index arrow kind $A -> \kappa$ must be well-kinded under
the empty type level context ($\cdot$).
That is, $A$ should to be a closed type of kind $*$,
which does not contain any free type variables or index variables.
For example, $(\textit{List}\,X -> *)$ is not a well-sorted kind,
while $((\forall X^{*}\!.\,\textit{List}\,X) -> *)$ is a well-sorted kind.

\paragraph{Typing contexts.}
Typing contexts are split into two parts.
Type level contexts ($\Delta$) for type level (static) bindings,
and term level contexts ($\Gamma$) for term level (dynamic) bindings.
A new form of index variable binding ($i^A$) can appear in
type level contexts in addition to the traditional type variable bindings ($X^\kappa$).
There is only one form of term level binding ($x:A$) that appears in
term level contexts.

\paragraph{Well formed typing contexts.}
A type level context $\Delta$ is well-formed if (1) it is either empty,
or (2) extended by a type variable binding $X^\kappa$ whose kind $\kappa$ is
well-sorted under $\Delta$, or (3) extended by an index binding $i^A$
whose type $A$ is well-kinded under the empty type level context at kind $*$.
This restriction is similar to the one that occurs in the sorting rule ($Ri$)
for term-indexed arrow kinds (see the paragraph {\bf\textit{Sorting}}).
The consequence of this is that, in typing contexts and in sorts,
$A$ must be closed type (not a type constructor!) without free variables.

A term level context $\Gamma$ is well-formed under a type level context
$\Delta$ when it is either empty or extended by a term variable binding
$x:A$ whose type $A$ is well-kinded under $\Delta$.


\paragraph{Type constructors and their kinding rules.}
We extend the type constructor syntax by three constructs,
and extend the kinding rules accordingly for these new constructs.

\newFi{\lambda i^A.F} is the type level abstraction over an index
(or, index abstraction). Index abstractions introduce indexed arrow kinds
by the kinding rule \newFi{(\lambda i)}. Note, the use of the new form of context
extension, $i^A$, in the kinding rule ($\lambda i$).


\newFi{F\,\{s\}} is the type level index application. In contrast to
the ordinary type level application ($F\,G$) where the argument ($G$) is
a type constructor, the argument of an index application ($F\,\{s\}$) is
a term ($s$). We use the curly bracket notation around an index argument in a type to
emphasize the transition from ordinary type to term, and to emphasize
that $s$ is an index term, which is erasable. Index applications eliminate
indexed arrow kinds by the kinding rule \newFi{(@i)}. Note, we type check
the index term ($s$) under the current type level context paired with
the empty term level context ($\Delta;\cdot$) since we do not want
the index term ($s$) to depend on any term level bindings. Allowing such
a dependency would admit true dependent types.

\newFi{\forall i^A . B} is an index polymorphic type.
The formation of indexed polymorphic types is governed by
the kinding rule \newFi{\forall i}, which is very similar to
the formation rule ($\forall$) for ordinary polymorphic types.

In addition to the rules ($\lambda i$), ($@ i$), and ($\forall i$),
we need a conversion rule \newFi{(Conv)} at kind level. This is because
the new extension to the kind syntax $A -> \kappa$ involves types.
Since kind syntax involves types, we need more than simple structural
equality over kinds. The new equality over kinds is the usual structural equality
extended by type constructor equality when comparing indexed arrow kinds
(see \Fig{eqFi}).

\paragraph{Terms and their typing rules}
The term syntax is exactly the same as other Curry-style calclui.
We write $x$ for ordinary term variables introduced by
term level abstractions ($\lambda x.t$).
We write $i$ for index variables introduced by
index abstractions ($\lambda i^A.F$) and by
index polymorphic types ($\forall i^A.B$). As discussed earlier, the distinction between
$x$ and $i$ is for the convenience of readability.

Since \Fi\ has index polymorphic types ($\forall i^A . B$),
we need typing rules for index polymorphism:
\newFi{(\forall I i)} for index generalization
and \newFi{(\forall E i)} for index instantiation.

The index generalization rule ($\forall I i$) is similar to
the type generalization rule ($\forall I$), but generalizes over
index variables ($i$) rather than type consturctor variables ($X$).
The rule ($\forall I i$) has two side conditions
while the rule ($\forall I$) has only one side conditions.
The additional side condition $i\notin\FV(t)$ in the ($\forall I i$) rule
prevents terms from accessing the type level index variables introduced by
index polymorphism. Without this side condition, $\forall$-binder
would no longer behave polymorphicaly, but instead would behave as
a dependent function, which are usually denoted by the $\Pi$-binder in
dependent type theories. The rule ($\forall I$) for ordinary
type generalization does not need such additional side condition
because type variables cannot appear in the syntax of terms.
The side conditions on generalization rules for polymorphism is fairly standard
in dependently typed languages supporting distinctions between polymorphism
(or, erasable arguments) and dependent functions (\eg, IPTS\cite{LingerS08},
ICC\cite{Miquel01}).

The index instantiation rule ($\forall E i$) is similar to
the type instantiation rule ($\forall E i$), except that
we type check the index term $s$ to be instantiated for $i$
in the current type level context paired with the empty term level context
($\Delta;\cdot$) rather than the current term level context.
Since index terms are at type level, they should not depend on
term level bindings.

In addition to the rules ($\forall I i$) and ($\forall E i$) for
index polymorphism, we need an additional variable rule \newFi{(:i)}
to be able to access the index variables already in scope. Terms ($s$) used
at type level in index applications ($F\{s\}$) should be able to access
index variables already in scope. For example, $\lambda i^A.F\{i\}$ should be
well-kinded under a context where $F$ is well-kinded,
justified by the derivation in Figure \ref{fig:ivarexample}.

\begin{figure*}
\[ \inference[($\lambda i$)]
      { \cdot |- A:* &
	\inference[($@i$)]{ \Delta, i^A |- F : A -> \kappa
                          & \inference[($:i$)]{ i^A\in \Delta,i^A
                                              & \Delta |- \cdot }
                                              {\Delta,i^A;\cdot |- i:A}
                          }
                          {\Delta, i^A |- F\{i\} : \kappa} }
      {\Delta |- \lambda i^A.F\{i\} :A -> \kappa}
\]
\caption{Kinding derivation for an index abstraction}
\label{fig:ivarexample}
\end{figure*}

