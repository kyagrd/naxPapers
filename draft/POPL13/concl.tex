\section{Conclusion and Future work}
\label{sec:concl}

System \Fi\ is a strongly-normalizing, logically-consistent
polymorphic lambda calculus that was designed to support the
definition of datatypes indexed by both terms and types.
In terms of expressivity, \Fi\ sits between System \Fw\ and CIC.
We designed System \Fi\ as a tool to reason about programming
languages with term-indexed datatypes. 

We have applied this tool to the programming language Nax (described elsewhere).
Nax is given semantics in terms of System \Fi . In Nax,
Mendler style operators are primitive operators with their
own typing rules. Nax has been designed to be expressive over
the Hindley-Milner fragment of System \Fi, and supports type
inference with minimal typing annotations. Typing annotations are
necessary only on case statements (for non-recursive term-indexed datatypes)
and Mendler style operators (for recursive term-indexed datatypes). Programs
involving only type-indexing require no annotations. A typing annotation
takes the form of a large elimination -- a function of types and terms to types.
The current implementation requires static annotations, but we believe we can extend
Nax to abstract over annotations, and still retain the strict separation between
terms and types that allows an erasure semantics.
 