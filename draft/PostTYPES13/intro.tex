\section{TODO intro}\label{sec:intro}
\begin{enumerate}
	\item discuss the advantages of the Mendler style,
	\item report that we can define an evaluator for the simply-typed HOAS
		using Mendler-style iteration with syntactic inverses
		(\lstinline{msfcata}), and
	\item propose a new recursion scheme (work in progress) whose
		termination relies on the invariants specified by
		size measures on indices.
\end{enumerate}

TODO TODO TODO TODO TODO TODO TODO TODO TODO TODO TODO TODO TODO TODO TODO TODO
TODO TODO TODO TODO TODO TODO TODO TODO TODO TODO TODO TODO TODO TODO TODO TODO
TODO TODO TODO TODO TODO TODO TODO TODO TODO TODO TODO TODO TODO TODO TODO TODO
TODO TODO TODO TODO TODO TODO TODO TODO TODO TODO TODO TODO TODO TODO TODO TODO
TODO TODO TODO TODO TODO TODO TODO TODO TODO TODO TODO TODO TODO TODO TODO TODO

Advantages of the Mendler style include allowing arbitrary definitions of
recursive datatypes, while still ensuring well-behaved use by providing
a rich set of principled eliminators (\ie, recursion schemes).
Certain concepts, such as HOAS, are most succinctly defined as
mixed-variant datatypes, which are unfortunately, outlawed in many existing
reasoning systems (\eg, Coq, Agda). One is forced to devise clever encodings
(\eg, \cite{PHOAS}) to use concepts like HOAS within such systems.

We believe it is worthwhile to allow definining all recursive datatypes
available in functional \emph{programming} languages, including those outlawed
in many reasoning systems. For example, the untyped $\lambda$-calculus can be
defined in HOAS as an Haskell dataype (\lstinline{Exp}):
\begin{lstlisting}
data Exp = Lam (Exp -> Exp) | App Exp Exp
\end{lstlisting}
Even if we assume all functions embedded in \lstinline{Lam} are non-recursive,
evaluating HOAS may still cause problems for logical reasoning, since
the untyped $\lambda$-calculus has diverging terms. However, there are many
well-behaved (\ie, terminating) computations over \lstinline{Exp}, such as
converting an HOAS expression to first-order syntax.
Ahn and Sheard \cite{AhnShe11} formalized a Mendler-style recursion scheme,
\lstinline{msfcata}  (\aka\ \textit{msfcata}), which captures these
well-behaved computations.

If the datatype \lstinline{Exp} had indices to assert invariants of
well-typed expressions (\eg, \lstinline{Exp Bool}\, for boolean expressions
and \lstinline{Exp Int}\, for integer expressions), we could rely on these
invariants to write even more expressive termiating programs, such as
a type-preserving evaluator. In Section\;\ref{sec:HOASeval}, we report
our novel discovery that we can define a type-preserving evaluator for
a simply-typed HOAS using \lstinline{msfcata}.

TODO TODO TODO TODO TODO TODO TODO TODO TODO TODO TODO TODO TODO TODO TODO TODO
TODO TODO TODO TODO TODO TODO TODO TODO TODO TODO TODO TODO TODO TODO TODO TODO
TODO TODO TODO TODO TODO TODO TODO TODO TODO TODO TODO TODO TODO TODO TODO TODO
TODO TODO TODO TODO TODO TODO TODO TODO TODO TODO TODO TODO TODO TODO TODO TODO
TODO TODO TODO TODO TODO TODO TODO TODO TODO TODO TODO TODO TODO TODO TODO TODO
TODO TODO TODO TODO TODO TODO TODO TODO TODO TODO TODO TODO TODO TODO TODO TODO
TODO TODO TODO TODO TODO TODO TODO TODO TODO TODO TODO TODO TODO TODO TODO TODO
