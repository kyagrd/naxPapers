\section{Type-preserving evaluation of the simply-typed HOAS}
\label{sec:HOASeval}
We can write an evaluator for a simply-typed HOAS in a simple manner,
as illustrated in Listing\;\ref{lst:HOASeval}.
We first define the simply-typed HOAS as a recursive indexed datatype
\lstinline{Exp  :: * -> *}. We take the fixpoint using \lstinline{Rec1}
(the fixpoint with a syntactic inverse). This fixpoint is taken over
a non recursive base structure \lstinline{ExpF  :: (* -> *) -> (* -> *)}.
Note that \lstinline{ExpF} is an indexed type. So expressions will be indexed
by their type. Using \lstinline{Rec1} the fixpoint of any structure is also
parameterized by the type of the answer. The use of the \lstinline{msfcata1}
requires that \lstinline{Exp} should be parametric in this answer type
(by defining \lstinline{type Exp  t =  forall a. Exp' a}). 
%% just as we did
%% in the untyped HOAS formatting example in Listing\;\ref{lst:HOASshow}.

\begin{figure}
\lstinputlisting[
	caption={Simply-typed HOAS evaluation using \lstinline{msfcata1}
		\label{lst:HOASeval}},
        firstline=4]{HOASeval.hs}
\vspace*{-3ex}
\end{figure}

The definition of \lstinline{eval} specifies how to evaluate
an HOAS expression to a host-language value (\ie, Haskell) wrapped by
the identity type (\lstinline{Id}\,). In the description below, we ignore
the wrapping (\lstinline{MkId}\,) and unwrapping (\lstinline{unId}\,) of
\lstinline{Id} by completely dropping them from the description.
See the Listing\;\ref{lst:HOASeval} (where they are not omitted)
if you care about these details. We discuss the evaluation for each of
the constructors of \lstinline{Exp}:
\begin{itemize}
	\item Evaluating an HOAS abstraction (\lstinline{Lam f}\,) lifts
		an object-language function \lstinline{(f)} over \lstinline{Exp}
		into a host-language function over values:
		\lstinline{(\v -> ev (f(inv v)))}.
		In the body of this host-language lambda abstraction,
		the inverse of the (host-language) argument value \lstinline{v}
		is passed to the object-language function \lstinline{f}.
		The resulting HOAS expression \lstinline{(f(inv v))} is
		evaluated by the recursive caller (\lstinline{ev}) to
		obtain a host-language value.

	\item Evaluating an HOAS application \lstinline{(App f x)} lifts
		the function \lstinline{f} and argument \lstinline{x} to
		host-language values \lstinline{(ev f}) and \lstinline{(ev x)},
		and uses host-language application to compute
		the resulting value. Note that the host-language application
		\lstinline{((ev f) (ev x))} is type correct since
		\lstinline{ev f :: a -> b}\, and \lstinline{ev x :: a},
		thus the resulting value has type \lstinline{b}.
\end{itemize}
We know that \lstinline{eval} indeed terminates since \lstinline{Rec1} and
\lstinline{msfcata1} can be embedded into System~\Fw\ in manner similar to
the embedding of \lstinline{Rec0} and \lstinline{msfcata0} into System~\Fw.

Listing\;\ref{lst:HOASeval} highlights two advantages of the Mendler style over
conventional style in one example. This example shows that the Mendler-style
iteration with syntactic inverses is useful for both \textit{negative} and
\textit{indexed} datatypes. \lstinline{Exp} in Listing\;\ref{lst:HOASeval} has
both negative recursive occurrences and type indices.

The \lstinline{showExp} example in Listing\;\ref{lst:HOASshow},
which we discussed in the previous section, has appeared in the work
of Fegaras and Sheard \cite{FegShe96}, written in conventional style.
So, the \lstinline{showExp} example, only shows that the Mendler style is
as expressive as the conventional style (although it is
perhaps syntactically more pleasant than the conventional style).
However, it is not obvious how one could extend the conventional-style
counterpart over indexed datatypes.

In contrast, \lstinline{msfcata} is uniformly defined over indexed datatypes of
arbitrary kinds. Both \lstinline{msfcata1}, used in the \lstinline{eval},
and \lstinline{msfcata0}, used in the \lstinline{showExp}, have exactly
the same syntatctic definition, differing only in their type signatures,
as lillustrated in Listing\;\ref{lst:reccomb}.


