\documentclass{llncs}

\usepackage[numbers]{natbib}
%% \usepackage[hang]{caption}
\usepackage{amsmath}
\usepackage{semantic}
%% \usepackage{framed}
\usepackage{color}
\usepackage{comment}
\usepackage{mylhs2tex}

\newcommand{\KYA}[1]{\textcolor{blue}{#1 -- Ki Yung}}
\newcommand{\TS}[1]{\textcolor{red}{#1 -- Tim}}

\title{The Nax programming language {\small(work in progress)}}
\titlerunning{The Nax programming language}
\author{Ki Yung Ahn\inst{1} \and Tim Sheard\inst{1} \and
	Author2 SName2\inst{2} % \and Author2 SName2\inst{2}
	}
\institute{
	Portland State University, Portland, Oregon, USA \thanks{supported by NSF grant 0910500.}
	\\ \email{kya@cs.pdx.edu} \qquad \email{sheard@cs.pdx.edu}
	\and
	University of Cambridge, Cambridge, UK
	\\ \email{todo@cl.cam.ac.uk} % \and \email{todo@cl.cam.ac.uk}
	}

\newcommand{\ie}{{i.e.}}
\newcommand{\eg}{{e.g.}}
\newcommand{\aka}{{a.k.a.}$\,$}

\newcommand{\Fig}[1]{Fig.\,\ref{fig:#1}}

\definecolor{grey}{rgb}{0.8,0.8,0.8}
\newcommand{\newFi}[1]{\colorbox{grey}{\ensuremath{#1}}}

\newcommand{\Wi}{\ensuremath{W_i}}

\newcommand{\Fi}{\ensuremath{\mathsf{F}_i}}
\newcommand{\Fw}{\ensuremath{\mathsf{F}_\omega}}
\newcommand{\fix}{\mathsf{fix}}
\newcommand{\Fix}{\mathsf{Fix}}
\newcommand{\Fixw}{\ensuremath{\Fix_{\omega}}}
\newcommand{\Fixi}{\ensuremath{\Fix_{i}}}

\newcommand{\Nat}{\ensuremath{\mathsf{Nat}}}
\newcommand{\Bool}{\ensuremath{\mathsf{Bool}}}
\newcommand{\sfList}{\ensuremath{\mathsf{List}}}
\newcommand{\sfVec}{\ensuremath{\mathsf{Vec}}}
\newcommand{\SAT}{\ensuremath{\mathsf{SAT}}}

\newcommand{\oz}{\oldstylenums{0}}
\newcommand{\ka}{{\check\kappa}}

\newcommand{\calS}{\mathcal{S}}
\newcommand{\calA}{\mathcal{A}}
\newcommand{\calR}{\mathcal{R}}
\newcommand{\dom}{\mathop{\mathsf{dom}}}
\newcommand{\FV}{\mathop{\mathrm{FV}}}
\newcommand{\In}{\mathsf{In}}

\newcommand{\defeq}{\stackrel{\mathrm{def}}{=}}
\newcommand{\tyinf}{\mathrel{\triangleright}}


%%% semantic package commands
\reservestyle[\mathinner]{\command}{\mathsf}
\command{MIt[MIt\;]}
\reservestyle[\mathinner]{\cmd}{\mathbf}
\cmd{let[let\;],in[\;in\;],data[data~],where[~where~]}

\begin{document}
\maketitle
%% \begin{abstract}
%% 	Write abstract here. Write abstract here.
%% 	Write abstract here. Write abstract here.
%% 	Write abstract here. Write abstract here.
%% 	Write abstract here. Write abstract here.
%% 	Write abstract here. Write abstract here.
%% 	Write abstract here. Write abstract here.
%% 	Write abstract here. Write abstract here.
%% 	Write abstract here. Write abstract here.
%% \keywords{asdf, asdf, asdf, asdf}
%% \end{abstract}

\chapter{Introduction}\label{ch:intro}

In this dissertation, we contribute to answering the question:
``how does one build a seamless system where programmers can both
write (functional) programs and formally reason about those programs''.
We set the scope of our research by clarifying what we mean by
\emph{programming} and \emph{formal reasoning}, and how they are related
(\S\ref{sec:intro:scope}). We motivate our research by describing the gap
between the designs of functional programming languages and
formal reasoning systems (\S\ref{sec:intro:motiv}), and assert
our thesis (\S\ref{sec:intro:motiv}).

TODO

preliminary concepts (\S\ref{sec:intro:concepts})

contributions (\S\ref{sec:intro:contrib})

methodology and overview (\S\ref{sec:intro:overview})

TODO

\section{Programming and Formal Reasoning}\label{sec:intro:scope}

\subsection{Scope of our research}
In our research, \emph{programming} refers to writing programs in
modern functional programming languages (\eg, Haskell, ML) that support
(recursive) datatypes, higher-order functions, (parametric) polymorphism,
and type inference. When we say \emph{functional programming languages},
\emph{functional languages}, or \emph{programming languages},
we mean such modern functional programming languages. We do not consider,
so called, dynamically typed functional languages (\eg, Lisp, Erlang)
in this dissertation.

\emph{Formal reasoning}, or \emph{logical reasoning},
refers to constructing proofs with dependently-typed formal reasoning systems
(\eg, Coq, Agda) that support (inductive) datatypes, higher-order functions,
polymorphism (\ie, type arguments to type constructors), and, of course,
dependent types (\ie, term arguments to type constructors). When we say
\emph{formal reasoning systems}, \emph{logical reasoning systems},
or, simply \emph{reasoning systems}, we mean such dependently-typed
formal reasoning systems.

Functional languages and reasoning systems are closely related.
Proofs in reasoning systems are similar in structure to programs
in functional languages. For example, the Haskell program \textit{id},
which computes a value of type $A$ when given a value of type $A$,
and, the Agda proof \textit{id}, which proves that the proposition $A$
is true when given a proof of $A$, are very similar:
\begin{center}
\begin{singlespace}
\begin{minipage}{.4\linewidth}
	\textsc{Haskell} \vspace*{.5em} \\
\textit{id} $::$ $A -> A$ \\
\textit{id} = $\lambda x -> x$
\end{minipage}
\begin{minipage}{.4\linewidth}
	\textsc{Agda}  \vspace*{.5em} \\
\textit{id} : $A -> A$ \\
\textit{id} = $\lambda x -> x$
\end{minipage}
\end{singlespace}
\end{center}\vspace*{.5em}
Such similarities are not accidental but intended by the design of
reasoning systems, based on the observation that proofs corresponds
to programs and propositions corresponds to types. This observation
is known as the Curry--Howard correspondence \cite{Howard69}.
The \emph{reasoning systems} we consider in this dissertation
are the ones that are based on the Curry--Howard correspondence.
We will explain the basic concepts to understand this correspondence
in the following subsection.

We assume that readers are familiar with \emph{programming}
in \emph{functional languages} and some basic concepts of lambda calculus
(\eg, $\beta$-reduction, normal forms). We assume that the readers have
basic knowledge on logic (\eg, concepts of axioms, inference, and proofs)
but not necessary familiar with \emph{formal reasoning}
in \emph{reasoning systems}.

\subsection{The Curry--Howard correspondence}
In late 1960s, \citet{Howard69} observed that intuitionistic natural deduction,
which is a proof system of a formal logic, and a typed lambda calculus,
which is a model of computation, are directly related. This relation,
known as the Curry--Howard correspondence, is established as follows:
\begin{itemize}
\item A proposition in the natural deduction corresponds to
	a type in the lambda calculus.
\item A proof for the proposition corresponds to
	a term of the type, to which that proposition corresponds.
\item Simplification of proofs correspond to computation,
	that is, simplification of terms.
\end{itemize}
We explain this correspondence with a very simple
version of intuitionistic natural deduction and
a simply-typed lambda calculus (Figure\;\ref{fig:nd}).\footnote{
	The readers who are familiar to the literature on natural deduction
	would notice that the left column in Figure\;\ref{fig:nd} is not
	the same style as the natural deduction formalized by
	\citet{Gentzen35,Gentzen69}. It has a has the flavours of
	both natural deduction and the sequent calculus. The context $\Gamma$
	are part of the syntax of judgments as in sequent calculus,
	but rather than the very syntactic structural rules (weakening,
	contraction, permutation) of sequent calculus, we simply rely on
	$A^x \in \Gamma$ to use the hypothetical propositions from $\Gamma$.
	We choose to formalize natural deduction this way to emphasize
	the structural similarities to the typical formalization
	for typing rules of lambda calculi.}

Traditionally, in Hilbert-style deduction, a formal logic is formalized
by a minimal set of inference rules (\eg, modus ponens
$\inference{\Gamma |- A -> B & \Gamma |- A}{\Gamma |- B}$)
and a set of axiom schemes (\eg, $A -> A$ and $A -> (B -> A)$ where
the meta-variables $A$ and $B$ can be instantiated to arbitrary propositions).
\emph{Natural deduction}, on the other hand, is another style of
formalizing logic mostly by a set of inference rules and a minimal
(often empty) set of axiom schemes.

There are many variations of formal logic based on natural deduction
(and also its corresponding typed lambda calculus) depending on
the set of logical connectives, constants, and quantifiers they support.
We illustrate a intuitionistic natural deduction with implication ($->$) and
falsity ($\bot$), and, its corresponding simply-typed lambda calculus (STLC)
in Figure\;\ref{fig:nd}. The implication and the falsity in natural deduction
corresponds to the function type ($->$) and the void type ($\bot$) in the STLC.
In the inference rules (\rulename{Ax}) and (\rulename{$->_\textsc{I}$}),
$x$ on $A^x$ is a meta-tag to distinguish between possibly multiple
occurrences of $A$ in $\Gamma$ (\eg, $\Gamma = A^{x_1},B^{x_2},A^{x_3}$).

\begin{figure}
\[
\begin{array}{ccc}
	\text{Intuitionistic natural deduction} & &
	\text{Typing rules of STLC} \\ \\
\inference{A^x \in \Gamma}{\Gamma |- A} & (\textsc{Ax}) &
\inference{x:A \in \Gamma}{\Gamma |- A} \\ \\
\inference{\Gamma,A^x |- B}{\Gamma |- A -> B} & (->_\textsc{I}) &
\inference{\Gamma,x:A |- B}{\Gamma |- \l x.t : A -> B} \\ \\
\inference{\Gamma |- A -> B & \Gamma |- A}{\Gamma |- B} & (->_\textsc{E}) &
\inference{\Gamma |- t : A -> B & \Gamma |- s : A}{\Gamma |- t\;s : B} \\ \\
\inference{\Gamma |- \bot}{\Gamma |- A} & (\bot_\textsc{E})  &
\inference{\Gamma |- t:\bot}{\Gamma |- \texttt{elim}_\bot\,t : A}
\end{array}
\]
\caption{Intuitionistic natural deduction with implication and falsity,
	and its corresponding simply-typed lambda calculus.}
\label{fig:nd}
\end{figure}

Note that each typing rule (in the right column) has exactly the same structure
as its corresponding inference rule (in the left column), except for
the variables and terms appearing on the right-hand side of the colon
(\eg, $x$ in $x:A$ and $t$ in $t:A -> B$). So, a type correct term,
which is justified by the derivation following the typing rules of
the lambda calculus, captures the structure of its corresponding proof
by natural deduction. For instance, $(\l x. x)$ is a term that captures
the structure of a proof for $A -> A$.
For this reason, such type correct terms in reasoning systems
are called proof terms, proof objects, or simply proofs.

We explained that proofs and propositions in the natural deduction
corresponds to terms and types in the lambda calculus. Lastly,
we need to show the correspondence between simplification of proofs
and simplification of terms (\ie, computation) in order to establish
the Curry--Howard correspondence between the natural deduction
and the lambda calculus.

There can be multiple proofs for the same proposition, and,
correspondingly, there can be more than one term of the same type.
Some of these terms are closely related while others are rather independent.
For example, $\l x_1.\l x_2.x_1$ and $\l x_1.\l x_2.x_2$ are rather 
independent terms that inhabit the same type $A -> A -> A$.
The typing derivations for these terms (right column) and
its corresponding proofs (left column) are illustrated in
Figure\;\ref{fig:proofAAA}.
\begin{figure}
\[\setpremisesend{.1em}
\inference[$(->_\textsc{I})^{x_1}$]
	{\hspace*{-3em}
	 \inference[$(->_\textsc{I})^{x_2}$]
	 	{\hspace*{-3em}
		 \inference[$(\textsc{Ax})^{x_1}$]
			{A^{x_1}\in A^{x_1},A^{x_2}}
			{A^{x_1},A^{x_2} |- A} }
		{A^{x_1} |- A -> A} }
	{|- A -> A -> A }
\qquad
\inference[($->_\textsc{I}$)]
	{\hspace*{-1.3em}
	 \inference[($->_\textsc{I}$)]
		{\hspace*{-1.5em}
		 \inference[(\textsc{Ax})]
			{x_1 : A \in x_1 : A, x_2 : A}
			{x_1 : A, x_2 : A |- x_1 : A} }
		{x_1 : A |- \l x_2.x_1 : A -> A} }
	{|- \l x_1.\l x_2.x_1 : A -> A -> A }
\]
\[\setpremisesend{.1em}
\inference[$(->_\textsc{I})^{x_1}$]
	{\hspace*{-3em}
	 \inference[$(->_\textsc{I})^{x_2}$]
		{\hspace*{-3em}
		 \inference[$(\textsc{Ax})^{x_2}$]
			{A^{x_2}\in A^{x_1},A^{x_2}}
			{A^{x_1},A^{x_2} |- A} }
		{A^{x_1} |- A -> A} }
	{|- A -> A -> A }
\qquad
\inference[($->_\textsc{I}$)]
	{\hspace*{-1.3em}
	 \inference[($->_\textsc{I}$)]
		{\hspace*{-1.5em}
		 \inference[(\textsc{Ax})]
			{x_2 : A \in x_1 : A, x_2 : A}
			{x_1 : A, x_2 : A |- x_2 : A} }
		{x_1 : A |- \l x_2.x_2 : A -> A} }
	{|- \l x_1.\l x_2.x_2 : A -> A -> A }
\]
\caption{Typing derivations (right) for terms of type $A -> A -> A$
	and their corresponding proofs (left).}
\label{fig:proofAAA}
\end{figure}

On the other hand, there are closely related terms of the same type.
For example, $(\l x_1.x_2) x_3$ reduces to $x_2$ by a $\beta$-reduction step
(\ie, $(\l x_1.x_2) x_3 -->_\beta x_2$). A term like $x_2$, which does not
($\beta$-)reduce any further, is called a ($\beta$-)\emph{normal form} or
a ($\beta$-)\emph{normal term}. \citet{Pra65} established a notion of
reduction and normalization for natural deduction. One can reduce a proof
when there is a consecutive use of introduction and elimination rules.
An introduction rule introduces a certain form of propositions
in the conclusion (below the horizontal bar), and, an elimination rule
uses propositions that form in the premises (above the horizontal bar).
The rules (\rulename{$->_\textsc{I}$}) and (\rulename{$->_\textsc{E}$})
are introduction and elimination rules for implication ($->$).
So, we need to show the correspondence between reduction of proofs
over implication in the natural deduction and ($\beta$-reduction) over
type correct terms in the lambda calculus,
in order to establish the Curry--Howard correspondence.

Figure\;\ref{fig:proofreduce} illustrates that
the reduction of a proof involving consecutive uses of
(\rulename{$->_\textsc{I}$}) and (\rulename{$->_\textsc{E}$})
corresponds to the $\beta$-reduction of well-typed terms.\footnote{
	Introduction and elimination rules for ($->$) consecutively used
	in the other order corresponds to $\eta$-reduction
	(\ie, $(\l x. t\;x) -->_\eta t$ where $x$ does not appear free in $t$).
	In this dissertation, we only consider $\beta$-reduction.}
To redcuce the proof, we replace the uses of $A^x$ in the premise of
$(->_\textsc{I})^x$ with the derivations $D'$, which deduces $A$ from
the original context $\Gamma$, so that we can remove $A^x$ from
the left-hand sides of the turstile ($|-$) throughout the proof.
The change of subscripts from $\mathcal{D}_{[\Gamma,A^x]}$ to
$\mathcal{D}_{[\Gamma]}$, before and after the reduction,
denotes that we consistently removed $A^x$ from the context
(\ie, left-hand sides of $|-$). We leave it as an exercise for
the readers to construct a corresponding proof for the reduction
$(\l x_1.x_2) x_3 -->_\beta x_2$ from the previous paragraph
(\textit{hints}: Start with $\Gamma = x_2:A,x_3:B$).

\begin{figure}
Reducing a proof involving consecutive uses of
(\rulename{$->_\textsc{I}$}) and (\rulename{$->_\textsc{E}$}):
\[
\inference[($->_\textsc{E}$)]
	{\hspace*{-3em}
	 \inference[$(->_\textsc{I})^{x}$]
		{\hspace*{-1em}
		 \inference*
			{\hspace*{-4em}
			 \inference[$(\textsc{Ax})^{x}$]
				{A^x\in\Gamma,A^x,\Gamma'}
				{\Gamma,A^x,\Gamma' |- A} \\
			 \vdots \\
			 \mathcal{D}_{[\Gamma,A^x]} }
			{\Gamma,A^x |- B} }
		{\Gamma |- A -> B}
	&\inference*{\mathcal{D}'_{[\Gamma]}}{\Gamma |- A} }
	{\Gamma |- B}
\quad
\longrightarrow
~
\inference*
	{\inference*
		{\mathcal{D}'_{[\Gamma,\Gamma']}}
		{\Gamma,\Gamma' |- A} \\
	 \vdots_{\phantom{G}} \\
	 \mathcal{D}_{[\Gamma]} }
	{\Gamma |- B}
\]
Reduction of a well-typed term along with its typing derivation:
\[
\inference[($->_\textsc{E}$)]
	{\hspace*{-3em}
	 \inference[$(->_\textsc{I})$]
		{\hspace*{-1em}
		 \inference*
			{\hspace*{-4em}
			 \inference[$(\textsc{Ax})$]
				{x:A\in\Gamma,x:A,\Gamma'}
				{\Gamma,x:A,\Gamma' |- x:A} \\
			 \vdots \\
			 \mathcal{D}_{[\Gamma,x:A]} }
			{\Gamma,x:A |- t:B} }
		{\Gamma |- \l x.t : A -> B}
	&\inference*{\mathcal{D}'_{[\Gamma]}}{\Gamma |- s:A} }
	{\Gamma |- (\l x.t)\,s:B}
\quad
\longrightarrow_\beta
~
\inference*
	{\inference*
		{\mathcal{D}'_{[\Gamma,\Gamma']}}
		{\Gamma,\Gamma' |- s:A} \\
	 \vdots_{\phantom{G}} \\
	 \mathcal{D}_{[\Gamma]} }
	{\Gamma |- t[s/x]:B}
\]
\caption{Reduction of a proof involving introduction and elimination rules
	for implication, and its corresponding $\beta$-reduction 
	of a well-typed term.}
\label{fig:proofreduce}
\end{figure}

We illustrated the Curry--Howard correspondence between
a minimal intuitionistic logic and its corresponding lambda calculus
In Figure\;\ref{fig:nd}. That is, a proposition, its proof, and
simplification of proofs (or, proof reduction) corresponds to
a term, its type, and simplification of terms (or, computation).
The first and second pieces of the corresponds, a proposition and
its proof corresponding to a term and its type, is quite obvious from
the structural similarity between the inference rules and the typing rules.
The last piece, correspondence between proof reduction and computation,
needs further analysis on the implication and elimination rules.
We illustrated the correspondence between proof reduction over implication
and $\beta$-reduction, which describes the computation over functions,
in Figure \ref{fig:proofreduce}. The correspondence between
the proof reduction over falsity and the computation over the void type
hold vacuously. There is no proof reduction and computation over $\bot$
since it lacks the introduction rule -- it only has the elimination rule
(\rulename{$\bot_\textsc{E}$}).

\section{Datatypes and recursion}
Programming languages support language constructs other than functions,
such as datatypes and polymorphism.

sums, products, aaaa

functional programming languages,
such as ML and Haskell,

are
logical connectives and 


TODO
$\forall$
$\exists$

What datatypes corresponds to
datatypes
In Figure\;\ref{fig:nd} has falsity ($\bot$), which corresponds to
a degenerative datatype (the void type), but




\section{Logical consistency and normalization}



if we know that every term is normalizing


datatypes and recursion



\section{Motivation}\label{sec:intro:motiv}
Since the observation of the Curry--Howard correspondence, logicians and
programming language researchers have dreamed of building a system in which
one can both write programs (\ie, model computation) and formally reason about
(\ie, construct proofs of) the properties (\ie, types) of those programs.

However, building a practical system that unifies programming and
formal reasoning, based on the Curry--Howard correspondence, is still
an open research problem. The gap between the conflicting
design goals of typed functional programming languages, such as ML and Haskell,
and formal reasoning systems, such as Coq and Agda, is still wide.

\begin{itemize}

\item
Programming languages are typically designed to achieve
computational expressiveness. They often sacrifice logical consistency
to achieve this goal. Programmers should be able to
conveniently express all possible computations, regardless of whether those
computations have a logical interpretation (by the Curry--Howard correspondence)
or not.

\item
Formal reasoning systems are typically designed to achieve logical consistency.
They often sacrifice computational expressiveness to achieve that goal.
Users expect that it is only possible to prove true propositions,
and it is impossible to prove falsity. They are willing to live with
the difficultly (or even inability) to express certain computations
within the reasoning system in order to achieve logical consistency.

\end{itemize}

As a result, the recursion schemes of programming languages and
formal reasoning systems differ considerably.
Programming languages provide unrestricted general recursion
to conveniently express computations that may or may not terminate.
Formal reasoning systems provide induction principles for sound reasoning,
or, in the computational view, principled recursion schemes
that can only express terminating computation.

The two different design goals also lead to significant differences
in their type system design as well.
Programming languages are based on \emph{recursive types},
which place only syntactic restrictions on the definition of new datatypes.
Programmers can express computations over a wide variety of datatypes.
In addition, most (statically typed) functional programing languages have
clear distinction between terms and types (\ie, terms do not appear in types).
Reasoning systems are usually based on \emph{inductive types},
which place semantic restrictions, only accepting datatype definitions
that support conventional\footnote{We will introduce
	what ``conventional'' style is in \S\ref{sec:intro:concpets:recursive}
	as opposed to the Mendler style in \S\ref{sec:intro:concepts:mendler}.
	} induction principles.
In addition, most reasoning systems, based on the Curry--Howard correspondence,
allow types to depend on terms (\ie, terms can appear in types) to specify
fine grained properties.

This dissertation explores a sweet spot where one can benefit from
the advantages of both programming languages and formal reasoning systems.
That is, we design a unified language system, called Nax, which is
logically consistent while being able to conveniently express
many useful computations. We do this by placing few restriction on type definitions,
as is done in programming languages, but also provide a rich set of
non-conventional recursion schemes (or, induction principles) that
always terminate. These non-conventional recursion schemes are known as
\emph{the Mendler style}. Another major design choice in Nax is
supporting \emph{term indices} in types, a middle ground, which sits between
polymorphic types and dependent types.

In the following section, we explain what we mean by the sweet spot between
programming languages and reasoning systems. Our thesis is that
the design choices we explain below are reasonable for achieving
the goal of combining programming languages and reasoning systems.

\section{Thesis}\label{sec:intro:thesis}
Whatever design choices we make, the sweet spot should have the following features.

\begin{enumerate}[(1)]
 \item \textbf{A convenient programming} style
         supported by the major constructs of
         modern functional programming languages: 
         parametric polymorphism, recursive datatypes,
         recursive functions, and type inference,
 \item \textbf{An expressive logic}
         that can specify fine-grained program properties using types, and terms that
         witness proofs of these properties 
         (the Curry--Howard correspondence),
 \item \textbf{A small theory} based upon a minimal foundational calculus that is
         expressive enough to support the programming features, expressive
         enough to embed propositions and proofs about
         programs, and logically consistent
         to avoid paradoxical proofs in the logic, and
 \item \textbf{A simple implementation} that keeps the trusted base small.
\end{enumerate}
We claim that a language design based on \emph{Mendler-style recursion schemes}
and \emph{term-indexed types} can lead to a system that supports these four
features.

\paragraph{}
From a bird's-eye view, the following chapters back up our claim as follows:
Mendler-style recursion schemes support (1) because they are based on
parametric polymorphism and well-defined over a wide range of datatypes.
Term-indexed types support (2), because they can statically track program
properties. For instance the size of data structures can be tracked by using
a natural number term in their types.
To support (3), we design several foundational calculi, each which extends
a well known polymorphic lambda calculus with term-indexed types.
Mendler-style recursion schemes also also support (4) because their
termination is type-based -- no need for an auxiliary termination checker.

In next section, we summarize important ideas mentioned in our thesis above.

\section{Preliminary concepts}\label{sec:intro:concepts}
We give summaries of the following preliminary concepts:
Curry--Howard correspondence (\S\ref{sec:intro:concepts:CH}),
Mendler-style recursion schemes
(\S\ref{sec:intro:concepts:CH}, \S\ref{sec:intro:concepts:mendler}),
and term-indexed types (\S\ref{sec:intro:concepts:indexed}).
Further details and historical backgrounds on each of these concepts
will appear in the following chapters (see \S\ref{sec:intro:overview}
for the overview of chapter organization).

\subsection{The Curry--Howard correspondence and Normalization}
\label{sec:intro:concepts:CH}
One promising approach to designing a system that unifies
logical reasoning and programming is \emph{the Curry--Howard correspondence}.
Howard \cite{Howard69} observed that a typed model of computation
(\ie, a typed lambda calculus) gives an interpretation to a (natural deduction)
proof system (for an intuitionistic logic). More specifically, one can interpret
a type (in the lambda calculus) as a formula (in the logic) and
a term of that type, as a proof for that formula. For instance,
the typing rule for function applications (APP) in a typed lambda calculus
corresponds to Modus Ponens (MP) in a logic:
\[ \inference[(APP)]{\Gamma |- t_1 : A -> B & \Gamma |- t_2 : A}{
        \Gamma |- t_1~t_2 : B}
 ~~~~~~~~
   \inference[(MP)]{A -> B & A}{B}
\]
As you can see above, combining terms ($t_1$ and $t_2$) to build a new term
($t_1~t_2$) can be interpreted as combining proofs for formulae
($A -> B$ and $A$), to construct a proof for a new formula ($B$).
More generally, we may expect that programming (\ie, building larger terms)
corresponds to constructing larger proofs, but only when the typed lambda calculi
meets certain standards -- \emph{type soundness} and \emph{normalization}.

The Curry--Howard correspondence is a promising approach to designing a
unified system for both logical reasoning and programming. Only one language
system is needed for both the logic and the programming language. An
alternate approach is to use an external logical language to talk about
programs as the objects that the logic reasons about. In this approach, one
has the obligation to argue that the soundness of the logic, with respect to
the programming language semantics, holds.

Under the Curry--Howard correspondence, the logic is internally related to the
semantics of program -- there is no need to argue for the soundness of the
logic,  externally outside of the programming language system. The soundness
of the logic follows directly from the type soundness of the language under
the Curry--Howard correspondence.

Let us consider a proposition to be true
(or, valid) when it has a canonical (\ie, cut-free) proof.
That is, there is a program, whose type is the proposition under
consideration, and that program has a normal form. 
By type soundness, any term,
of that type, will preserve its type during the reduction steps. Thus
reduction preserves truthfulness. If we assume
that the language is normalizing (\ie, every well-typed term reduces to
a normal form), then any term of that type which is a non-canonical proof,
implies the existence of a canonical proof, which in turn implies that
the proposition specified by the type is indeed true. That is, all provable
propositions are valid (\ie, the logic is sound) when the language is
\emph{type sound} and \emph{normalizing}.

\emph{Normalization} is also essential for the consistency of the logic.
For the lambda calculus to be interpreted as a \emph{consistent} logic,
there must be no diverging terms. A diverging term (\ie, a term that does
not have a normal form) may inhabit any arbitrary type. Thus, a diverging term
can be a proof for any proposition under the Curry--Howard correspondence.
General purpose functional programming languages (\eg, Haskell and ML), that
support unrestricted general recursion, cannot be interpreted as a consistent
logic, since they allow diverging terms (\ie, non-terminating programs).
For example, a diverging Haskell definition $\textit{loop} = \textit{loop}$
can be given an arbitrary type such as
$\textit{loop}\mathrel{::}\textit{Bool}$,
$\textit{loop}\mathrel{::}\textit{Int} -> \textit{Bool}$,
and even $\textit{loop}\mathrel{::}\forall a. a$, which is a proof of false.


Therefore, useful logical reasoning systems based on the Curry--Howard
correspondence (\eg, Coq, Agda) never support language features that can
lead to diverging terms. For example, in both Coq and Agda,
unrestricted general recursion (at term level) is not supported. 
Instead, these logical reasoning systems
often provide principled recursion schemes over recursive types that are
guaranteed to normalize. 

Recursive types (\ie, recursion at type level)
can also lead to diverging terms when they are not restricted carefully.
Many of the conventional logical reasoning systems, based on
Curry--Howard correspondence, restrict recursive types in a way,
which is not an ideal design choice, if one's goal is a unified system for
logic and programming. Our approach explores another design space not yet
completely explored. We introduce both approaches to restricting recursive
types to ensure normalization in the following two subsections.


\subsection{Restriction on recursive types for normalization}
\label{sec:intro:concpets:recursive}
We have argued that normalization is essential for logical reasoning systems
based on the Curry--Howard correspondence. One challenge to the successful
design of reasoning systems is how to restrict recursion at the type level
so that all well-typed terms have normal forms. There are two different
design choices illustrated in Figure~\ref{fig:approaches}. 
The conventional approach restricts the formation of recursive types
(\ie, the restriction is in datatype definition), and
the Mendler-style approach restricts the elimination of the values of
recursive types (\ie, the restriction is in pattern matching).

\begin{figure}
{\centering
\begin{tabular}{p{3cm}|p{12.5cm}}
\parbox{3cm}{~~Functional\\programming\\$~~~~$language} &
\parbox{12.5cm}{
 kinding:~
  \inference[($\mu$-form)]{\Gamma |- F : * -> *}{\Gamma |- \mu F : *} \\
 \\
 typing:\quad
  \inference[($\mu$-intro)]{\Gamma |- t : F (\mu F)}{\Gamma |- \In~t:\mu F} ~~~~
  \inference[($\mu$-elim)]{\Gamma |- t : \mu F}{\Gamma |- \unIn~t : F (\mu F)}\\
 \\
 reduction:
  \inference[(\unIn-\In)]{}{\unIn~(\In~t) \rightsquigarrow t}
} \\
\\ \hline\hline
\parbox{3cm}{$~$Conventional\\$~~~$approach for\\consistent logic} &
\parbox{12.5cm}{$\phantom{a}$\\
 kinding:~
  \inference[($\mu$-form$^{+}$)]{ \Gamma |- F : * -> * 
                           & \mathop{\mathsf{positive}}(F)}
                           {\Gamma |- \mu F : *} \\
 \\
 typing:~
  \text{{\small($\mu$-intro)} and {\small($\mu$-elim)}
                same as functional language} \\
  \[\inference[(\It)]{\Gamma |- t : \mu F & \Gamma |- \varphi : F A -> A}
                     {\Gamma |- \It~\varphi~t : A}\]
 reduction:~ \text{{\small(\unIn-\In)} same as functional language}
  \[\inference[(\It-\In)]{}{\It~\varphi~(\In~t) \rightsquigarrow
                            \varphi~(\textsf{map}_F~(\It~\varphi)~t)}\]
}
\\ \hline
\parbox{3cm}{Mendler-style\\$~~$approach for\\consistent logic} &
\parbox{12.5cm}{$\phantom{a}$\\
 kinding:~ \text{{\small($\mu$-form)} same as functional language} \\
 \\
 typing:~
  \text{{\small($\mu$-intro)} same as functional language}
  \[\inference[(\MIt)]
     { \Gamma |- t : \mu F &
       \Gamma |- \varphi : \forall X . (X -> A) -> F X -> A}
     {\Gamma |- \MIt~\varphi~t : A} \]
 reduction:~
  \inference[(\MIt-\In)]
     {}
     {\MIt~\varphi~(\In~t) \rightsquigarrow \varphi~(\MIt~\varphi)~t}
}
\end{tabular} }
\caption{Two different approaches to designing logical reasoning systems
         (in contrast to functional languages)}
\label{fig:approaches}
\end{figure}

\paragraph{Recursive types in functional programming languages.}
Let us start with a review of the theory of recursive types used
in functional programming languages. Here, the term
language is not expected to be normalizing, so restrictions are few.

Just as we can capture the essence of unrestricted general recursion at term
level, by a fixpoint operator (usually denoted by \textsf{Y} or \textbf{fix}),
we can capture the essence of recursive types by the
use of a recursive type operator, $\mu$, at type level. 
The rules for the formation {\small($\mu$-form)},
introduction {\small($\mu$-intro)}, and elimination {\small($\mu$-elim)} of
the recursive type operator $\mu$ are described in Figure \ref{fig:approaches}.
We also need a reduction rule {\small(\unIn-\In)}, which relates \In,
the data constructor for recursive types, and \unIn, the destructor for
recursive types, at the term level.

Surprisingly (if you hadn't known), the recursive type operator, $\mu$,
as described in Figure \ref{fig:approaches}, is already powerful enough to
express non-terminating programs, even without introducing the general recursive
{\em term} operator, \textbf{fix}, to the language. We illustrate this below.
First, here is a short reminder of how a fixpoint at the term level operates.
The typing rule and the reduction rule for \textbf{fix} can be given as follows:
\[ \text{typing:}~ \inference{\Gamma |- f : A -> A}{\textbf{fix}\,f : A}
 \qquad\qquad
   \text{reduction}:~ \textbf{fix}\,f \rightsquigarrow f(\textbf{fix}\,f)
\]
We can actually implement \textbf{fix}, using $\mu$, as follows
(using some Haskell-like syntax):
\begin{align*}
& \textbf{data}~T\;a\;r = C\;(r -> a) \quad
          \texttt{-}\texttt{-}~\text{\small a non-recursive datatype} \\
& w \,:\, \mu(T\;a) -> a ~~ \quad
          \texttt{-}\texttt{-}~\text{\small an encoding of the untyped
                                     $(\lambda x.x\;x)$
                                     in a typed language}~ \\
& w = \lambda x . \,\textbf{case}~\unIn~x~\textbf{of}~C\;f -> f\;x \\
& \textbf{fix} \,:\, (a -> a) -> a \quad
          \texttt{-}\texttt{-}~\text{\small an encoding of 
                                     $(\lambda f.(\lambda x.f(x\;x))\,
                                                 (\lambda x.f(x\;x)))$} \\
& \textbf{fix} = \lambda f. (\lambda x. f (w\;x))\,(\In(C(\lambda x. f (w\;x))))
\end{align*}

Thus, to avoid the loss of termination guarantees, we need to alter the rules
for $\mu$, in someways, to ensure a consistent logic. One way, is to restrict
the rule {\small $\mu$-form}; the other way, is to restrict the rule
{\small $\mu$-elim}. Once we decide which of these two alterations of the
rules we will use, the design of principled recursion combinators (\eg, \It\
for the former and \MIt\ for the latter) follows from that choice.

\paragraph{Positive (recursive) datatypes and negative (recursive) datatypes.}
Positive datatypes have recursion only in covariant positions.
For example, $\mu\,T_2$, where $\textbf{data}\;T_2 = C_2\,(Bool -> r)$,
is a positive datatype since the recursive argument $r$ in
the base structure $T_2$ only appears in the covariant position.
Recursive datatypes that have no function arguments are by default
positive datatypes. For instance, the natural number datatype $\mu\,N$,
where $\textbf{data}\,N\,r=S\,r \mid Z$, is a positive datatype.


Negative datatypes have recursion in contravariant positions.
Note that $\mu(T\;a)$ in the example above is a negative datatype
since the recursive argument $r$ in the base structure $T$ appears
in the contravariant position. Another example of a negative datatype is
$\mu\,T'$, where $\textbf{data}\;T'\;r = C'\,(r -> r)$, since $r$ in $T'$
appears in both contravariant and covariant positions.

\paragraph{Recursive types in the conventional approach to consistent logic.}
In the conventional approach, the formation (\ie, datatype definition) of
recursive types is restricted, but arbitrary elimination (\ie, pattern matching)
over the values of recursive types is allowed. In particular, the formation of
negative recursive types is restricted. Only positive recursive types are
supported. Thus, in Figure \ref{fig:approaches}, we have a restricted version of
the formation rule {\small($\mu$-form$^{+}$)} has an additional condition that
require $F$ to be positive. The other rules {\small($\mu$-intro)},
{\small($\mu$-elim)}, and {\small(\unIn-\In)} remain the same as for
functional languages. Since we have restricted the recursive types
at the type level and we do not have general recursion at the term level,
the language is indeed normalizing. However, we can neither write
interesting (\ie, recursive) programs that involves recursive types nor
inductively reason about those programs, unless we have principled recursion
schemes that are guaranteed to normalize. One such recursion scheme is called
iteration (\aka\ catamorphism). The typing rules for the conventional iteration
\It\ are illustrated in Figure \ref{fig:approaches}. Note, we have the typing
rule {\small(\It)} and the reduction rule {\small(\It-\In)} for \It\,
in addition to the rules for the recursive type operator $\mu$.

\paragraph{Recursive types in the Mendler-style approach to consistent logic.}
In the Mendler-style approach, we allow arbitrary formation
(\ie, datatype definition) of recursive types, but we restrict
the elimination (\ie, pattern matching) over the values of recursive types. 
The formation rule {\small($\mu$-form)} remains the same as
for functional languages. That is, we can define arbitrary recursive types,
both positive and negative. However, we no longer have the elimination
rule {\small($\mu$-elim)}. That is, we are not allowed to pattern match over
the values of recursive types in the normal fashion. We can only pattern match
over the values of recursive types through the Mendler-style recursion
combinators. The rules for the Mendler-style iteration combinator \MIt\
are illustrated in Figure \ref{fig:approaches}.
Note, there are no rules for \unIn\ in the Mendler-style approach.
The typing rule {\small($\mu$-elim)} is replaced by {\small(\MIt)} and
the reduction rule {\small(\unIn-\In)} is replaced by {\small(\MIt-\In)}.
More precisely, the typing rule {\small \MIt} is both an elimination rule
for recursive types and a typing rule for the Mendler-style iterator.
You can think of the rule {\small(\MIt)} as replacing both the elimination rule
{\small($\mu$-elim)} and the typing rule for conventional iteration
{\small(\It)}, but in a safe way that guarantees normalization.

\subsection{Justification of the Mendler-style as a design choice.}
\label{sec:intro:concepts:mendler}
We choose to base our approach to the design of a seamless synthesis of both
logic and programming on the Mendler-style. It restricts the elimination (\ie,
pattern matching) over values of recursive types, rather restricting the
formation (\ie, datatype definition) of recursive types (a more conventional
approach). The impact of this design choice is that it enables the logic to
include all datatype definitions that are used in functional programming
languages.

Functional programming promotes ``functions as first class values''.
It is natural to pass functions as arguments and embed functions into
(recursive) datatypes. If embedding functions in datatypes is allowed,
we can embed a function whose domain is the very type we are defining.
For example, the recursive datatype definition
$\mathbf{data}~T = C\;(T -> \textit{A})$ in Haskell is such a recursive
datatype definition. Such datatypes are called negative recursive datatypes
since the recursive occurrence $T$ appears in a negative position.
We say that $T$ is in a negative position, since $(T -> A)$ is analogous to
$(\neg T \land A)$ when we think of $->$ as a logical implication. There exist
both interesting and useful examples in functional programming involving
negative datatypes. In \S\ref{sec:msf}, we illustrate that
the Mendler-style recursion scheme we discovered can express
interesting examples involving negative datatypes.

Recall that the motivation of this dissertation research
(quoting again from \S\ref{sec:intro:motiv})
is to contribute to answering the question of ``how does one build a
seamless system where programmers can both write (functional) programs and
formally reason about those programs''. Under the Curry--Howard correspondence,
to formally reason about a program, the logic needs to refer to the type of
the program, since the type, interpreted as a proposition, describes
the property of the program. Since the Mendler-style approach does not
restrict recursive datatype definitions, we can directly refer to the types
of programs that use negative recursive datatypes.

The Mendler style is a promising approach to building a unified system because
all the recursive types (both positive and negative) are definable and
the recursion schemes over those types are normalizing.
Although the conventional approach is widely followed
in the design of formal reasoning systems (\eg, Coq, Agda), it cannot directly
refer to programs that use non-positive recursive types.One may object that
it is possible to indirectly model negative recursive types
in the conventional style, via alternative equivalent encodings
which map negative recursive types into positive ones. But, such
encodings do not align with our motivation towards a seamless unified
system for both programming and reasoning. It is undesirable to require
programmers to significantly change their programs just to reason about them.
If the change is unavoidable, it should be kept small. That is,
the changed program should syntactically resemble the original program,
which the programmer would usually write in a functional programming language.
In Chapter 3, we show a number of examples of programs written in
the Mendler style that look more close to the programs written using
general recursion than the programs written in the conventional style.

\subsection{Term-indexed types, type inference, and datatypes}
\label{sec:intro:concepts:indexed}
One of the most frequently asked questions about our design choices for Nax,
regarding term-indexed types, is ``why not dependent types?". Our answer
is that a moderate extension to the polymorphic calculus is a better candidate
than a dependently typed calculus as the basis for a practical programming
system. Recall, that we hope to design a unified system for programming
as well as reasoning. Language designs based on indexed types can
benefit from existing compiler technology and type inference algorithms
for functional programming languages. In addition, theories for
term-indexed datatypes are simpler than theories for full-fledged
dependent datatypes, because term-indexed datatypes can be encoded as
functions (using Church-like encodings).

The implementation technology for functional programming languages based on
polymorphic calculi is quite mature. There exist industrial
strength implementations, such as the Glasgow Haskell Compiler (GHC),
whose intermediate core language is an extension of \Fw.
Our term-indexed calculi described in Part \ref{part:Calculi} are closely
related to \Fw\ by an index-erasure property. The hope is that
our implementation can benefit from these technologies.

Type inference algorithms for functional programming languages are often
based on certain restrictions of the Curry-style polymorphic lambda calculi.
These restrictions are designed to avoid higher-order unification during
type inference.
We develop a conservative extension of Hindley--Milner type inference for
Nax (Chapter \ref{ch:naxTyInfer}). This is possible because Nax is based on our
term-indexed calculi (Part \ref{part:Calculi}). Dependently typed languages,
on the other hand, are often based on bidirectional type checking, which
requires annotations on top level definitions, rather than
Hindley--Milner-style type inference.

In dependent type theories, datatypes are usually supported as primitive
constructs with axioms, rather than as functional encodings
(\eg, Church encodings). One can give functional encodings for datatypes
in a dependent type theory, but one soon realizes that the induction principles
(or, dependent eliminators) for those datatypes cannot be derived within
the pure dependent calculi \cite{Geuvers01}.
So, dependently typed reasoning systems support datatypes as primitives.
For instance, Coq is based on Calculus of Inductive Constructions, which
extends Calculus of Constructions \cite{CoqHue86} with dependent datatypes
and their induction principles.

In contrast, in polymorphic type theories, all imaginable datatypes
within the calculi have functional encodings (\eg, Church encodings).
For instance, \Fw\ need not introduce datatypes as primitive constructs,
since \Fw\ can embed all imaginable datatypes, including non-regular
recursive datatypes with type indices. 

Another reason to use term-indexed calculi, rather than dependent type theories,
is to extend the application of Mendler-style recursion schemes,
which are well-understood in the context of \Fw.
Researchers have thought about (though not published)\footnote{
     Tarmo Uustalu described this on a whiteboard
     when we met with him at the University of Cambridge in 2011.
     We discuss this in the related work chapter (\S\ref{sec:relwork:dep}).}
Mendler-style primitive recursion over dependently-typed functions
over positive datatypes (\ie, datatypes that have a map), but not for
negative (or, mixed-variant) datatypes. In our term-indexed calculi,
we can embed Mendler-style recursion schemes (just as we embedded them in \Fw)
that are also well-defined for negative datatypes.

\section{Contributions}\label{sec:intro:contrib}
This dissertation makes contributions in several areas.
\begin{itemize}
\item[1.]
    It organizes and expands the realm of \emph{Mendler-style recursion schemes}
    (Part~\ref{part:Mendler})

\item[2.] It establishes a meta-theories for \emph{term-indexed types}
        (Part~\ref{part:Calculi}),

\item[3.] It designs a practical language (with an implementation)
        \emph{in the sweet spot} between programming and logical reasoning
        (Part~\ref{part:Nax}), and

\item[4.] It identifies several interesting open problems related to above.
\end{itemize}

\subsection{Contributions related to the Mendler style}
We organize a hierarchy of Mendler-style recursion schemes in two dimensions.
The first dimension is the abstract operations they support. For instance,
the Mendler-style iteration (\MIt) supports a single abstract operation
the recursive call. All the other Mendler-style recursion schemes
support the recursive call and an additional set of abstract operations. 
The second dimension is over the kind of the datatypes they operate over.
For example, \texttt{Nat} has kind $*$, while \texttt{Vec}
has kind $* -> \mathtt{Nat} -> *$. Each recursion scheme is actually a
family of recursion combinators sharing the same term definition
(\ie, uniformly defined) but with different type signatures at each kind.

We expand the realm of Mendler-style recursion schemes in several ways.
First, we report on a new recursion scheme $\MsfIt$, which is useful
for negative datatypes.  Second, we study the termination behaviors
of Mendler-style recursion schemes. Some recursion schemes (\eg, \MIt, \MsfIt)
always terminate for any recursive type, while others (\eg, \McvPr) only
terminate for certain classes of recursive types. Third, we extend
all Mendler-style recursion schemes to be expressive over term-indexed types.
The Mendler style has been studied in the context of \Fw\ (and several
extensions) which can express \emph{type}-indexed types. To extend Mendler-style
recursion schemes to be expressive over \emph{term}-indexed types, we report on
several theories for calculi (\Fi\ and \Fixi) that support term indices.
This is another important area of our contribution.

We provide examples that illustrate when each recursion scheme is useful
in Chapter \ref{ch:mendler}. The most interesting example among them is
the type-preserving evaluator for a simply-typed Higher-Order Abstract Syntax
(HOAS) (\S\ref{sec:evalHOAS}), which involves negative datatypes with indices.
This example is our novel discovery, which reports that
a type-preserving evaluator for a simply-typed HOAS can be expressed within \Fw.

In addition, we develop a better understanding of some existing
Mendler-style recursion schemes. For instance, the existence of
Mendler-style course-of-values recursion (\McvPr) is reported
in the literature, but the calculus that can embed \McvPr\ was unknown.
We embed Mendler-style course-of-values recursion into \Fixi\ 
(or into \Fixw\ \cite{AbeMat04}, when we do not consider term-indices).

\subsection{Contributions to the theory of Term-Indexed Types}
Mendler-style recursion schemes have been studies in the context of
polymorphic lambda calculi. For instance, \citet{AbeMatUus03} embedded 
Mendler-style iteration (\MIt) into \Fw\ and \citet{AbeMat04} embedded
Mendler-style primitive recursion (\MPr) into \Fixw. These calculi
support type-indexed types.

To extend the realm of Mendler-style recursion schemes to include
term-indexed types, we extended \Fw\ and \Fixw\ to support term indices.
In Part~\ref{part:Calculi}, we present our new calculi
\Fi\ (Chapter~\ref{ch:fi}), which extends \Fw\ with term indices, and
\Fixi\ (Chapter~\ref{ch:fixi}), which extends \Fixw\ with term indices.
These calculi have an erasure property that states that well-typed terms
in each calculus are also well typed terms (when erased) in the 
underlying calculus. For instance, any well typed term in \Fi\ is also
a well-typed term in \Fw, and there are no additional well-typed terms
in \Fi\ that are not well-typed in \Fw.

Our new calculi, \Fi\ and \Fixi, are strongly normalizing and
logically consistent. We show strong normalization and logical consistency
using the erasure properties. That is, strong normalization and
logical consistency of \Fi\ and \Fixi\ are inherited from \Fw\ and \Fixw.
Since \Fi\ and \Fixi\ are strong normalizing and logically  consistent,
the Mendler-style recursion schemes that can be embedded into these calculi
are adequate for logical reasoning as well as programming.

\subsection{Contributions in the design of the Nax language}
We design and implement a prototypical language Nax that explores
the sweet spot between programming oriented systems and logic oriented systems.
The language features supported by Nax provide the advantages
of both programming oriented systems and logic oriented systems.
Nax supports both term- and type-indexed datatypes,
rich families of Mendler-style recursion combinators,
and a conservative extension of Hindley--Milner type inference.
We designed Nax so that its foundational theory and
implementation framework could be kept simple.

Term- and type-indexed datatypes can express fine grained program properties
via the Curry--Howard correspondence, as in logic oriented systems. Although
not as flexible as full-fledged dependent types, indexed datatypes can
still express program invariants, such as stack-safe compilation
(\S\ref{sec:example}), and size invariants on data structures.
Index types can simulate much of what
dependent types can do using singleton types. Since Nax has only erasable
indices, the foundational theory can be kept simple, and it supports
features that have the advantages of programming oriented systems 
(\eg, type inference, arbitrary recursive datatypes).

Adopting Mendler style provides merits of both programming oriented systems
and logic oriented systems. Since Mendler style is elimination based, one can
define all recursive datatypes usually supported in functional programming
languages. In addition, the programs written using Mendler-style recursion
combinators look more similar to the programs written using general recursion
than programs written in Squiggol style.
Since Nax supports only the well-behaved (\ie, strongly normalizing)
Mendler-style recursion combinators, it is safe to construct proofs using them.
In addition, Mendler-style recursion combinators are naturally well-defined
over indexed datatypes, which are essential to express fine-grained program
properties. Mendler style provides type based termination, that is, termination
is a by-product of type checking. Thus, it makes the implementation framework
simple since we do not need extra termination checking theories or algorithm.

Hindley--Milner-style type inference is familiar 
to functional programmers.
Nax can infer types for all programs that involve only regular datatypes,
which are already inferable in Hindley--Milner, without any type annotation.
Nax requires programs involving indexed datatypes to annotate their eliminators
by index transformers, which specify the relation between the input type index
and the result type. Eliminators of non-recursive datatypes are case expressions
and eliminators of recursive datatypes are Mendler-style recursion combinators.

\subsection{Contributions identifying open problems}
We identify several open problems alongside the contributions mentioned
in previews subsections. We will discuss the details of these open problems
in the future work chapter (Chapter \ref{ch:futwork}).
Here, we briefly introduce two of them.

\paragraph{Handling different interpretations of $\mu$ in one language system:}
Nax provides multiple recursion schemes (or, induction principles) used
to describe different kinds of recursive computations over recursive datatypes.
These recursion schemes are all motivated by concrete examples, which explains
the need for multiple schemes. It is more convenient to express various kinds of
recursive computations in Nax, by choosing a recursion scheme that fits
the structure of the computation, than in those systems that provide
only one induction scheme. However, there is theoretical difficulty
handling multiple interpretations of the recursive type operator $\mu$
in one language system.

Recall that we can embed datatypes as functional encodings in
our indexed type theory. Recursive datatypes and their recursion schemes in Nax
are embedded using Mendler-style encodings.
In Mendler style, one encodes the recursive type operator $\mu$
and its eliminator (the recursion scheme) as a pair.
So, there are several different encodings of $\mu$,
one for each recursion scheme. Some recursion schemes subsume others
(\ie, the more expressive one can simulate the other).

It would have been easy to describe the theory for Nax if we had
one most powerful recursion scheme that subsumes all the others,
which leads to a single interpretation of $\mu$. Unfortunately, we know of
no Mendler-style recursion scheme that subsumes all the other recursion schemes
in Nax. For instance, iteration (\MIt) can be subsumed by either 
iteration with a syntactic inverse (\MsfIt) or primitive recursion (\MPr).
But, there is no known recursion scheme that can subsume both \MsfIt\ and \MPr.

However, we strongly believe that it is okay to apply \MsfIt\ to
the result of \MPr\ (when \MPr\ produces a recursive value) and vice versa.
Intuitively, the different interpretations of $\mu$ only matter during
the internal computation of the recursion scheme. That is, one may consider
that (recursive) values resulting from different recursion schemes
share a common abstract representation of $\mu$.
The theoretical justification for this is still ongoing work.

\paragraph{Deriving positivity (or monotonicity) from polarized kinds:}
One can extend the kind syntax of arrow kinds in \Fw\ with polarities
($p\kappa_1 -> \kappa_2$ where the polarity $p$ is either $+$, $-$, or $0$)
to track whether a type constructor argument is used in
covariant (positive), contravariant (negative), or
mixed-variant (both positive and negative) positions.
It is still an open problem whether it is possible to derive monotonicity
(\ie, the  existence of a map) for a type constructor from its polarized kind,
without examining the type constructor definition.

We identified a useful application for a solution to this open problem.
We discovered an embedding of Mendler-style course-of-values recursion in
a polarized system for positive (or monotone) type constructors.
That is, once you can show the existence of a map for a datatype,
course-of-values recursion always terminates.
However, in a practical language system, it is not desirable to burden users
with the manual derivation for every datatype on which they might want to
perform course-of-values recursion. If the type system can automatically
categorize datatypes that have maps from their polarized kinds,
this burden can be alleviated.

\newpage{}
\section{Methodology and Overview}\label{sec:intro:overview}
\begin{figure}
\centering
$\xymatrix{
 \textrm{STLC}
     \ar@{-->}[dd]_{\begin{smallmatrix}
                      \textrm{type}\\
                      \textrm{polymorphism}
                    \end{smallmatrix}} \\
      &&& \textrm{Hindley--Milner} \ar[dd] & \\
 \F  \ar[dd]_{\begin{smallmatrix}
                \textrm{type constructor} \\
                \textrm{polymorphism}
              \end{smallmatrix}}
     \ar@{.>}[rrru]|{\textrm{rank-1 restriction}}
                &&&& \\
                 &&& \textrm{Nax's type inference\hspace{-3em}} & \\
 \Fw   \ar@/^/@{=>}[rr]
       \ar@{~>}[dd]_{\begin{smallmatrix}
                       \textrm{polaraized kinds,}\\
                       \textrm{equi-recursive } \mu
                     \end{smallmatrix}}
                && \Fi \ar@{~>}[dd] \ar@/^/@{:>}[ll] \ar@{.>}[ru] &&
             \\ &&&& ~ \MIt \;~\phantom{\subset}~~~\; \MsfIt \hspace{-4em}
                            \ar@/^1pc/@{{ }{>}{>}}[llu]
                          & \ar@<-.5ex>@/_1pc/@{{ }{>}{>}}[lllu] \\
 \Fixw \ar@/^/@{=>}[rr]^{\textrm{term index}}
                && \Fixi \ar@/^/@{:>}[ll]^{\textrm{index erasure}}
                  && ~ \MPr \quad
                             \ar@{}[u]^{\bigcap}
			     \ar@/_/@{{ }{>}{>}}[ll]_{\textrm{embeded into}} \\
                  &&&& \McvPr
                             \ar@{}[u]^{\bigcap}
                      \ar@<.5em>@{}[u]_{\begin{smallmatrix}
                                         _\textrm{(termination behavior~}\\
                                         ^\textrm{~is not subsumed here)}
                                        \end{smallmatrix}}
			     \ar@/^/@{{ }{>}{>}}[llu] \\
}$

\caption{Summary on how key concepts are related}
\label{fig:overview}
\end{figure}

This dissertation consist of five parts:
\begin{itemize}
\item Part \ref{part:Prelude} (Prelude),
\item Part \ref{part:Mendler} (The Mendler style),
\item Part \ref{part:Calculi} (Term-indexed lambda calculi),
\item Part \ref{part:Nax} (The Nax language), and
\item Part \ref{part:Postlude} (Postlude).
\end{itemize}
The three parts in the middle describe the three steps of our approach. 
First, we explore new ideas about
Mendler-style recursion schemes driven from concrete examples
using Haskell (with some GHC extensions). Second, we develop
theories (\ie, lambda calculi) for term-indexed datatypes to prove that
the Mendler-style recursion schemes are well-defined over indexed datatypes
and have the expected termination behavior. Third, we design a language system
with practical features that implements our ideas and is based on the
theory we have developed. Figure~\ref{fig:overview} summarizes the organization of
key concepts throughout the dissertation.

\paragraph{Part \ref{part:Prelude} (Prelude)}\hspace{-1em} 
comprises Chapter \ref{ch:intro} (which you are currently reading)
and Chapter \ref{ch:poly} which 
reviews the theory of several well-known typed lambda calculi:
the simply-typed lambda calculus (STLC) (\S\ref{sec:stlc}),
System \F\ (\S\ref{sec:f}),
System \Fw\ (\S\ref{sec:fw}), and
the Hindley--Milner type system (\S\ref{sec:hm}).

In Sections \ref{sec:stlc}-\ref{sec:fw}, we review strong normalization proofs
(using saturated sets) for each of the three calculi:
STLC (no polymorphism), System \F\ (polymorphism over types), and
System \Fw\ (polymorphism over type constructors).

The later proofs each extend the normalization proof of the previous calculus.
We will use the strong normalization of System \Fw\ to show that
our term-indexed lambda calculi in Part \ref{part:Calculi} are
strongly normalizing. Readers familiar with strong normalization proofs
of these calculi may skip or quickly skim over these sections.
It is worth noticing two stylistic choices in our formalization of
System \F\ and \Fw: (1) terms are in Curry style and
(2) typing contexts are divided into two parts
    (one for type variables and the other for term variables).
This choice prepares readers for our formalization of the term-indexed calculi
in Part \ref{part:Calculi}, which have Curry-style terms and
typing contexts divided into two parts.

In \S\ref{sec:hm}, we review the type inference algorithm for
the Hindley--Milner type system (\S\ref{sec:hm}).
The Hindley--Milner type system (HM) is a restriction of System~\F,
which makes it possible to infer types without any type annotation on terms.
Later in Part~\ref{part:Nax} Chapter \ref{ch:naxTyInfer},
we formulate a conservative extension of HM, which restricts
the term-indexed calculus System \Fi\ (Chapter \ref{ch:fi}) in a similar manner.

\paragraph{Part \ref{part:Mendler} (the Mendler style)}\hspace{-1em} introduces
the concept of Mendler-style recursion schemes (Chapter \ref{ch:mendler})
using examples written in Haskell (with some GHC extensions). The readers
of Chapter \ref{ch:mendler} need no background knowledge on typed lambda calculi
but only some familiarity with functional programming. We explain the concepts of
a number of Mendler-style recursion schemes, their termination properties, and
how one recursion scheme is related to another. 
We also provide semi-formal proofs of termination
for some of the recursion schemes (\MIt\ and \MsfIt) by embedding them
into the \Fw\ fragment of Haskell. More formal and general
proofs, by embedding the schemes into our term-indexed lambda calculi, come later in
Part~\ref{part:Calculi}.

The Mendler-style recursion schemes discussed in Chapter \ref{ch:mendler}
include iteration (\MIt), iteration with syntactic inverse (\MsfIt),
primitive recursion (\MPr), course-of-values iteration (\McvIt),
and course-of-values recursion (\McvPr). Of these, \MsfIt\ was discovered
while writing this dissertation.
There are even more Mendler-style recursion schemes, which are not
discussed in Chapter \ref{ch:mendler} -- we give pointers to them in our
related work chapter (Chapter \ref{ch:relwork} of Part \ref{part:Postlude}).

\paragraph{Part \ref{part:Calculi} (term-indexed lambda calculi)}\hspace{-1em}
develops theories for term-indexed types.
We formalize two term-indexed lambda calculi,
System \Fi\ (Chapter \ref{ch:fi}) and System \Fixi\ (Chapter \ref{ch:fixi}),
which are extensions of polymorphic calculi.
System \Fi\ extends System \Fw\ with term indices and
System \Fixi\ extends System \Fixw\ \cite{AbeMat04} with term indices.

We prove both strong normalization and logical consistency of
these term-indexed calculi using an index erasure property.
The index erasure property of a term-indexed calculus
projects a typing in the term-index calculi into
the polymorphic calculus that the term-indexed calculus extends.
That is, all well-typed terms in \Fi\ and \Fixi\ are
also well-typed typed terms in \Fw\ and \Fixw.


By embedding Mendler-style  recursion schemes into our term-indexed lambda calculi,
we prove that those schemes are well-defined and
terminate over term-indexed datatypes.  For instance,
\MIt\ and \MsfIt\ can be embedded into System \Fi,
and, \MPr\ and \McvPr\ can be embedded into System \Fixi.

\paragraph{Part \ref{part:Nax} (the Nax language)}\hspace{-1em} consists of
three chapters.
First, we introduce the features of Nax (Chapter \ref{ch:naxFeatures})
in a tutorial format using small Nax code snippets as examples.
Next, we discuss the design principles of the type system (Chapter \ref{ch:nax})
by comparing it to two other systems: Haskell's datatype promotion and Agda.
In Chapter \ref{ch:nax} we develop
larger and more practical examples,
a type-preserving interpreter and a stack safe compiler.
Lastly, we discuss type inference in Nax (Chapter \ref{ch:naxTyInfer}),
which is a conservative extension of the Hindley--Milner type system (HM).
That is, any program whose type is inferable in HM, can also have its type
inferred in Nax without any annotation. Programs involving
term- or type-indexed datatypes, which are not supported in HM, need
some annotation to infer their types in Nax. These annotations are only
required on three syntactic entities (datatype declarations, case expressions,
and Mendler-style recursion combinators).

\paragraph{Part \ref{part:Postlude} (Postlude)}\hspace{-1em} closes
the dissertation by summarizing
  related work (Chapter~\ref{ch:relwork}),
  future work (Chapter~\ref{ch:futwork}), and
  conclusions (Chapter~\ref{ch:concl}).



% Proof assistants have shown its practical use in mechanizing certain
% mathematical proofs \cite{4 color problem}, verifying specifications
% \cite{something}, and mechanizing metatheories of programming languages.
% However, verified programming, or \emph{``correct by construction''} approach
% to programming, is yet far from being practical for everyday programming tasks,
% because proof assistants were designed for aiding formal proof generation,
% not for writing executable programs.


% \subsection{Proof assistants}
% \emph{Proof assistants} based on Curry-Howard correspondence over
% dependently-typed $\lambda$-calculi is a major application area of type theory.
% Such proof assistants can \emph{express fine-grained properties} using
% dependent types and also \emph{guarantee correctness} since they are based on
% strongly normalizing and logically consistent calculi. For instance, Coq is
% based on the Calculus of Inductive Constructions, which is a dependently-typed
% $\lambda$-calculi known to be strongly normalizing and logically consistent.
% 
% Proof assistants have shown its practical use in mechanizing certain
% mathematical proofs \cite{4 color problem}, verifying specifications
% \cite{something}, and mechanizing metatheories of programming languages.
% However, verified programming, or \emph{``correct by construction''} approach
% to programming, is yet far from being practical for everyday programming tasks,
% because proof assistants were designed for aiding formal proof generation,
% not for writing executable programs.

% There are basically two ways of
% \emph{verified programming} in such proof assistants: either
% (1) extract program exploiting Curry-Howard correspondence (shallow embedding)
% or
% (2) write your own verified compiler for your object language (deep embedding).

% since the generated proof term has a notion executable
% semantics (shallow embedding)
% (\ie, cut-elimination corresponds to 
% reduction of proof terms).

% \subsection{Functional programming languages}
% 
% \emph{Functional programming languages} based on type-safe (\ie, if a term
% evaluates to a value, the value has the same type as the original term)
% polymorphic $\lambda$-calculi is another major application of type theory.
% Many popular functional programming languages, such as Haskell\cite{Haskell 98}
% and ML, are based on the Hindley-Milner type system, which is not only
% \emph{type-safe} but also \emph{type-inferable}. Type inference makes type-safe
% programming easy and productive for everyday programming tasks since programmers
% are free from tedious type annotations.
% 
% But, fine-grained properties are not expressible in these functional languages
% due to the lack of dependent types. Even if those fine-grained properties were
% somehow expressible, the type system of these languages cannot give guarantee
% those properties. Recall that general purpose programming languages are neither
% strongly normalizing nor logically consistent, because they are capable of
% expressing diverging computations by design.

%% \subsection{Indexed datatypes}
%% 
%% Type-indexed datatypes
%% 
%% Term-indexed datatypes
%% 
%% \subsection{Mendler-style recursion combinators}
%% 
%% TODO


\section{Nax by Example}\label{sec:bg}

In this section we introduce programming in our implementation of Nax by providing
examples. An example usually consists of several parts.

\begin{itemize}
\item Introducing data definitions to describe the data of interest. Recursive
data is introduced in two stages. We must be careful to separate parameters
from indices when using indices to describe static properties of data.

\item Introduce macros, either by explicit defintion, or by automatic derivation
to limit the amount of explicit notation that must be supplied by the programmer.

\item Write a series of definitions that describe how the data is to be
manipulated. Deconstruction of recursive data can only be performed with Mendler style operators
to ensure strong normalization.

\end{itemize}

\subsection{Two level types}
Non recurisive datatypes are introduced by the \verb+data+ declaration.
The data declaration can include arguments. The kind and separation of
arguments into parameters and a indices is infered. For example, 
the three no recursive data types, \verb+Bool+, \verb+Either+, and 
\verb+Maybe+, familiar to many functional programmers, are introduced
in a manner almost identical manner to traditional functional languages.
{\small
\begin{verbatim}   
data Bool = False | True
data Either a b = Left a | Right b
data Maybe a = Nothing | Just a
\end{verbatim}}
Kind inference infers the kinds:
(\verb+Bool: *+), (\verb+Either: * -> * -> *+), and 
(\verb+Maybe: * -> *+), where \verb+*+ is the kind of ordinary types that
classify values.

To introduce a recursive type, we first introduce a non recursive datatype
that uses a parameter where the usual recursive components occur. By design,
normal parameters of the introduced type are written first (\verb+a+ in \verb+L+)
and the parameter to stand for the recursive component is written last
(the \verb+r+ of \verb+Nat+, and the \verb+r+ of \verb+L+).
{\small
\begin{verbatim}
data N r = Zero | Succ r        -- the fix point of N will the natural numbers
data L a r = Nil | Cons a r     -- the fix point of L will be polymorphic lists   
\end{verbatim}}

Kind inference infers the following kinds:
(\verb+N: * -> *+),  (\verb+L: * -> * -> *+). 
The recursive type is just the fixpoint of a (perhaps partially applied)
non recursive type. Thus the traditional natural numbers are typed by \verb+Mu[*] N+
and the traditional lists with components of type \verb+a+ are typed by
\verb+Mu[*] (L a)+.  Note that the \verb+Mu+ type operator is itself specialized with a kind argument inside
square brackets (\verb+[ _ ]+). The fixpoint operator (\verb+Mu[k] f+) is well formed only
if the operand \verb+f+ has kind \verb+k -> j+, in which case the (\verb+Mu[k] f+) type
has kind \verb+j+. Since both \verb+N+ and \verb+(L a)+ have kind \verb+* -> *+,
the types  \verb+Mu[*] N+ and \verb+Mu[*] (L a)+ have kind \verb+*+, i.e. they 
are both types, not type constructors.

\subsection{Creating values}

Values of a particular data type are created by use of constructor functions. For
example \verb+True+ and \verb+False+ are constructor (constants) of type \verb+Bool+.
(\verb+Left 4+) is a value of type (\verb+Either Int a+). Values of recursive types
(i.e. those values with types such as (\verb+Mu[k] f+) are formed by using the \verb+In+
expression. Thus \verb+Nil+ has type \verb+L a+, and \verb+(In[*] Nil)+ has type
(\verb+Mu[*] (L a)+). In general, applying the operator \verb+(In[k])+ injects a term of
type \verb+f(Mu[k] f)+ to the recursive type (\verb+Mu[k] f+). Thus a list of Bool could be
created using the term \verb+(In[*] (Cons True  (In[*] (Cons False  (In[*] Nil)))))+. 
A general rule of thumb, is to apply \verb+In[k]+ to terms of non-recursive type
to get terms of recursive type. Writing programs using two level types, and recursive
injections has definite benefits, but it surely makes programs harder to write. Thus
we have provided Nax with a simple but powerful synonym (macro) facility.

\subsubsection{Synonyms, constructor functions, and derivation}

We may codify that some type is the fixed point of another, once and for all, by introducing
a type synonym (macro).

{\small
\begin{verbatim}
synonym Nat = Mu[*] N
synonym List a = Mu[*] (L a)     
\end{verbatim}}

In a similar manner we can introduce constructor functions that create
recursive values without explicit mention of \verb+In+ at their
call sites (potentially many), but only at their site of defintion (exactly once).

{\small
\begin{verbatim}
zero = In[*] Zero
succ n = In[*] (Succ n)

nil = In[*] Nil
cons x xs = In[*] (Cons x xs)
\end{verbatim}}
This is such a common occurence that recursive synonyms and recursive constructor functions
can be automatically derived. With automatic synonym and constructor derivation
using Nax is both concise ans simple. To illustrate, we provide side-by-side comparisons
of Haskell and two different uses of Nax.

\begin{tabular}{l|l|l}
{\em Haskell}  & {\em Nax with synonyms} &  {\em Nax with derivation} \\ \hline
\begin{minipage}[l]{.26\linewidth}
{\small
\begin{verbatim}
data List a 
  = Nil 
  | Cons a (List a)
  
x = Cons 3 (Cons 2 Nil)  
\end{verbatim}}
\end{minipage}

& 

\begin{minipage}[l]{.34\linewidth}
{\small
\begin{verbatim}

data L: * -> * -> * where
  Nil: L a b
  Cons: a -> b -> L a b
  
synonym List a = Mu[*] (L a)
nil = In[*] Nil
cons x xs = In[*] (Cons x xs)

x = cons 3 (cons 2 nil)  
\end{verbatim}}
\end{minipage}

&

\begin{minipage}[l]{.26\linewidth}
{\small
\begin{verbatim}
data L a r 
     = Nil 
     | Cons a r
  deriving fixpoint List

x = cons 3 (cons 2 nil)    
\end{verbatim}}
\end{minipage}

\end{tabular}


\subsection{Mendler operators for non-indexed types}
There are no restrictions on what kind of datatypes
can be defined, or on the construction of values using constructor
functions and \verb+In+ in Nax. To ensure strong normalization, analysis of constructed
terms is restricted to non-recursive types. This manifests in the lack of
a pattern that matches the construction (\verb+In x+). Simple pattern matching functions
over non-recursive data is supported. For example:

{\small
\begin{verbatim}
not True = False
not False = True

unJust0 (Just x) = x
unJust0 Nothing  = 0
\end{verbatim}}
Analysis of recursive data is performed with Mendler style operators. In our
implementation we provide 4 operators \verb+mcata+ (fold or catamorphism), 
\verb+mprim+ (primitive recursion), \verb+mhist+ (courses of values recursion),
and \verb+msfcata+ (fold for recursive types with negative occurrences).

{\small
\begin{verbatim}
length y = 
   mcata y with
     len Nil = 0
     len (Cons x xs) = 1 + len xs

tail x = 
  mprim x with
    tl cast Nil = []
    tl cast (Cons y ys) = (cast ys )
    
factorial x =
  mprim x with
    fact cast Zero = 1
    fact cast (Succ n) = (1 + nat2int(cast n)) * fact n
                      
fibonocci n =
  mhist n with
    fib out Zero = 1
    fib out (Succ n) = case (out n) of
                        Zero -> 1
                        (Succ m) -> fib n + fib m 
\end{verbatim}}


\subsection{Types with static indices}\label{sec:bg:ixty}


\subsection{Mendler operators for indexed types}

A representative example of a term
indexed type is the length indexed list type (often called the vector type).
A regular polymorphic list type ($\sfList\;a$) is parametrized by a type
parameter ($a$), which can be instantiated to a specific type, while the
vector type ($\sfVec\;a\;\{n\}$) has an additional term index ($n$), which
can be instantiated to a specific natural number value. The curly brackets
($\{\cdots\}$) around $n$ is to syntactically distinguish term indices from
other type arguments. 

\subsection{Recursive types of unrestricted polarity but restricted elimination}
\label{sec:bg:recty}
TODO

\subsection{Mendler style iteration and recursion combinators}
\label{sec:bg:Mendler}


\section{Examples}\label{sec:examples}



\section{The Nax language}\label{sec:Nax}

To validate the second part of my thesis -- the Mendler-style can support 
a simple and expressive programming language, we have designed the
programming language Nax.

Nax supports non-recursive datatypes, the recursive type operator
($\mu^\kappa$), and the Mendler-style recursion combinators (\MIt, \MPr,
\McvIt, \McvPr, \MsfIt). The calculi, \Fi\ and \Fixi, discussed
in the previous sections, are theoretically powerful enough to capture
the semantics of both recursive types and Mendler-style recursion schemes.
However, they are far from practical programming languages. Even very simple
datatypes, such as the boolean type, must be defined via impredicative 
Church-style encodings (\eg, an encoding for the boolean type is
$\forall a . a -> a -> a$). Fortunately, it is possible to provide
a more user friendly layer of abstraction over these calculi. Nax supports
datatypes and Mendler-style recursion combinators as primitive constructs of
the language. I will introduce other programmer friendly features of
the Nax language in \S\ref{sec:Nax:tysym}. Then, in \S\ref{sec:Nax:theory},
I will discuss the desired properties of Nax, which will be demonstrated
in the thesis.

\subsection{Type synonyms and fixpoint derivations}\label{sec:Nax:tysym}

Nax was designed with two goals in mind. First it must be directly definable
in terms of these calculi, and it must exhibit a user friendly interface to
programmers. Towards these goals, Nax has a Haskell-like syntax, and
conservatively extends Hindley-Milner type inference. It also supports
type synonyms and fixpoint derivations, which are useful for defining
recursive types more conveniently. The examples in \S\ref{sec:mendler} are
really reformatted Nax programs. If you know Haskell, you will see that
the Nax syntax is similar to Haskell. We have already discussed how Nax
performs type inference (or, type reconstruction) based on user provided
index transformers attached to Mendler-style recursion combinators. I also
mentioned that Nax supports type synonyms, but did not show any details of
the concrete syntax in \S\ref{sec:mendler}. In Figure \ref{fig:naxderiving},
I illustrate the syntax for type synonyms and fixpoint derivations, which make
recursive type definitions more convenient. In the figure I give three versions.
A Haskell version for reference, and the other two versions of Nax programs.
The first Nax version makes explicit the type synonyms (\textit{List})
and constructor functions (\textit{nil} and \textit{cons}).
In the second Nax version we use the fixpoint derivation mechanism.
The two versions define equivalent programs.

In the type synonym version, we abbreviate the fixpoint $\mu^{*}(L\;a)$
using the type synonym $List$, and define constructor functions
$\textit{nil} : a -> \textit{List}\;a$ and
$\textit{cons} : a -> \textit{List}~a -> \textit{List}~a$.
Recall, that we define recursive types using two levels in Nax.  We take
the fixpoint of the base structure (\eg, $L$) to define a recursive type.
The type synonym (\textit{List}), and its constructor functions
\textit{nil} and \textit{cons}, correspond to the Haskell definition of
the recursive datatype (\textit{List}) and its data constructors
(\textit{Nil} and \textit{Cons}).

In the other version using the fixpoint derivation, the derivation syntax
(\textbf{deriving fixpoint} \textit{List}) attached to the definition of
the base datatype ($L$) instructs the Nax implementation to automatically
generate the type synonym (\textit{List}) for the recursive type and
its constructor functions (\textit{nil} and \textit{cons}) with names
derived by dropping the case of the initial alphabetic character.

\begin{figure}
\begin{tabular}{c||c|c}
Haskell  & Nax with synonyms &  Nax with derivation \\
\hline

\begin{minipage}[t]{.28\linewidth}
\small\vspace{.1em}
\textbf{data} \textit{List} $a$\\
$~~~$ $=$ \textit{Nil}\\ 
$~~~$ $\;|\;$ \textit{Cons} $a$ (\textit{List} a)
\vspace{4.9em}\\
$x = \textit{Cons}~3~(\textit{Cons}~2~\textit{Nil}\,)$
\end{minipage} 

& 

\begin{minipage}[t]{.32\linewidth}
\small\vspace{.1em}
\textbf{data} $L : * -> * -> *$ \textbf{where}\\
$~~~$  \textit{Nil}$~~\; :$ $L\;a\;r$\\
$~~~$  \textit{Cons}    $:$ $a -> r -> L\;a\;r$
\vspace{.3em}\\
\textbf{synonym} \textit{List} $a = \mu^{*} (L\;a)$
\vspace{.3em}\\
\textit{nil}$~~~~~~~~~= \In^{*} \textit{Nil}$ \\
\textit{cons}  $x\;xs = \In^{*} (Cons\;x\;xs)$
\vspace{.5em}\\
$x = \textit{cons}~3~(\textit{cons}~2~\textit{nil}\,)$
\end{minipage}

&

\begin{minipage}[t]{.32\linewidth}
\small\vspace{.1em}
\textbf{data} $L : * -> * -> *$ \textbf{where}\\
$~~~$  \textit{Nil}$~~\; :$ $L\;a\;r$\\
$~~~$  \textit{Cons}    $:$ $a -> r -> L\;a\;r$\\
$~~$ \textbf{deriving} \textbf{fixpoint} \textit{List}
\vspace{3.5em}\\
$x = \textit{cons}~3~(\textit{cons}~2~\textit{nil}\,)$
\end{minipage}

\end{tabular}
\caption{Two versions of a recursive type definition of \textit{List} in Nax --
         with a \textit{type synonym} and with a \emph{derivation macro}
         (\cf\ Haskell definition of $List$)}
\label{fig:naxderiving}
\end{figure}

\subsection{Properties of Nax}\label{sec:Nax:theory}

Nax is designed to demonstrate my thesis. As such, the Nax language must
exhibit a number of properties. I list a few of them here.

\paragraph{Type safety:}
Nax, being a typed programming language, should be type safe.
I will prove type safety of Nax in the thesis.

\paragraph{Normalization:}
I will prove normalization of Nax by a reduction preserving embedding into
either \Fi\ or \Fixw, depending on which families of Mendler-style recursion
combinators we decide to include in Nax. Since it is likely that 
primitive recursion is useful in most cases, the eventual target of
the embedding will be \Fixw. But, for simplicity of illustration, I will
initially illustrate an embedding of a subset of Nax, with \MIt\ and \MsfIt\
families only, into \Fi. Then, I will illustrate an embedding of Nax,
including other families, into \Fixw, with gradually increasing complexity.

\paragraph{Type inference:}
There are several properties to show regarding type inference --
soundness and completeness, and most general typing.

I will describe the type inference algorithm for Nax, which is similar to
the $\mathcal{W}$-algorithm for the Hindley-Milner type system, and show
that the type inference algorithm is sound and complete with respect to
the type system of Nax. The proof of soundness and completeness shall be
done in two steps. First, I will define a syntax directed version of
the type system, where the conversion rule is inlined, and show that
the syntax directed system is equivalent to the standard version of
the Nax type system. Then, I will show that the inference algorithm is
sound and complete with respect to the syntax directed system.

I will also show that the type inference algorithm infers the most general type
of a given program with respect to the index transformer annotations provided by
the programmer.




%% \cite{Girard72} dummy citation
%% \cite{AhnShe11}
%% 
%% \citet{Girard72} citet test
%% \citet{AhnShe11} citet test
%% 
%% \citeauthor{Girard72} citeauthor test
%% \citeauthor{AhnShe11} citeauthor test

\section{Embedding Nax into strongly normalzing calculi}
We will prove strong normalization and logical consistency of Nax
by embedding Nax into a calculi known to have theose desired properties.
We have developed System \Fi, which is an extension of \Fw\ with 
erasable term indices, and proved its properties.\footnote{
	We submitted a paper on System \Fi\ on POPL recently.}
We can embed datatypes of Nax and the \textsf{MIt} and \textsf{MsfIt}
combinator families into System \Fi. But, other combinator families
such as \textsf{MPr} is known to be embeddable into a different
calculus, System \Fixw\ \cite{AbeMat04}, when there are no term indices.
We can similarly extend \Fixw\ with erasable term indices -- we call this
calculus \Fixi, which can embed \textsf{MPr} with term indices.
Properties if \Fixi\ needs to be checked but we strongly believe
that the desired properties, \ie, strong normalization
and logical consistency, will hold in \Fixi\ as well as in \Fi.

\section{Implementation}
We used Haskell and its libraries (\eg, unbound \cite{unbound})
to implement to the proptotpye of Nax.

\section{Summary}

\subsection{Ongoing and future work}
Here is the list of ongoing and future work:
\begin{itemize}
\item Correctness of the type inference algorithm
\item Large eliminations
\item Generalized arrow types in the recursion combinators
\item Non-logical language fragments
\end{itemize}

\bibliographystyle{splncs03}
%% \bibliographystyle{splncsnat}
\bibliography{main}

\end{document}
