\section{Summary}\label{sec:summary}
My dissertation will consist of ten chapters.
Tentative titles of the chapters are listed below:
\begin{quote}
\begin{enumerate}[1.]
\item Introduction
\item Polymorphic type systems (System \textsf{F}, System \Fw, Hindley-Milner)
\item Mendler style recursion schemes
\item System \Fi
\item System \Fixi
\item The Nax language
\item Case studies
\item Related work
\item Future work
\item Conclusion
\end{enumerate}
\end{quote}

In the introduction, I will state my thesis and motivation for my work.
Also, I plan to give a summary on the history of typed programming languages
and formal reasoning. in particular, I will cover the subjects including
Curry-Howard correspondence, recursive types, and induction.

Then, I will review System \textsf{F}, System \Fw, and the Hindley-Milner
type system, in order to lead the discussions in the following sections
on Mendler style recursion schemes, System \Fi, System \Fixi, and
the Nax langauge.

In the case studies section, I will demonstrate larger examples of Nax programs
to demonstrate the expressiveness of Nax. One of the examples I have in mind
is normalization by evaluation.

In thre related work section, I will provide further details on what I have
summeraized in the introduction secction, and also discuss some recent work
related to my approach.

Then, I will discuss future work, which will mainly be discussion on further
extensions to the Nax language, and their ramifications to
the target calculi and the type inference algorithm.

Finally, I will summarize and conclude my thesis.

%% \paragraph{Achivements, progress, and plans:}
%% TODO

